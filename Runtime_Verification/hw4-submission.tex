\documentclass{article}

\usepackage[margin=3cm]{geometry}
\usepackage{amsmath}
\usepackage{amssymb}
\usepackage{xspace}
\usepackage{enumerate}
\usepackage{listings}
\lstset{
  basicstyle=\ttfamily,
  mathescape
}

\title{\Large\bfseries CS 598: Runtime Verification \\
Spring 2017 \\
Homework 4}
\author{Chiao Hsieh, chsieh16@illinois.edu}

\begin{document}
\maketitle

\begin{enumerate}

\item The garbage collection technique discussed in this chapter, specifically in Section~17.4,
has been developed several years before the foundations of parametric monitoring presented in Chapter~15.
Discuss the garbage collection technique and prove its correctness with respect to the algorithm in Figure~15.4.

Ans.

In order to prove the correctness of the garbage collection w.r.t the algorithm,
we describe the proof informally by discussing the usage of $\Delta$ at line 3,
10, and 17 in Figure~15.4.

\newcommand{\pmapsto}{\ensuremath{\rightharpoondown}}

The $\Delta(\theta_{max})$ at line 3 is for searching $max(\theta]_{\Theta}$
by searching $\theta_{max} \in dom(\Theta)$ s.t $\theta_{max} \sqsubset \theta$ in reverse post-order.
The incorrect garbage collection means that the monitor for $max(\theta]_{\Theta}$
is unnecessary (and later garbage-collected) but some 
$\theta' \sqsubset max(\theta]_{\Theta}$ is not,
and it leads us to find wrong $\theta_{max}$. 
According to Section 17.4, a monitor is marked unnecessary only when an instantiation
$v_i$ of a parameter $x_i$ is already garbage-collected in the last event $e_m$
stored in the monitor.
Once an instantiation $v_i$ is proved unnecessary,
it's impossible that newly arrived event $e\langle\theta\rangle$ will have
$\theta(x_i) = v_i$.
Therefore, the garbage-collection won't cause incorrectness here.

The $\Delta(\theta_{comp} \sqcup \theta)$ at line 10 is to help build new monitors
from existing compatible $\theta_{comp}$.
The incorrect garbage collection means it deletes the monitor for
$\theta_{comp} \sqcup \theta$ incorrectly, and therefore build a new monitor again.
This is also impossible due to the same reason that,
once an instantiation $v_i$ is proved unnecessary, it should not appear again.
Hence if some $\theta_{comp} \sqcup \theta$ which contains $v_i$ is
garbage-collected, it won't be constructed again.

The $\Delta(\theta')$ with $\theta' \in \{\theta\} \cup \mathcal{U}$ at line 17
is to update the states for monitors.
The incorrect garbage collection would delete the monitors supposed to be updated
before here.
There are 2 cases to consider. First case is for those monitors created at line 7
or line 11 when processing current event.
It's trivially correct since newly created monitor is impossible to be 
garbage-collected in its initially state.
Second case is that the monitor is marked unnecessary after previous events.
This is also impossible because a removed instantiation $v_i$ will never
appear again.

In summary, we proved that, for every usage of $\Delta$ in Figure 15.4,
the garbage collection works correctly.

\item The causal maximal model techniques presented in Chapters~19 and 20 assumed
that nothing is known about the code that produced the observed trace $\tau$
except that it can produce $\tau$ when executed.
In additional to the usual concurrent object events considered in the original
technique in Chapter~19,
the technique in Chapter~20 assumes that simple branch events of the form 
$branch(t)$ are logged in $\tau$ whenever thread $t$ makes a control-flow choice.
 In practice, such a choice is given by the result of some computation,
such as, for example, evaluating some boolean expression, say $b$.

This exercise requires you to extend the maximal causal model technique to take
into account additional knowledge about the execution of the program,
namely when the evaluation of the boolean expression $b$ starts and when it ends.
Specifically, suppose that the events $branch(t)$ are replaced with pairs of
events $branch(t, begin)$ and $branch(t,end)$ stating when the expression $b$
starts being evaluated and, respectively, when its evaluation has terminated.

Discuss how you extend or modify at least each of the following notions and
results: feasible set of executions, feasibility closure, soundness theorem,
maximality theorem, encoding as SMT formula.

\newcommand{\Rest}{\ensuremath{\upharpoonright}}
\newcommand{\F}{\ensuremath{\mathcal{F}}}
Ans.

If $branch(t, begin)$ and $branch(t,end)$ are also encoded in event traces,
all prefixes of a consistent trace must contains equally many or one more
$branch(t, begin)$ than $branch(t, end)$ because the evaluation must begin before end,
and some begins may not match ends when the trace stop at the middle of evaluating $b$.
Note that we assume branch evaluations cannot be nested and 
a $branch(t, end)$ always matches the closest $branch(t, begin)$ in the past.
However, these are not necessarily true since one can invoke a function call during
evaluation and lead to branches in that function call,
and, if exception handling happens during evaluation, $branch(t, end)$ may not be
emitted for some $branch(t, begin)$.   

Formally, for a given thread $t$ and $\tau\Rest_{t,branch} = e_1 e_2 \cdots e_n$,
odd indexed events $e_{2k+1} = branch(t, begin)$ and even indexed events
$e_{2k+2} = branch(t, end)$ for all $k > 0$.


\textbf{Feasible Set of Executions}

We replace the $branch$ axiom of \emph{Local Determinism} in Section 20.2.3
with these 2 new axioms for $branch$ events.
\begin{itemize}
\item \textbf{Branch Begin:}\newline
	If $op(e_1) = branch(t, begin)$ and 
	$\tau_1\Rest_{t,read} = \tau_2\Rest_{t,read}$ and $\tau_2 e_1$ is consistent
	then $\tau_2 e_1 \in \F$
\item \textbf{Branch End:}\newline
	If $op(e_1) = branch(t, end)$ and 
	$\tau_1\Rest_{t,read} = \tau_2\Rest_{t,read}$ and $\tau_2 e_1$ is consistent
	then $\tau_2 e_1 \in \F$
\end{itemize}

A feasible set of executions, \F, must satisfy \emph{Prefix-closeness} and
the modified \emph{Local Determinism}.
We now discuss the feasibility of traces constructed with the new $branch$ axioms.

For \textbf{Branch Begin}, same as the argument in Section 20.2, the trace $\tau_2$ must reach
the exact same branch condition as $\tau_1 e_1$ when every $read$ returns the
same value.
Further, we also require that $\tau_2 e_1$ is consistent;
hence we won't append another $branch(t, begin)$ if it causes nested branch evaluation.
Therefore, we can safely append $e_1 = branch(t, begin)$ to $\tau_2$
because $\tau_2$ will reach the exact branch that $e_1$ is from in the program.

For \textbf{Branch End}, in order for $b$ evaluates to the same value,
all $read$ values between $branch(t, begin)$ and $branch(t, end)$ should be the same.
Also we require $\tau_2 e_1$ consistent so the appended $branch(t, end)$ will match
the last $branch(t, begin)$.
Therefore, the traces inferred from these 2 axioms must be feasible.

\textbf{Feasibility Closure}

The feasibility closure $feasible(\tau)$ of a given trace $\tau$ is the smallest set
of traces that contains $\tau$ and feasible as defined in Section 19.3.  

\textbf{Soundness Theorem}

To prove the soundness, we just have to modify the $branch$ part of the proof for
Theorem 31 in Section 20.2.
For both $branch(t, begin)$ and $branch(t, begin)$,
the proof is exactly the same as the original proof for $branch$
because $\tau_1\Rest_{t,read} = \tau_2\Rest_{t,read}$ ensures
that the same assignment to symbolic values is still valid for $\tau_2 e_1$.

\textbf{Maximality Theorem}

To prove maximality, we modify the proof for Theorem 32 in Section 20.2.4.
In order to generate a witness CONC trace program which doesn't contain any 
$\tau' \notin feasible(\tau)$,
we use the same same trick to add an if statement with the condition
to ensure the same $read$ values for every $branch(t, begin)$ and
$branch(t, end)$ events.
For other kind of events, we follow the same way to construct the program
as described in Theorem 32. 
Any trace $\tau'\notin feasible(\tau)$ cannot be produced by the program

\textbf{Encoding as SMT formula}
In order to support encoding constraints as described in Section 20.3.2,
we encode $branch(t, begin)$ and $branch(t, end)$ the same way.
$$
\Phi_{cf}(e) = \bigwedge_{r \in \tau_e \Rest_{t, read}} \Phi_{cf}(r),
\quad op(e) \in \{branch(t, begin), branch(t, end)\}
$$

\item Same as Exercise 30, but suppose that now you know the entire expression $b$.
That is, replace events $branch(t)$ with $branch(t,b)$,
where $b$ is a Boolean expression over (some) shared variables.
Assume that $b$ contains nothing but the usual arithmetic expression constructs
and shared variables.

Ans.

The consistency regarding branch events is still the same to ensure $branch(t, b, begin)$
and $branch(t, b, begin)$ are paired properly except for the last $branch(t, b, begin)$.


\textbf{Feasible Set of Executions}

If we know that $b$ is only over shared variables (concurrent objects) and
no local variable of the thread is involved,
we can further relax \emph{Local Determinism} with weaker constraints.
We use the notations $br\_seg(\tau, e)$ to denote that,
in a given trace $\tau$ w.r.t a branch event $e \in {begin, end}$,
$br\_seg(\tau, e)$ is the segment in $\tau$ between $e$ and closest
$branch(t, b, begin)$ before $e$ or $branch(t, b, end)$ after $e$.

\begin{itemize}
\item \textbf{Branch Begin:}\newline
	Let $end_1 \in \tau_1\Rest_{t,branch(t,b,end)}$ and
	$end_2 \in \tau_2\Rest_{t,branch(t,b,end)}$\newline
	If $op(e_1) = branch(t, b, begin)$ and for all
	$br\_seg(\tau_1, end_1)\Rest_{t,read} = br\_seq(\tau_2, end_2)\Rest_{t,read}$
	and $\tau_2 e_1$ is consistent
	then $\tau_2 e_1 \in \F$
\item \textbf{Branch End:}\newline
	Let $begin_1 \in \tau_1\Rest_{t,branch(t,b,begin)}$ and
	$begin_2 \in \tau_2\Rest_{t,branch(t,b,begin)}$\newline
	If $op(e_1) = branch(t, b, end)$ and for all
	$br\_seg(\tau_1, begin_1)\Rest_{t,read} = br\_seg(\tau_2, begin_2)\Rest_{t,read}$
	and $\tau_2 e_1$ is consistent
	then $\tau_2 e_1 \in \F$
\end{itemize}

For \textbf{Branch Begin}, we argue that the trace $\tau_2$ must reach
the exact same branch condition as $\tau_1 e_1$ because every branch evaluation
in $\tau_1$ happened before current branch $e$ should match those evaluation in $\tau_2$.
Therefore, the program must take the same path since each branch condition $b$ 
evaluate to same Boolean value in $\tau_1$ and $\tau_2$ since every $read$ value during
branch evaluation matches and no local variable is in $b$.
This is weaker than considering all $read$ ever happened since we only check $read$ in
branch evaluation segments.
 
For \textbf{Branch End}, the same argument can be applied.
Therefore, the traces inferred from these 2 axioms must be feasible.

\textbf{Soundness Theorem}

For both $branch(t, b, begin)$ and $branch(t, b, begin)$,
we know that the assignment to symbolic values $\theta_1$ and $\theta_2$
can be different.
However, because $br\_seg(\tau_1, br_1)\Rest_{t,read} = br\_seq(\tau_2, br_2)\Rest_{t,read}$
ensures the assigned value to variables used in $b$ still matches,
the same assignment $\theta_2$ to symbolic values can guarantee $\tau_2 e_1$
goes same direction as $\tau_1 e_1$.

\textbf{Maximality Theorem}

To construct the witness CONC program which doesn't contain any 
$\tau' \notin feasible(\tau)$,
we have to build checks to ensure for all
$br\_seg(\tau_1, br_1)\Rest_{t,read} = br\_seq(\tau_2, br_2)\Rest_{t,read}$.
Assume given trace $\tau$ where $\tau \Rest_{t, branch} = begin_1end_1begin_2end_2\cdots$,
we build a check for $begin_i$ and $end_i$ by
\begin{lstlisting}
$begin_i$ := $end_{i-1}$;
if($begin_i$){
  ... // Reading Values
  $r_i$ := $read(x_i)$;
  ... // End of reading
  $end_i$ := $begin_i$ && $(r_1 = v_1)$ && $(r_2 = v_2) \cdots$;
  if($end_i$)
     $b$ := $end_i$;
}
\end{lstlisting}
where $r_j = v_j$ ensures the $read$ values in evaluating $b$ should remain the same.

\textbf{Encoding as SMT formula}

$$
\Phi_{cf}(e) = \bigwedge_{r \in br\_seg(t, e) \Rest_{t, read}} \Phi_{cf}(r),
\quad op(e) \in \{branch(t, b, begin), branch(t, b, end)\}
$$

\end{enumerate}

\end{document}