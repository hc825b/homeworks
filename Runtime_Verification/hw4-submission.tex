\documentclass{article}

\usepackage[margin=3cm]{geometry}
\usepackage{amsmath}
\usepackage{amssymb}
\usepackage{xspace}
\usepackage{enumerate}

\title{\Large\bfseries CS 598: Runtime Verification \\
Spring 2017 \\
Homework 4}
\author{Chiao Hsieh, chsieh16@illinois.edu}

\begin{document}
\maketitle

\begin{enumerate}

\item The garbage collection technique discussed in this chapter, specifically in Section~17.4,
has been developed several years before the foundations of parametric monitoring presented in Chapter~15.
Discuss the garbage collection technique and prove its correctness with respect to the algorithm in Figure~15.4.

Ans.

\item The causal maximal model techniques presented in Chapters~19 and 20 assumed
that nothing is known about the code that produced the observed trace $\tau$
except that it can produce $\tau$ when executed.
In additional to the usual concurrent object events considered in the original
technique in Chapter~19,
the technique in Chapter~20 assumes that simple branch events of the form 
$branch(t)$ are logged in $\tau$ whenever thread $t$ makes a control-flow choice.
 In practice, such a choice is given by the result of some computation,
such as, for example, evaluating some boolean expression, say $b$.

This exercise requires you to extend the maximal causal model technique to take
into account additional knowledge about the execution of the program,
namely when the evaluation of the boolean expression $b$ starts and when it ends.
Specifically, suppose that the events $branch(t)$ are replaced with pairs of
events $branch(t, begin)$ and $branch(t,end)$ stating when the expression $b$
starts being evaluated and, respectively, when its evaluation has terminated.

Discuss how you extend or modify at least each of the following notions and
results: feasible set of executions, feasibility closure, soundness theorem,
maximality theorem, encoding as SMT formula.


Ans.

\item Same as Exercise 30, but suppose that now you know the entire expression$b$.
That is, replace events $branch(t)$ with $branch(t,b)$,
where $b$ is a Boolean expression over (some) shared variables.
Assume that $b$ contains nothing but the usual arithmetic expression constructs
and shared variables.

Ans.

\end{enumerate}

\end{document}