\documentclass{article}

\usepackage[margin=3cm]{geometry}
\usepackage{amsmath}
\usepackage{amssymb}
\usepackage[noend]{algorithm2e}
\usepackage{xspace}
\usepackage{enumerate}
\usepackage{tikz}
\usetikzlibrary{positioning, automata}
\usepackage{caption}
\usepackage{subcaption}
\usepackage{float}

\title{\Large\bfseries CS 598: Runtime Verification \\
Spring 2017 \\
Homework 3}
\author{Chiao Hsieh, chsieh16@illinois.edu}

\begin{document}
\maketitle

\begin{enumerate}
\item Theorem 19 showed that only two temporal operators are actually needed.
All the others can be regarded as derived operators.
This is similar to how only two logical connectives are actually needed in 
propositional logic, say $\neg$ and $\land$.
Therefore, it is tempting to pick only two temporal operators and then desugar
all the others to them.
Explain the drawbacks of doing so in terms of the time/space complexity of the
resulting monitors,
addressing each of the discussed monitoring approaches separately:
rewriting (Section 10.4), monitor generation (Section 10.5) including
post-generation optimizations (Section 10.5.3),
as well as directly-optimal monitor generation (Section 10.7).

Ans.


\item Consider the four derived operators in Section 11.3.
Modify the \textsc{ptCaRet} monitor synthesis algorithm in Section 11.4.2,
as well as the related results that yield its correctness,
to generate monitoring code directly for these derived operators,
without first desugaring them to the standard operators.

(do only $\overline{S_c}$)

Ans.


\item Manually execute the CFG monitoring algorithm in Figure 12.6 for the
\texttt{SafeLock} property in Section 12.1.1 with its LALR(1) table in Table 12.2,
on the following observed trace:
\texttt{begin begin acquire acquire release release end begin acquire release end
end begin end}.
Explain only the major steps.
The purpose of this exercise to is to demonstrate that you understand the CFG
monitoring algorithm, including the handling of the \$ event.

Ans.

\item Revise the pattern-match automata and rewriting algorithms in Figures~13.4
and 13.8, respectively.
In class, it was suggested that there was a problem in these algorithms with how
pattern-matching was done.
Can you find a problem, or you think they are correct?
Explain with enough detail to show that you understand the algorithm.

Ans.


\item String rewriting is more general than CFG monitoring.
Specifically, we can associate an equivalent string rewrite system to any CFG,
and then use the former to do monitoring. 
Describe the general procedure to translate a CFG to a string rewrite system.
Then apply the general procedure manually to the \texttt{SafeLock} CFG property
in Exercise 22,
and redo Exercise 26 with the resulting SRS.
How does the resulting SRS monitor compare with the one in Exercise 26?
How about the one in Exercise 22?
Comment on the asymptotic complexity of monitoring CFG (Chapter 12) vs monitoring
the corresponding SRS.

Ans.


\newcommand{\ALTL}{\texttt{ALTL}\xspace}
\item Definition 62 shows how to translate \ALTL to future-time LTL.
Give an equivalent translation of \ALTL to past-time LTL,
and exemplify it on the \ALTL formula in Example 18.
Compare monitoring \ALTL using the algorithm described in Section 14.4,
with monitoring the corresponding past-time LTL using the optimized
algorithm in Chapter 10.

Ans.

\end{enumerate}

\end{document}