\documentclass{article}

\usepackage[margin=3cm]{geometry}
\usepackage{amsmath}
\usepackage{amssymb}
\usepackage{xspace}

\title{\Large\bfseries CS 598: Runtime Verification \\
Spring 2017 \\
Homework 1}
\author{Chiao Hsieh, chsieh16@illinois.edu}

\begin{document}
\maketitle

\newcommand{\FinTraces}{\ensuremath{\Sigma^*}\xspace}
\newcommand{\InfTraces}{\ensuremath{\Sigma^\omega}\xspace}
\newcommand{\prefixes}{\ensuremath{\mathtt{prefixes}}\xspace}
\newcommand{\Nat}{\ensuremath{\mathbb{N}}\xspace}
\newcommand{\Real}{\ensuremath{\mathbb{R}}\xspace}
\newcommand{\Bool}{\ensuremath{\mathbb{B}}\xspace}


\begin{enumerate}
\item Show that each of the following sets have cardinal $c$: streams
(i.e., infinite sequences) of Booleans; streams of natural numbers;
streams of real numbers; closed intervals of real numbers;
open intervals of real numbers; closed sets of real numbers;
open sets of real numbers.

Ans.

\textbf{Stream of Booleans}

We can easily prove $|\Bool^\omega| \leq c$ by mapping
each $\alpha \in \Bool^\omega$ to a real number $0.\bar{\alpha}$.
The function is trivially injective.

To prove $|\Bool^\omega| \geq c$, we create a function
$\varphi : (0,1) \mapsto \Bool^\omega$ which $\alpha = \varphi(r)$ is
the sub-string after the point of the binary representation of $r$,
i.e., $r = 0.\bar{\alpha}$.
If the binary string of $r$ is finite,
we can just append $0^\omega$ after the end.
$\varphi$ is injective since the binary representation is unique.
Hence, we infer that $|\Bool^\omega| \geq c$.

\textbf{Stream of natural numbers}

$|\Nat^\omega|\geq|\Bool^\omega| = c$ is trivially true since $|\Bool|<|\Nat|$.

To prove $|\Nat^\omega| \leq c$, we can consider an intermediate sets which is
the set of all possible functions $(\Nat \mapsto \Nat)$.
$\varphi: \Nat^\omega \mapsto (\Nat \mapsto \Nat)$ is a bijective function since
a sequence is equivalent to a function from index to number.
Hence, $|\Nat^\omega| = |(\Nat\mapsto\Nat)| \leq |\wp(\Nat\times\Nat)| = 2^{\aleph_0} = c$

\textbf{Stream of real numbers}

$|\Real^\omega|\geq|\Bool^\omega| = c$ is trivially true since $|\Bool|<|\Real|$.
To prove $|\Real^\omega| \leq c$, we can consider an intermediate sets which is
the set of all possible functions $(\Nat \mapsto \Real)$.
Also we know there is a bijective function from \Real to $\Bool^\omega$,
i.e., $(\Nat \mapsto \Bool)$.
Therefore, we construct $\varphi: \Real^\omega \mapsto (\Nat \mapsto (\Nat \mapsto \Bool))$,
a bijective function.

Hence, $|\Real^\omega| = |(\Nat\mapsto(\Nat\mapsto\Bool))| \leq |\wp(\Nat\times\Nat\times\Bool)| = c$

\textbf{Closed intervals of real numbers}

For any closed intervals, we can find exactly a pair of real numbers,
$(r1, r2) \in \Real\times\Real$ to represent it.
Hence, its cardinality $\leq |\Real\times\Real| = c\cdot c = c$.

For any real number $r$, we can find a unique element $[r, r]$.
Hence, its cardinality $\geq |\Real| = c$

\textbf{Open intervals of real numbers}

\bigskip

\textbf{Closed sets of real numbers}

\bigskip

\textbf{Open sets of real numbers}

\bigskip

\item The $\prefixes: \wp(\FinTraces) \to \wp(\FinTraces)$ in Definition 1
is a closure operator: it is extensive ($P \subseteq \prefixes(P)$),
monotone ($P \subseteq P'$ implies $\prefixes(P) \subseteq \prefixes(P')$),
and idempotent ($\prefixes(\prefixes(P)) = \prefixes(P)$).

Ans.

Extensive:

For any given trace $w \in P$, $w \in \prefixes(w)$ and
$\prefixes(w) \subseteq \prefixes(P)$ by definition of \prefixes.
Hence, $P \subseteq \prefixes(P)$

\medskip

Monotone:

Assuming \prefixes is not monotonic, that is,
$\prefixes(P) \not\subseteq \prefixes(P')$.
Then we can find at least one prefix $p$ of a word $w \in P$ s.t.
$p \notin \prefixes(P')$.
However, since $w \in P \subseteq P'$,
$p \in \prefixes(w) \subseteq \prefixes(P')$ by definition.
Therefore, no such prefix $p$ exists. Prove by contradiction.

\medskip

Idempotent:

Since \prefixes is extensive, we already know
$\prefixes(P)\subseteq\prefixes(\prefixes(P))$.
We now prove $\prefixes(P)\supseteq\prefixes(\prefixes(P))$.

For any given $p \in \prefixes(\prefixes(P))$,
there must be a prefix $q \in \prefixes(P)$ s.t. $p \in \prefixes(q)$.
Similarly, there must be a trace $w \in P$ s.t. $q \in \prefixes(w)$.
By definition of $\prefixes$, $\exists w'. qw' = w \land \exists w''. pw'' = q$.
Therefore, we can infer that $pw'w'' = w$; hence $p \in \prefixes(w)$.
Since $w \in P$, we conclude that $p \in \prefixes(P)$. 

\item (Counter-)Example 4 showed that the finiteness of $\Sigma$ was necessary
in order for Proposition 2 to hold, by defining a property $P$ over
$\Sigma = \Nat\cup\{\infty\}$ in which all non-empty words start with $\infty$.
Can we remove $\infty$ from $\Sigma$ and from all the words in $P$?
Why, or why not?

Ans.

If we don't introduce $\infty$, then property $P$ is defined as
$$
P = \{\epsilon\} \cup \{n(n-1)\dots(m+1)m \mid 0 \leq m \leq n+1 \}
$$
$P$ is still an infinite set, and the only persistent subset of $P$
is still $\emptyset$.
The argument is the same. For any persistent subset $P' \subseteq P$,
$P'$ does not contain any trace ending in $0$ since it cannot be continued.
Inductively, we can assume any trace ending with some $n \in \Nat$ is also
not in $P'$ and infer that all traces ending with $n+1$ are not in $P'$.
Hence, we prove that the only persistent subset of $P$ is $\emptyset$
and $P^{\circ} = \emptyset$.


\item Same like Exercise 3, but for Example 5 instead of Example 4.

Ans.

Following the proof in Example 5, we know that $P_i$ for $P$ without $\infty$ is
$$
P_i = \{\epsilon\} \cup \{n(n-1)\dots(m+1)m \mid i \leq m \leq n+1 \}
$$
We can still prove that $\epsilon \in \bigcap_{i\geq 0} P_i$.
First, $0 \in P_0$ trivially.
Second, since we know $i \in P_i$ for any $i$ when we set $m = n = i$,
we know $\epsilon \in P_{i+1}$ by definition of $P^-$.
Therefore, we proved that $\bigcap_{i\geq 0} P_i \neq P^{\circ}$ which is empty.


\item Prove Proposition 8.

Ans.

\begin{itemize}
\item $\square \prefixes(\square P) = \square P$ for any $P \subseteq \FinTraces$

\medskip
For $\square \prefixes(\square P) \subseteq \square P$:

Since $\prefixes(u) \subseteq P$ for every $u \in \square P$,
we know $\prefixes(\square P) \subseteq P$.
Now because $\square$ is monotonic by definition,
$\square \prefixes(\square P) \subseteq \square P$.

\medskip
For $\square \prefixes(\square P) \supseteq \square P$:

We first assume there exists $u \in \square P$ such that
$u \notin \square \prefixes(\square P)$.
By definition of $\square$, $\prefixes(u) \not\subseteq \prefixes(\square P)$. 
However, $\prefixes(u) \subseteq \prefixes(\square P)$ because $u \in \square P$.
Therefore, we proved it by contradiction.

\item $Q$ is a safety property iff $\square \prefixes(Q) = Q$

\medskip
$Q$ is a safety property $\implies$ $\square \prefixes(Q) = Q$:

Since $Q$ is a safety property, there exists $P$ s.t $\square P = Q$.
We therefore have to prove $\square \prefixes(\square P) = \square P$
which is proven.

\medskip
$Q$ is a safety property $\impliedby$ $\square \prefixes(Q) = Q$:

We can find $P = \prefixes(Q)$ so that $\square P = Q$.
By definition, $Q = \square P \in Safety_{\square}$.
 

\end{itemize}


\item The ``closure under limits'' operation in Definition 7 is indeed a closure
operator on $\InfTraces$: it is extensive ($Q \subseteq \overline{Q}$),
monotone ($Q\subseteq Q'$ implies $\overline{Q} \subseteq \overline{Q'}$),
and idempotent ($\overline{\overline{Q}} = \overline{Q}$).

Ans.

Extensive:

For any infinite trace $q \in Q$,
we can always choose an infinite sequence $u^{(1)}=q, u^{(2)}=q, u^{(3)}=q, \dots$
s.t. $q = \lim_i u^{(i)} \in \overline{Q}$.
Therefore, $Q \subseteq \overline{Q}$.

\medskip

Monotonic:

For any sequence of traces $u^{(1)}, u^{(2)},\dots$,
$u^{(i)} \in Q$ implies $u^{(i)} \in Q'$ because $Q \subseteq Q'$,
so $\lim_i u^{(i)} \in \overline{Q}$ implies
$\lim_i u^{(i)} \in \overline{Q'}$.
Therefore, $\overline{Q} \subseteq \overline{Q'}.$

\medskip

Idempotent:

Since $\overline{\overline{Q}} \supseteq \overline{Q}$ is proven by extensiveness,
we only have to prove $\overline{\overline{Q}} \subseteq \overline{Q}$.

Assuming there is a trace $u \in \overline{\overline{Q}}$ such that
$u \notin \overline{Q}$.
There exists an infinite sequence of traces where each $u^{(i)} \in \overline{Q}$
and $u = \lim_i u^{(i)}$.
Since each $u^{(i)}$ belongs $\overline{Q}$,
there is an infinite sequence of traces where each $u^{(i)(j)} \in Q$ s.t.
$u^{(i)} = \lim_j u^{(i)(j)}$.
We now can construct a new infinite sequence $u^{(1)(1)}, u^{(2)(2)}, \dots$
by selecting those $u^{(k)(k)} \in Q$,
and we argue that $u = \lim_k u^{(k)(k)}$.
By definition, we know the following is true.
$$
\begin{array}{rl}
      & \forall m\geq 0. \exists n\geq 0 . \forall k \geq n.
        u^{(k)(k)}_1 u^{(k)(k)}_2\dots u^{(k)(k)}_m = u^{(k)}_1 u^{(k)}_2 \dots u^{(k)}_m \\
\land & \forall m\geq 0. \exists n\geq 0 . \forall k \geq n.
        u^{(k)}_1 u^{(k)}_2\dots u^{(k)}_m = u_1 u_2 \dots u_m
\end{array}
$$

Although existential quantifer is not distributive over $\land$,
we argue that one can always select the larger $n$ so that
both clauses will be valid.
Therefore, we can infer that
$$
\forall m\geq 0. \exists n\geq 0 . \forall k \geq n.
u^{(k)(k)}_1 u^{(k)(k)}_2\dots u^{(k)(k)}_m = u_1 u_2 \dots u_m
$$
and prove that $u = \lim_k u^{(k)(k)} \in \overline{Q}$ which
contradicts our assumption. 

\item Define a canonical monitor for the property
\begin{center}
``A file can only be accessed if it is open.''
\end{center}
That is, the file can only be accessed if it was opened at some moment
in the past and it was not closed since then.
Suppose $\Sigma$ consists of the events/actions $\{o,a,c\}$,
where $o$ stands for ``file open'', $a$ for ``file access'',
and $c$ for ``file close''.

Ans.

\newcommand{\monitor}{\ensuremath{\mathcal{N}}\xspace}
For this property, we only need 2 states to indicate
when the file is closed (state 0) and open (state 1).
The canonical monitor \monitor for the property is defined as follow.
$$
\begin{array}{l}
\monitor : \Nat \times \Sigma \rightharpoondown \Nat \\
\monitor(0, c) = 0\\
\monitor(0, o) = 1\\
\monitor(1, c) = 0\\
\monitor(1, o) = 1\\
\monitor(1, a) = 1
\end{array}
$$
Following the definition, $S_\monitor = \{0, 1\}$ and
$\mathcal{L}(\monitor) = (c+o(o+a)^*c)^*(\epsilon+o(o+a)^*)$

\end{enumerate}

\end{document}