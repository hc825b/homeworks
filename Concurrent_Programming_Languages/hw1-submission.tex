\documentclass{article}

\usepackage[margin=1.5cm]{geometry}
\usepackage{amsmath}
\usepackage{amssymb}
\usepackage{xspace}

\title{\Large\bfseries
CS 524: Concurrent Programming Languages \\
Fall 2017 \\
Homework 1}
\author{Chiao Hsieh, chsieh16@illinois.edu}

\begin{document}
\maketitle

\begin{enumerate}
\item Consider a computation path from an actor configuration.
Define \textit{global time} using the order of message receives along the
path\textemdash global time is initially zero and is incremented each time a
message is received by some actor.
Define \textit{delivery} time as the difference between the global time of the
transition which causes a given message $(a, m)$ to be sent and the time of
the event in which the message $m$ is received by $a$.
You may assume that the oldest message of the form $(a, m)$ is received before
any subsequent message $m$ sent to $a$.
The order of events on computation paths may be constrained by one of the
following:

\textbf{Local Boundedness Axiom}:
The set of delivery times of all messages which share the same target $a$ has
an upper bound.

\textbf{Global Boundedness Axiom}:
The set of delivery times of all messages has an upper bound.

\textbf{Equiprobable Delivery Axiom}:
All pending events are equally likely to occur next.

Prove or disprove:
\begin{enumerate}
\item Local boundedness implies fairness.

Ans.

Yes. If each message sent to an actor $a$ has a bounded delivery time $B$, we 
know that, when transition $<\texttt{rec}: a, m>$ is enabled at time $t_0$, 
there are at most $B-1$ messages $a$ can receive before accepting $(a,m)$ by 
definition of local boundedness. Therefore, either $(a,m)$ is received by $a$ 
before $t_0+B$, or $<\texttt{rec}: a, m>$ is disabled forever. In either 
case, the computation path is fair. Hence, we conclude that local boundedness 
implies fairness.

\item Local boundedness implies global boundedness.

Ans.

No. Consider Actor system below,
\begin{verbatim}
A2 = rec(lambda b. lambda c. lambda k. lambda n. lambda m.
         if(k < n,
            ready(A2(c, k+1, n)),
            seq(letactor {succ := A2(succ, 0, n+1)}
                    for(i=0; i<n+1, i++) send(succ, nil),
                ready(sink))))
letactor {a2 := A2(a2, 0, 1)} send(a2, nil)
\end{verbatim}
In this system, each actor will increment \texttt{k} when receiving a message 
until \texttt{k>=n}, and it will create a new actor, \texttt{succ}, and send 
\texttt{n+1} messages to \texttt{succ}. For each actor, we know exactly 
\texttt{n} messages are sent from its creator to itself, and all other actors 
already became \texttt{sink}, so there is no other message. The worst case 
delivery time for each actor therefore will be locally bounded by an integer 
\texttt{n}. However, a newly created actor can always have a larger bound, so 
there is no global upper-bound for infinitely many actors. Hence we found a 
counter-example for the statement.

\item Global boundedness implies fairness.

Ans.

By definition, global boundedness implies local boundedness. From (a), we 
know global boundedness implies fairness.

\item If global boundedness is assumed, unbounded nondeterminism is not
possible.

Ans.

Assuming unbounded nondeterminism is possible, there exists a configuration 
$\kappa$ has unbounded many choices of enabled transitions. 

\item If local boundedness is assumed then unbounded nondeterminism is not
possible.

\end{enumerate}

\item Assume that fairness is equivalent to the property that the probability
of eventual message delivery for all messages is 1. If all transitions from a
configuration are always equiprobable, does fairness hold?

Ans.

Consider the following actor system,
\begin{verbatim}
A3 = rec(lambda b. lambda c. lambda k. lambda m.
         if(m == stop,
            ready(sink),
            seq(for(i=1; i<k*k; i++) send(c, next),
                ready(A3(c,k+1))),
            )
         )
letactor {a3 := A3(a3, 2)} seq(send(a3, next), send(a3,stop))
\end{verbatim}
The behavior of the actor is simple. It becomes a \texttt{sink} whenever 
\text{stop} message is received, or it sends \texttt{k*k-1} messages to 
itself and increment \texttt{k}. Therefore, except for the beginning 2 
messages, there is always $k^2$ total messages if \texttt{stop} message is 
not received; thus the probability of not choosing \texttt{stop} is 
$\frac{k^2 - 1}{k^2}$. We can then calculate the probability that 
\texttt{stop} message is never received.
$$
\begin{aligned}
p &= \frac{1}{2} \cdot \prod_{k=2}^{\infty} \frac{k^2 - 1}{k^2}
   = \frac{1}{2} \cdot \prod_{k=2}^{\infty} \frac{(k-1)(k+1)}{k^2}
   = \frac{1}{2} \cdot \prod_{k=2}^{\infty} \frac{k-1}{k}\cdot\frac{k+1}{k} \\
  &= \frac{1}{2} \cdot \left(\frac{1}{2}\cdot\frac{3}{2}\right)\cdot
                 \left(\frac{2}{3}\cdot\frac{4}{3}\right)\cdot 
                 \left(\frac{3}{4}\cdot\frac{5}{4}\right) \cdots
                 \left(\frac{n-1}{n}\cdot\frac{n+1}{n}\right) \qquad n \to \infty\\
  &= \frac{1}{2} \cdot \frac{1}{2}\cdot
                 \left(\frac{3}{2}\cdot\frac{2}{3}\right)\cdot
                 \left(\frac{4}{3}\cdot\frac{3}{4}\right)\cdot 
                 \left(\frac{5}{4}\cdot\frac{4}{5}\right)\cdots
                 \left(\frac{n}{n-1}\cdot\frac{n-1}{n}\right)\cdot
                 \frac{n+1}{n} \qquad n \to \infty\\
  &= \frac{1}{2} \cdot \frac{1}{2}\cdot 1 \cdot 1 \cdot 1 \cdots 1
                 \cdot \frac{n+1}{n} \qquad n \to \infty\\
  &= \frac{1}{4}\\
\end{aligned}
$$
For this example, the probability of eventual message delivery for \texttt{stop}
is $1-p = \frac{3}{4} \neq 1$. Therefore, fairness is not guaranteed even when
all transitions are always equiprobable.

\item \textbf{Brock-Ackerman Anomaly.} Consider the following two actor
systems (assume the sink behavior ignores all messages):

\begin{enumerate}
\item A system $\alpha_1$ has three actors. The actor $r$ sends two copies of
any message it receives to an actor $s$. For every message $m$ it receives,
$s$ sends the message to an actor called $v$. See code below.
\begin{verbatim}
B-recep = rec(lambda b. lambda c. lambda m.
                seq(send(c,m),
                send(c,m),
                ready(B-recep(c))))

B-eager = rec(lambda b. lambda c. lambda k. lambda m
    if((zero? k)
        seq(send(c,m),
            ready(B-eager(c, 1)),
        seq(send(c,m),
            ready(B-sink)))))

B-viewer = rec(lambda b. ...)

letactor {r = B-recep(s),
          s = B-eager(v, (0, nil)),
          v = B-viewer( )}
    send(r, nil)
\end{verbatim}

\item Another system $\alpha_2$ also has three actors. The actor \texttt{r}
sends two copies of any message it receives to an actor called \texttt{s}. The
actor \texttt{s} sits on the first message \texttt{m1} and when it receives a
second message \texttt{m2}, it sends the two message \texttt{m1} and \texttt{m2}
to an actor \texttt{v}. See code below:
\begin{verbatim}
B-recep = rec(lambda b. lambda c. lambda m
                seq(send(c,m),
                    send(c,m),
                    ready(B-recep(c))))

B-lazy = rec(lambda b. lambda c. lambda k. lambda m1. lambda m.
    if((zero? k)
        ready(B-lazy(c,1,m))
        seq(send(c,m1),
            send(c,m),
            ready(B-sink))))

B-viewer = rec(lambda b. ...)

letactor {r = B-recep(s),
          s = B-lazy(v, (0, nil)),
          v = B-viewer( )}
    send(r, nil)
\end{verbatim}
\end{enumerate}

Assume the actor \texttt{v} can invoke a 0-ary function \texttt{event()} if it
receives a message.
\begin{enumerate}
\item Define the behavior of \texttt{B-viewer} so that if it is composed with
$\alpha_2$ it will never invoke \texttt{event()}, but if is composed with
$\alpha_1$, it may invoke \texttt{event()}. Give proofs.

Ans.

\begin{verbatim}
B-viewer = rec(lambda b. lambda c. lambda k. lambda m
                if((zero? k),
                   seq(send(c, true),
                       ready(B-viewer(c, 1))),
                   if(m,
                      seq(event(),ready(B-sink)),
                      ready(B-sink)))))
\end{verbatim}
We also need to change \texttt{letactor} so that \texttt{v = B-viewer(r, 0)}.
To trigger \texttt{event()}, the viewer must receive a message that is
\texttt{true}. Actor \texttt{r} will send actor \texttt{s} will become \texttt{sink} after receive 2
messages for both eager and lazy versions.

\item Assume that all messages from a given sender to a given recipient are
ordered. Are the two systems now equivalent? Prove.

\item As above, assume that all messages from a given sender to a given
recipient are ordered. Modify the two systems so that they are not equivalent:
the two modified systems should have the same behavior (modulo who they
communicate with) for the receptionist, viewer, and any other actors, but differ
only in that one has an ``eager'' actor (as above) and the other has a ``lazy
actor.'' Prove.
\end{enumerate}

\item In actor semantics, names (called addresses) of actors cannot be computed
and the names of an actor are unique and persistent. Suppose we wish to develop
an actor model for mobile robots with sensors. In particular, we want to address
actors by their ``current'' physical location. Assume a 2d world and that
configurations contain only contiguous points in a continuous 2d surface. Change
the behavior definition of an actor and an actor configuration to model
mobility. What is a reasonable notion of a receptionist, external actor,
composition of configurations? How should the messages and transitions be
defined?
\begin{enumerate}
\item Write out some interesting example in your system. State some properties
you may want to prove.

\item How would you define equivalence of actor expressions in your system?
\end{enumerate}

\end{enumerate}

\end{document}
