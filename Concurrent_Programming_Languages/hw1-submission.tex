\documentclass{article}

\usepackage[margin=3cm]{geometry}
\usepackage{amsmath}
\usepackage{amssymb}
\usepackage{xspace}

\title{\Large\bfseries
CS 524: Concurrent Programming Languages \\
Fall 2017 \\
Homework 1}
\author{Chiao Hsieh, chsieh16@illinois.edu}

\begin{document}
\maketitle

\begin{enumerate}
\item Consider a computation path from an actor configuration.
Define \textit{global time} using the order of message receives along the
path\textemdash global time is initially zero and is incremented each time a
message is received by some actor.
Define \textit{delivery} time as the difference between the global time of the
transition which causes a given message $(a, m)$ to be sent and the time of 
the event in which the message $m$ is received by $a$.
You may assume that the oldest message of the form $(a, m)$ is received before
any subsequent message $m$ sent to $a$.
The order of events on computation paths may be constrained by one of the 
following:

\textbf{Local Boundedness Axiom}: 
The set of delivery times of all messages which share the same target $a$ has 
an upper bound.

\textbf{Global Boundedness Axiom}:
The set of delivery times of all messages has an upper bound.

\textbf{Equiprobable Delivery Axiom}:
All pending events are equally likely to occur next.

Prove or disprove:
\begin{enumerate}
\item Local boundedness implies fairness.

Ans.

No. Consider following Actor system,
\begin{verbatim}
A = rec(lambda b. lambda c. lambda m.
           seq(send(c, m),
               non-terminating lambda expression,
               ready(A(c))))
letactor {a := A(a)} send(a, nil)
\end{verbatim}
The actor program \texttt{A} will send itself a message and does not
terminate. By definition of global time, the time is incremented only when an
actor receives a message. Since there is no other messages being sent, we
know the delivery time of each message sent by \texttt{A} is exactly 1. 
Therefore, it's locally bounded. However, the actor is never idle and hence 
cannot process the message sent by itself.

\item Local boundedness implies global boundedness.
\begin{verbatim}
A2 = rec(lambda b. lambda c. lambda k. lambda m.
         if(k < m,
            ready(A2(c, k+1)),
            seq(letactor {succ := A2(succ, 0)}
                    for(i=0; i<m+1, i++) send(succ, m+1),
                ready(sink))))
letactor {a2 := A2(a2, 0)} send(a2, 1)
\end{verbatim}

\item Global boundedness implies fairness.

Ans.

See (a).

\item If global boundedness is assumed, unbounded nondeterminism is not 
possible.

\item If local boundedness is assumed then unbounded nondeterminism is not 
possible.

\end{enumerate}
\end{enumerate}

\end{document}