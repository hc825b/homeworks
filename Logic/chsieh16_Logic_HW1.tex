\documentclass[11pt]{homework}

\usepackage{algorithm2e}

% TODO: replace these with your information
\newcommand{\hwname}{Chiao Hsieh}
\newcommand{\hwemail}{chsieh16@illinois.edu}
\newcommand{\hwtype}{Homework}
\newcommand{\hwnum}{1}
\newcommand{\hwclass}{CS 498 MP3: Logic}
\newcommand{\hwlecture}{}
\newcommand{\hwsection}{}

\renewcommand{\questiontype}{Problem}

\begin{document}
\maketitle

% Your content

\question
For the Frege proof system introduced in lecture 2, prove the Deduction Theorem
(without assuming the completeness theorem).
That is, if \(\Gamma \cup \{A\} \vdash B\) then \(\Gamma \vdash A \to B\).
\textit{Hint:} Let \(B_1, B_2,\dots, B_m\) be a proof from \(\Gamma \cup \{A\}\).
Show by induction on \(i\) that \(\Gamma \vdash A \to B_i\).

\begin{induction}
\basecase
For the base case, we need to show that if \(\Gamma \cup \{A\} \vdash B_1\) then \(\Gamma \vdash A \to B_1\).
Since there are only three possible ways in Frege system to derive a new formula in a single step, we only have to discuss the three cases
\begin{itemize}
  \item \(B_i\) is in the hypothesis, i.e., \(\Gamma \vdash B_i\) \\
        (The case \(B_i = A\) is skipped since \(\vdash A \to A\) is proven during the class)

  \item \(B_i\) is an axiom, i.e., \(\vdash B_i\)
  \item \(B_i\) is derive from \(B_j\) and \(B_k\) where \(j, k < i\) via
        Modus Ponens
\end{itemize}

Clearly the 3rd case is not applicable for \(i = 1\),
so we only provide proof for 1st and 2nd case.
Following Frege system, we can derive
\begin{align}
& \Gamma \vdash B_1                 && Hypothesis\ or\ Axiom\\
& \Gamma \vdash B_1 \to (A \to B_1) && Axiom\ 1  \\
& \Gamma \vdash A \to B_1           && Modus\ Ponens\ of\ (1) \ and\ (2)
\end{align}
\setcounter{equation}{0}
Hence, \(\Gamma \vdash A \to B_1\)

\indstep
For the inductive step, we starts from the assumption that 
if \(\Gamma \cup \{A\} \vdash B_i\) then \(\Gamma \vdash A \to B_i\)
for every \(i < m\).
We need to discuss the three cases. For 1st and 2nd case, we can derive in the
same way except \(B_1\) is replaced with \(B_m\).
Therefore, we only discuss the case that \(\Gamma \cup \{A\} \vdash B_m\) is derived from Modus Ponens of \(\Gamma \cup \{A\} \vdash B_j\) and \(\Gamma \cup \{A\} \vdash B_k\) for some \(j, k < m\).
W.l.o.g, we can say \(B_k\) is of the form \(B_j \to B_m\) so that we can apply Modus Ponens.
Following Frege system,
\begin{align}
& \Gamma \vdash A \to B_j           && Inductive\ Hypothesis\ on\ B_j\\
& \Gamma \vdash A \to (B_j \to B_m) && Inductive\ Hypothesis\ on\ B_k\\
& \Gamma \vdash (A \to (B_j \to B_m)) \to ((A \to B_j) \to (A \to B_m))
   && Axiom\ 2 \\
& \Gamma \vdash (A \to B_j) \to (A \to B_m)
   && Modus\ Ponens\ of\ (2) \ and\ (3) \\
& \Gamma \vdash A \to B_m           && Modus\ Ponens\ of\ (1) \ and\ (4)
\end{align}
By induction, if \(\Gamma \cup \{A\} \vdash B\), then \(\Gamma \vdash A \to B\).

\end{induction}

\question
The Davis-Putnam proof showing the completeness of resolution, outlines an 
algorithm to determine satisfiability of a set of clauses.
Use the Davis-Putnam algorithm to determine if the following sets of clauses are
satisfiable.
Outline all the steps of the algorithm in each case.

\begin{algorithm}
  \KwIn{A finite set of clauses \(\Gamma\) over \(n\) varibles \(v_1,\dots,v_n\)}
  \KwOut{SAT or UNSAT}
  \(\Gamma_0 := \Gamma\)
  
  k := 0

  \ForEach{\(v_i\)}{
    \If{both \(v_i\) and \(\neg v_i\) appear in \(\Gamma_k\)}{
      \(\Gamma_{k+1}\) := Resolve \(v_i\) in \(\Gamma_k\)
      
      \If{empty clause \( \in \Gamma_{k+1}\)}{
        \Return{UNSAT}
      }
      k := k + 1
    }
  }
  \Return{SAT}
\end{algorithm}

\begin{enumerate}
  \item
  \(\Gamma = \{ \{a,b,c\}, \{b,\neg c,\neg f\}, \{\neg b, e\} \}\)

  Variable \(a\) is skipped because it appears only in positive phase. \\
  Derive \(\Gamma_1\) by resolving \(b\) in \(\Gamma\)
\[
  \begin{array}{rcl}
    \Gamma_1^b &=& \emptyset \\
    \Gamma_2^b &=& \{\{a,b,c\}, \{b,\neg c,\neg f\}\} \\
    \Gamma_3^b &=& \{\{\neg b, e\}\} \\
      
    \Gamma_1 = \Gamma^b &=& 
      \emptyset \cup \{\{a,c,e\}, \{\neg c,\neg f, e\}\}
  \end{array}
\]

  Derive \(\Gamma_2\) by resolving \(c\) in \(\Gamma_1\)
  \[
  \begin{array}{rcl}
  \Gamma_1^c &=& \emptyset \\
  \Gamma_2^c &=& \{\{a,c,e\}\} \\
  \Gamma_3^c &=& \{\{\neg c, \neg f, e\}\} \\

  \Gamma_2 = \Gamma^c &=& 
    \emptyset \cup \{\{a,e,\neg f\}\}
  \end{array}
  \]
  No other variable appears in both phase. The loop is finished.
  The result is satisfiable.


  \item
  \(\Gamma = \{ \{a,b\}, \{a,\neg b\}, \{\neg a, c\}, \{\neg a,\neg c\} \}\)
    Derive \(\Gamma_1\) by resolving \(a\) in \(\Gamma\)
    \[
    \begin{array}{rcl}
    \Gamma_1^a &=& \emptyset \\
    \Gamma_2^a &=& \{\{a,b\}, \{a,\neg b\}\} \\
    \Gamma_3^a &=& \{\{\neg a, c\}, \{\neg a,\neg c\}\} \\
    
    \Gamma_1 = \Gamma^a &=& 
    \emptyset \cup \{\{b,c\}, \{b, \neg c\}, \{\neg b, c\}, \{\neg b,\neg c\}\}
    \end{array}
    \]
    
    Derive \(\Gamma_2\) by resolving \(b\) in \(\Gamma_1\)
    \[
    \begin{array}{rcl}
    \Gamma_1^c &=& \emptyset \\
    \Gamma_2^c &=& \{\{b,c\}, \{b, \neg c\}\} \\
    \Gamma_3^c &=& \{\{\neg b, c\}, \{\neg b,\neg c\}\} \\
    
    \Gamma_2 = \Gamma^c &=& 
      \emptyset \cup \{\{c\}, \{c, \neg c\}\} \\
               &=& \{\{c\}\} \\
    \end{array}
    \]
    No other variable appears in both phase. The loop is finished.
    The result is satisfiable.

\end{enumerate}

\question
A positive clause is a clause \(C\) all of whose literals are positive.
A resolution step is said to be positive if one of the hypothesis is a positive
clause.
Show that positive resolutions are sufficient to refute any
set of clauses.
In other words, if \(\Gamma\) is unsatisfiable then \(\Gamma\) has a resolution
refutation all of whose resolution steps are positive.

We prove this by induction on the number of variables in an unsatisfiable formula \(\Gamma\).

\begin{induction}
\basecase
\(\Gamma\) doesn't contain any variable and is unsatisfiable. Therefore, \(\Gamma\) must be \(\{\emptyset\}\) and there is nothing to prove.

\indstep
Our inductive hypothesis is that, for any unsatisfiable formula, there is a positive refutation if the number of variables is less then \(n\).
We want to show that there is also a positive refutation for a unsatisfiable
formula \(\Gamma\) which contains \(n\) variables.

Take any variable \(p\) appears in \(\Gamma\), we can compute \(\Gamma^p_\top\) and  \(\Gamma^p_\bot\) as shown below (borrowing the notations in the slides).
\[
\begin{array}{lcl}
\Gamma^p_\top &=& \Gamma^p_1 \cup \{B| B \cup \{p\} \in \Gamma^p_2 \} \\
\Gamma^p_\bot &=& \Gamma^p_1 \cup \{C| C \cup \{\neg p\} \in \Gamma^p_3\}
\end{array}
\]
Both \(\Gamma^p_\top\) and \(\Gamma^p_\bot\) must be unsatisfiable because \(\Gamma\) is unsatisfiable. The reason is straight forward because, if there is a satisfiable assignment \(v\) for \(\Gamma^p_\top\)(\(\Gamma^p_\bot\)), we can construct a satisfiable assignment for \(\Gamma^p\) by setting \(v(p) = \top(\bot)\).
The number of variables in both of them must less than \(n\) since \(p\) is removed.
Thus, there exists a positive refutation for both \(\Gamma^p_\top\) and \(\Gamma^p_\bot\).

Now, we prove that for any unsatisfiable \(\Gamma\), \(\Gamma^p_\bot\) can be derived from \(\Gamma\) with only positive resolution steps.

By definition, \(\Gamma^p_1 \subseteq \Gamma\); hence, no resolution step is needed to derive clauses in \(\Gamma^p_1\). 
Then, consider each clause \(C\) in \(\Gamma^p_\bot\), resolution step to derive \(C\) is to resolve each clause in \(\Gamma^p_3\) with \(\{p\}\), and this step is positive resolution since clause \(\{p\}\) is positive.

Finally, clause \(\{p\}\) is derivable from \(\Gamma\) with positive resolution.
Since, by inductive hypothesis, there are positive resolution steps from \(\Gamma^p_\top\) to empty clause, these steps can apply on \(\Gamma^p_1 \cup \Gamma^p_2\) with some mapping from \(B\) to original clause \(B \cup \{p\}\).
The result must be either empty clause or \(\{p\}\) because all other variables are resolved and no occurrence of \(\neg p\).

Thereby, a positive refutation for \(\Gamma\) must exist.
First, we can derive clause \(\{p\}\) from \(\Gamma^p_1 \cup \Gamma^p_2\),
then derive \(\Gamma^p_\bot\) with \(\{p\}\) and \(\Gamma^p_3\),
and finally derive empty clause from \(\Gamma^p_\bot\) under inductive hypothesis.

\end{induction}

\end{document}
