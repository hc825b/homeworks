\documentclass[11pt]{homework}

\usepackage[noend]{algorithm2e}
\usepackage{enumitem}
\usepackage{xcolor}

% TODO: replace these with your information
\newcommand{\hwname}{Chiao Hsieh}
\newcommand{\hwemail}{chsieh16@illinois.edu}
\newcommand{\hwtype}{Homework}
\newcommand{\hwnum}{2}
\newcommand{\hwclass}{CS 498 MP3: Logic}
\newcommand{\hwlecture}{}
\newcommand{\hwsection}{}

\renewcommand{\questiontype}{Problem}

\newcommand{\Nat}{\ensuremath{\mathbb{N}}}
\renewcommand{\P}{\ensuremath{\text{P}}}
\newcommand{\PSPACE}{\ensuremath{\text{PSPACE}}}
\newcommand{\DSPACE}{\ensuremath{\text{DSPACE}}}
\newcommand{\NP}{\ensuremath{\text{NP}}}
\newcommand{\coNP}{\ensuremath{\text{co-NP}}}

\begin{document}
\maketitle

% Your content

\question
 Recall that in class we saw that RE-complete problems are not recursive.
 Are there problems in RE that are not RE-complete but nonetheless not recursive?
 As we will show here, the answer is yes.
\begin{enumerate}[label=(\alph*)]
  \item
    Recall that \(L_d = \{i|i\not\in L(M_i)\}\).
    Prove that \(\overline{L_d}\) is RE-complete.
    
Ans.
  
We first prove that \(\overline{L_d} = \{i|i \in L(M_i)\}\) is RE by reduction to \(L_u\).
From lecture slides, we know \(L_u = \{\langle i,w\rangle|w \in L(M_i)\}\) is RE,
so there exists \(M_u\) which accepts \(L_u\).
For \(M\), we simply transform the input \(i\) to \(\langle i,i\rangle\)
and ask if \(M_u\) accepts the string.
The transformation is apparently computable (linear time).
Therefore, \(M\) can decide whether \(i \in \overline{L_d}\),
so \(\overline{L_d}\) is RE.

Now, we prove that \(\overline{L_d}\) is RE-hard by showing \(L_u \leq_m \overline{L_d}\) because \(L_u\) is proven to be RE-complete.

For any membership query \(\langle i, w\rangle\) for \(L_u\),
there exists a TM \(M_j\) which only checks if \(w \in L(M_i)\).
\[
  L(M_j) =
  \begin{cases}
    \Sigma^* & \text{if } M_i \text{ accepts } w\\
    \emptyset & \text{otherwise}
  \end{cases}
\]
Notice that \(j \in L(M_j) = \Sigma^*\) if \(\langle i, w\rangle \in L_u\), and \(j \not\in L(M_j) = \emptyset\) if \(\langle i, w\rangle \not\in L_u\).
We proved that \(\langle i, w\rangle \in L_u \Leftrightarrow f(\langle i, w\rangle) = j \in \overline{L_d}\).

The reduction function \(f(\langle i, w\rangle) = j\) is computable.
An implementation of \(f\) to produce \(M_j\) is adding additional states into \(M_i\)
that erase and write \(w\) to input tape,
so the computation is linear to the length of \(w\).

Therefore, we showed that \(L_u \leq_m \overline{L_d}\), and \(\overline{L_d}\) is RE-complete.

\end{enumerate}

Let us now define languages that are not RE in a very strong sense.
A language \(L\) is said to be \emph{productive}
if there is a computable function \(g\) such that whenever
\(L(M_i) \subseteq L, g(i) \in L \setminus L(M_i)\).
Informally, productive sets are non-RE languages for which you can always produce
a witness for why it is not equal to a particular RE language.
\begin{enumerate}[label=(\alph*), resume]
  \item Prove that \(L_d\) is productive.

Ans.

We consider two case of \(i\) such that \(L(M_i) \subseteq L_d\).

If \(i \in L_d\), then \(g(i) = i \in L \setminus L(M_i)\) because \(i \not\in L(M_i)\).

If \(i \not\in L_d\), then we know that \(i \in L(M_i)\);
it contradicts to our premise that \(L(M_i) \subseteq L_d\),
so this case is impossible.

Therefore, we can construct a computable function \(g(i) = i\), i.e., the identity function, to show that \(L_d\) is productive.
\end{enumerate}

Productive languages have the following two properties that you may assume without proof.

\textbf{Fact 1} If \(A \leq_m B\) and \(A\) is productive then \(B\) is productive.

\textbf{Fact 2} If \(A\) is productive then \(A\) contains an infinite RE subset.
\begin{enumerate}[label=(\alph*), resume]
    \item Using parts (a) and (b), and Fact 1 (above), 
          prove that if \(L\) is RE-complete then \(\overline{L}\) is productive.

Ans.

We show that \(L_d \leq_m \overline{L}\) for any RE-complete \(L\) to prove the claim.

If \(L\) is RE-complete, then there is a reduction \(f\) to show that \(\overline{L_d} \leq_m L\).
Consider \(f\), we know that
\[
\begin{array}{rl}
           & x \in \overline{L_d} \Leftrightarrow f(x) \in L \\
\therefore & x \in L_d \Leftrightarrow f(x) \in \overline{L}
\end{array}
\]
Thus, \(L_d \leq_m \overline{L}\).
By (b) and \textbf{Fact 1}, \(L_d\) is productive, so is \(\overline{L}\).

\end{enumerate}

Recall that the function \(f:\Nat \times \Nat \mapsto \Nat\) defined as
\[
f((m,n)) = \frac{(m+n)(m+n+1)}{2} + m
\]
is a bijection.
Using this function we can define a total order on \(\Nat \times \Nat\) as follows:
\((m_1,n_1)\leq (m_2,n_2)\) iff \(f((m_1,n_1))\leq f((m_2,n_2))\).
For any \(i \in \Nat\), define
\[
L_i =
\begin{cases}
\emptyset & \text{if } M_i \text{ does not accept any } j \geq 2i \\
\{j\}     & (j,k) \text{ is the smallest pair s.t. } j \geq 2i \text{, and } M_i \text{ accepts } j \text{ in } k \text{ steps}
\end{cases}
\]
Take \(L_* = \bigcup_{i\in\Nat} L_i\).

\begin{enumerate}[label=(\alph*), resume]
  \item Prove that \(L_*\) is RE.

Ans.

For any given \(w \in \Nat\), we can determine if \(w \in L_*\) through
asking if \(\langle i,w\rangle \in L_u\) to know if \(w \in L_i\) for every \(i\) that \( 0\leq i \leq w/2\).
However, we also have to devise the interleaving
between membership queries of \(L_u\) because not all queries will halt.
The detail is in Algorithm~\ref{alg:lstar}.

\begin{algorithm}
    \KwIn{\(w \in \Nat\)}
    \KwOut{``yes'' or ``no''}
    
    smallest := 0
    
    \(S_M\) := \(\{M_i \mid 2i<=w\}\)

  \While{\(S_M \neq \emptyset\)}{
    \((j, k)\) := \(f^{-1}\)(smallest)

    \ForEach{\(M_i \in S_M\)}{
        \If{\(M_i\) accepts \(j\) in \(k\) steps}{
            \If{j = w}{ \Return ``yes''}
            \Else{\(S_M\) := \(S_M \setminus \{M_i\}\)}
        }
    }
    smallest := smallest + 1
  }
  
  \Return ``no''
  
   \caption{Check if \(w \in L_*\)}
   \label{alg:lstar}
\end{algorithm}

The result is correct because the \((j,k)\) pairs is always the smallest,
and the algorithm will accept and halt if any of \(M_i\) accepts \(w\) with the smallest pair.
Otherwise, it continues until all \(M_i\) accept other \(j \neq w\) earlier and reject the input \(w\).

  \item Prove that \(\overline{L_*}\) is infinite but \(\overline{L_*}\)
        does not contain any infinite RE language.

Ans.

\(\overline{L_*}\) is infinite because, for any \(j \in \{0, 1, \dots, 2n-1\} \subset \Nat\), \(j \not\in L_i\) when \(i >= n > j/2\).
Thereby, only at most \(n\) numbers could possibly belong to \(L_*\),
and at at least \(n\) numbers belong to \(\overline{L_*}\).
\(\therefore |\overline{L_*}| \geq \frac{|\Nat|}{2} = \aleph_0\) is infinite.
 
To prove \(\overline{L_*}\) does not contain any infinite RE language,
we assume there is an infinite RE language accepted by a TM \(M_i\) s.t. \(L(M_i) \subseteq \overline{L_*}\).
Then, any integer \(j \in L(M_i)\) should not belong to \(L_*\).
The only possible scene is that \(M_i\) rejects any \(j \geq 2i\),
or otherwise there is always a smallest pair \((j, k)\) and makes \(j \in  L_i \subseteq L_*\).
Consequently, \(0 <= j < 2i\) for all \(j \in L(M_i)\); \(L(M_i)\) must be finite.
Contradiction.

  \item Using Fact 2, and part (c),
        conclude that \(L_*\) is neither recursive nor RE-complete.

Ans.

If \(L_*\) is recursive, then \(\overline{L_*}\) should be recursive,
but \(\overline{L*}\) is not recursive because, by (e), it doesn't contain any infinite RE language including itself.

If \(L_*\) is RE-complete, then \(\overline{L_*}\) should be productive by (c).
However, from (e), we know there is no infinite RE language subset in \(\overline{L_*}\).
By contrapositive of \textbf{Fact 2}, we know \(\overline{L_*}\) is not productive.

We conclude that \(L_*\) is neither recursive nor RE-complete.
\end{enumerate}

\question

\newcommand{\Lpad}{\ensuremath{L_{\text{pad}}}}

Prove that if \(\DSPACE(n) \subseteq \P\)
then \(\P = \PSPACE\).
\emph{Hint:} Let \(L \subseteq \{0,1\}^* \) be recognized by Turing machine \(M\) in space \(n^k\).
Define \(\Lpad = \{x\$^{|x|^k - |x|} | x \in L\}\).
What can you say about space needed to solve \(\Lpad\)?

Ans.

From lecture slides, we already know that \(\P \subseteq \PSPACE\),
so we only have to prove that,
if \(\DSPACE(n) \subseteq \P\), then \(\PSPACE \subseteq \P\).

Consider the reduction from a language \(L \in \DSPACE(n^k)\) to the \(\Lpad\) language,
we can devise an algorithm to determine if some \(y \in \Lpad\) through
\begin{enumerate}
  \item Check if \(y \) is in the form \(x\$^{|x|^k - |x|}\)
  
  In this case, at most \(|y| = |x\$^{|x|^k - |x|}| = |x| + |x|^k - |x| = |x|^k\) space is required.
  
  \item Check if \(x \in L\) using the deterministic TM \(M\)
  
  In this case, \(O(|x|^k)\) is required.
\end{enumerate}
As a result, the space needed is \(|x|^k + O(|x|^k) = O(|y|)\).
In other words, \(\Lpad \in \DSPACE(n)\), and we conclude that \(\Lpad \in \P\) from the hypothesis.

By definition, \(\PSPACE = \bigcup_k \DSPACE(n^k)\).
For every \(L \in \PSPACE\), there is a corresponding \(\Lpad\).
Obviously, the reduction from \(L\) to \(\Lpad\) is a polynomial time computable function.
Hence, \(L \leq_p \Lpad\), so \(L \in \P\) and \(\PSPACE \subseteq \P\).

\question
A strong non-deterministic TM is one that has three possible outcomes:
``yes'', ``no'', and ``maybe''.
We say that such a machine decides \(L\) iff the following holds:
If \(x \in L\) then all computations end up with either ``yes'' or ``maybe'',
and at least one ends up with ``yes''.
If \(x \not\in L\) then all computations end up with ``no'' or ``maybe'',
and at least one ends up with ``no''.
A polynomial time strong nondeterministic TM is a strong nondeterministic TM,
where every possible computation takes at most \(p(n)\) time,
where \(n\) is the input length.
Show that \(L\) is decided by a polynomial strong nondeterministic TM
if and only if \(L \in \text{NP} \cap \text{co-NP}\).

Ans.

Given \(M\) is a polynomial strong nondeterministic TM, we prove
\begin{itemize}
\item \(\exists M. L = L(M) \implies L \in \text{NP} \cap \text{co-NP}\)

Because there is at least one computation should end up with ``yes'',
by definition, \(M\) terminates and answers ``yes'' in \(p(n)\) time
for any string \(x \in L\) with length \(n\).
Therefore, \(L\) is in NP.
Similarly, \(M\) terminates and answers ``no'' in \(p(n)\) time
for any string \(x \in \overline{L}\) with length \(n\).
We can easily construct another TM \(\overline{M}\) that simply answer ``yes'' if \(M\) answers ``no''.
\(\overline{M}\) decides \(x \in \overline{L}\) in polynomial time; thus,
\(\overline{L} \in \text{NP}\) so \(L\) is also in co-NP.
Hence, \(L \in \text{NP} \cap \text{co-NP}\)

\item \(\exists M. L = L(M) \impliedby L \in \text{NP} \cap \text{co-NP}\)

Since \(L \in \text{NP}\), if \(x \in L\),
then there is a proof \(y\) of length at most \(p(n)\),
and we can construct a deterministic TM \(M_{yes}\) verifies \(x\#y\).
That is, \(L(M_\text{yes}) \in \text{P}\) and
\[
  L = \{x|\exists y. |x\#y| \leq p(n) \land x\#y \in L(M_\text{yes}) \}
\] where \(\#\) is an alphabet never used before.
Similarly, since \(L \in \text{co-NP}\),
we have \(L(M_\text{no}) \in \text{P}\) and
\[
  \overline{L} = \{x|\exists z. |x\#z| \leq p(n) \land x\#z \in L(M_\text{no}) \}
\]

Here, our polynomial strong nondeterministic TM \(M\) will guess \(y\) and \(z\).
If one \(x\#y\) is accepted by \(M_\text{yes}\) then answer ``yes'' otherwise answer ``maybe''.
If one \(x\#z\) is accepted by \(M_\text{no}\) then answer ``no'' otherwise answer ``maybe''.
Since both \(M_\text{yes}\) and \(M_\text{no}\) are deterministic polynomial time algorithm,
all computations should halts in \(p(n)\).

Moreover, there is no such case that \(M_\text{yes}\) and \(M_\text{no}\) both accept as the same time,
because it is impossible that \(x \in L \land x \in \overline{L}\).

Therefore, the constructed TM complies with the definition of polynomial strong nondeterministic TM.

\end{itemize}

\end{document}
