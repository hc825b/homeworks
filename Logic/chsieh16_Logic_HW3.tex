\documentclass[11pt]{homework}

\usepackage[noend]{algorithm2e}
\usepackage{enumitem}
\usepackage{xcolor}

% TODO: replace these with your information
\newcommand{\hwname}{Chiao Hsieh}
\newcommand{\hwemail}{chsieh16@illinois.edu}
\newcommand{\hwtype}{Homework}
\newcommand{\hwnum}{3}
\newcommand{\hwclass}{CS 498 MP3: Logic}
\newcommand{\hwlecture}{}
\newcommand{\hwsection}{}

\renewcommand{\questiontype}{Problem}

\begin{document}
\maketitle

% Your content

\question
 Recall that we showed that Th(\((\mathbb{N},0,1,+)\)) is decidable. 
 Show that the theory continues to be decidable when we add the following predicates.

\begin{enumerate}
  \item \(<\), a binary predicate, such that \(< ij\) is true iff \(i < j\)
  \item even, a unary predicate, such that even(\(i\)) iff \(i\) is even
  \item power2, a unary predicate,
        such that power2(\(i\)) iff \(i\) is a power of 2,
        or there is a \(j\) that \(i = 2^j\)
\end{enumerate}

Ans.

We follow the encoding in the lecture slides to map any given formula 
\(\varphi (x_1, x_2,\dotsc,x_n)\) in Presburger Arithmetic to a MSO formula
\(\varphi'(X_1, X_2,\dotsc,X_n)\).

The construction of \(\varphi'\) by structural induction for logic connectives
is still valid even if new predicates are introduced into the theory.
Therefore, we only have to consider how to build a corresponding MSO formula for
each new predicate introduced.
To simplify the notations used in MSO logic,
we further use the predicate \((x \prec y)\) to denote position \(x\) is
before position \(y\).
The predicate can be defined in terms of \(S\) similar to MSO logic on words,
so we omit the definition here.

\begin{enumerate}
    \item \(<\), a binary predicate, such that \(< ij\) is true iff \(i < j\)

      To compare two binary encoded natural numbers \(X_i\) and \(X_j\),
      the systematic way is to find the most significant bit that differs in two numbers.
      If \(i < j\), then there must be a \(k\)-th bit that is 1 in \(j\) and is 0 in \(i\).
      That is, \(k \in X_j \land k \not\in X_i\).
      Besides, all bits more significant than \(k\)-th bit should match.
\[
\begin{array}{rcl}
	< i0 & \mapsto & \neg (0 = 0) \\
	< 0i & \mapsto & \exists k(X_i k)                                                                                            \\
	< ij & \mapsto & \exists k(\neg X_i k \land X_j k \land 
    \forall k'(k \prec k' \to (X_i k' \leftrightarrow X_j k')) )
\end{array}
\]

    \item even, a unary predicate, such that even(\(i\)) iff \(i\) is even
      
      To check if a binary encoding \(X_i\) is even,
      we simply check if the 0th bit is 1. That is, \(0 \not\in X_i\).
\[
\begin{array}{rcl}
	\text{even}(0) & \mapsto & 0 = 0       \\
	\text{even}(i) & \mapsto & \neg X_i(0)
\end{array}
\]

    \item power2, a unary predicate,
    such that power2(\(i\)) iff \(i\) is a power of 2,
    or there is a \(j\) that \(i = 2^j\)
    
      To check if \(i\) is the number of \(2^j\),
      we actually only need to check if only one bit is 1.
      That is, \(X_i\) contains only one element.
\[
\begin{array}{rcl}
	\text{power2}(0) & \mapsto & \neg(0 = 0) \\
	\text{power2}(i) & \mapsto & \exists j( X_i(j) \land \forall k(X_i k \to k=j))
\end{array}
\]

\end{enumerate}

\question
Recall that in the translation from automata to MSO presented in class,
the sentence constructed existential quantifies \(n\) set variables,
where \(n\) is the number of states in the automaton.
The set \(X_i\) encodes all positions where the automaton is in state \(i\).  
The number of existentially quantified set variables in the translation can,
however, be reduced to \(\lceil \log n \rceil\) using the following idea.
In the new translation, a position \(p\) will be in set \(X_i\)
if the automaton is in state \(s\) at position \(p\),
and in the binary representation of \(s\),
the \(i\)-th bit is 1.
So, for example, suppose the automaton has 4 states 1, 2, 3, 4 encoded by strings 00, 01, 10, 11, respectively.
Then \(p \in X_1\) will mean that, at position \(p\),
the automaton is either in state 3 or 4 (the two states with 1 in position 1).
This problem requires you to carry out this idea on the following example.

Consider the machine with states \{1, 2, 3, 4\}
with 1 as the initial state and 2 as the only final state.
The transitions of the machine are as follows:
\(\delta(1,b)=2\), \(\delta(2,a)=3\), \(\delta(3,a)=3\), \(\delta(3,b)=4\),
and \(\delta(4,b)=1\).
Construct an equivalent MSO sentence that has only 2 quantifiers for set variables.

Ans.

We follow the step in slides to construct the formula.
\[
\begin{array}{rl}
	\varphi = \exists X_1 \exists X_0
       & \text{exactly one state in each position}                \\
 \land & \forall x (first(x) \to (\neg X_1(x) \land \neg X_0(x))) \\
 \land & \forall x \forall y ( S x y \to \text{State at \(y\) follows from State at \(x\) by transition}) \\
 \land & \forall x (last(x) \to \text{Final state is reached from state at \(x\)})
\end{array}
\]

\textbf{Exactly one state in each position}

Since either \(p \in X_i\) or \(p \notin X_i\) for every \(X_i\),
the binary string and the state is uniquely defined for any given position \(p\).
Hence, this is trivially true.

\textbf{State at \(y\) follows from State at \(x\) by transition}
\[
\begin{array}{rlcl}
	     & (\neg X_1 x \land \neg X_0 x \land Q_b \land \neg X_1 x \land X_0 x) & \cdots & \delta(1,b)=2 \\
	\lor & (\neg X_1 x \land X_0 x \land Q_a \land X_1 x \land \neg X_0 x) & \cdots & \delta(2,a)=3 \\
	\lor & (X_1 x \land \neg X_0 x \land Q_a \land X_1 x \land \neg X_0 x) & \cdots & \delta(3,a)=3 \\
	\lor & (X_1 x \land \neg X_0 x \land Q_b \land X_1 x \land X_0 x) & \cdots & \delta(3,b)=4 \\
    \lor & (X_1 x \land X_0 x \land Q_b \land \neg X_1 x \land \neg X_0 x) & \cdots & \delta(4,b)=1 \\
\end{array}
\]

\textbf{Final state is reached from state at \(x\)}
\[
(\neg X_1 x \land \neg X_0 x \land Q_b)\ \cdots\ \delta(1, b) = 2
\]

\question
There are many ways to define real numbers.
One way is to use \emph{Dedekind cuts}.
Intuitively, every real number \(a \in \mathbb{R}\) can be expressed as a partition of rational numbers \((S,T)\),
where \(S = \{s \in \mathbb{Q} \mid s \leq a\}\) and \(T = \{t \in \mathbb{Q} \mid t > a\}\);
since \(T = \mathbb{Q}\setminus S\),
we can think of the cut as just a single set \(S\).
Conversely, for any set \(S\) that is downward closed
(i.e., \(x < y\) and \(y \in S\) implies \(x \in S\))
corresponds to the (unique) real number \(r\) that is the supremum of \(S\).
For example, \(\sqrt{2}\) is represented by the set 
\(S = \{s \in \mathbb{Q} \mid s^2 \leq 2 \lor s < 0\}\).
Observe that \(\sqrt{2} \not\in S\) but \(\sqrt{2} = \sup(S)\)

Assume that Dedekind cuts are a faithful representation of all real numbers (which they are).
Using Dedekind cuts, show how to interpret first-order logic formulas over reals,
i.e., over \((\mathbb{R}, +, <, =, 0, 1)\),
into the monadic second-order theory over the rational numbers
\((\mathbb{Q}, +, <, =, 0, 1)\).
In other words, come up with a uniform way to map sentences FOL sentences \(\varphi\)
over reals with addition to an MSOL sentence \(\varphi'\) over rationals with addition,
such that \(\varphi\) holds over reals iff \(\varphi'\) holds over rationals.

Ans.

In order to map a given formula \(\varphi (x_1, x_2,\dotsc,x_n)\) in FOL over real numbers to a \(\varphi'(X_1, X_2,\dotsc,X_n)\) in MSOL over rational numbers,
we consider following bijective function \(\sup^{-1} : \mathbb{R} \mapsto \mathcal D\) 
where \(\mathcal D \subset \mathcal P(\mathbb{Q})\) is the set of all downward closed sets.
Based on the bijective function,
we constructs \(\varphi'\) from \(\varphi\) such that
\[
(\mathbb{R}, +, <, =, 0, 1) \models \varphi[\alpha] \iff
(\mathbb{Q}, +, <, =, 0, 1) \models \varphi'[[X_i \mapsto {\sup}^{-1} (\alpha(x_i))]_{i=1}^{n}]
\]
Since every set \(X_i\) in our model is assumed to be downward closed,
we need following axioms for every variable \(X_i\) in \(\varphi'\) we construct.
\[
DownwardClosed(X_i) := \forall x y((X_i y \land x < y) \to X_i x)
\]
\[
\varphi' := \bigwedge_{i=1}^n DownwardClosed(X_i) \land \varphi''
\]
where \(\varphi''\) can be constructed \(\varphi'\) by structural induction.
Below are constructions for predicate \(=\), \(<\), and \(+\) in FOL.
Induction on logical connectives is the same as those on the slides.
\[
\begin{array}{rclcl}
	        \varphi &         & \varphi''                                &  \\
	          0 = 0 & \mapsto & 0 = 0                                    &  \\
	          0 = 1 & \mapsto & 0 = 1                                    &  \\
	          1 = 1 & \mapsto & 1 = 1                                    &  \\
	        0 = x_i & \mapsto & \forall y (X_i y \to (y < 0 \lor y = 0)) & \cdots & X_i = \{y \in \mathbb{Q} \mid y \leq 0\} \\
	        1 = x_i & \mapsto & \forall y (X_i y \to (y < 1 \lor y = 1)) &  \\
	      x_i = x_j & \mapsto & \forall y (X_i y \leftrightarrow X_j y)  & \cdots & X_i = X_j \\
	        0 < x_i & \mapsto & \exists y (X_i y \land 0 < y)            &  \\
	        1 < x_i & \mapsto & \exists y (X_i y \land 1 < y)            &  \\
	        x_i < 0 & \mapsto & \forall y (X_i y \to y < 0)              &  \\
	        x_i < 1 & \mapsto & \forall y (X_i y \to y < 1)              &  \\
	      x_i < x_j & \mapsto & \exists y (\neg X_i y \land X_j y)       & \cdots & y \notin X_i \land y \in X_j \\
	x_i + x_j = x_k & \mapsto & \multicolumn{3}{l}{\forall y_i y_j ((X_i y_i \land X_j y_j) \to \exists y_k(X_k y_k \land y_i + y_j = y_k))}       \\
	                &         & \multicolumn{3}{l}{\land\ \forall y_k (X_k y_k \to \exists y_i y_j (X_i y_i \land X_j y_j \land y_i + y_j = y_k))}
\end{array}
\]
\end{document}
