\documentclass[10pt]{homework}

\usepackage[noend]{algorithm2e}
\usepackage{enumitem}
\usepackage{xcolor}
\usepackage{qtree}
\usepackage{subcaption}

% TODO: replace these with your information
\newcommand{\hwname}{Chiao Hsieh}
\newcommand{\hwemail}{chsieh16@illinois.edu}
\newcommand{\hwtype}{Homework}
\newcommand{\hwnum}{4}
\newcommand{\hwclass}{CS 498 MP3: Logic}
\newcommand{\hwlecture}{}
\newcommand{\hwsection}{}

\renewcommand{\questiontype}{Problem}

\begin{document}
\maketitle

% Your content

\question
Let an \emph{almost deterministic top down tree automaton} be defined like a top
down deterministic tree automaton,
except that it can have a set of initial states \(Q_0\) (instead of a single
initial state).
Like for top down NTA, an accepting run can begin with any of the states in
\(Q_0\).
Show that there is a regular tree language that cannot be recognized by an
almost deterministic top down tree automaton.

Ans.

Given an almost deterministic top down tree automaton \(A = (Q, \Sigma, \delta, Q_0)\)
with \(Q \supseteq Q_0 = \{q_1,\dots,q_n\}\),
we can find a set of top down DTA \(\{A_1, \dots, A_n\}\) 
where \(A_i = (Q, \Sigma, \delta, \{q_i\})\) such that
\[
L(A) = \bigcup_{i=1}^n L(A_i)
\]
The proof is straight-forward since any tree accepted by \(A\) will produce
an accepting run starting from one of the initial states, \(q_i\).
The accepting run is also valid for automaton \(A_i\),
so the tree should be accepted by \(A_i\) as well.
The other direction can be proved similarly.
According to the above observation, we can conclude that the language of
a top-down almost DTA must be the language union of a finite set of top-down DTA.

Therefore, we want to come up with a regular tree language that cannot be represented as
the language union of a finite set of top-down DTA.
We consider following regular tree language \(L(N)\).
Any tree accepted is in following form that the right side of the tree 
are recurring `\(r\)'s and stop with a \(\bot\),
and the left child of every `\(r\)' would be `\(f\)'.
The subtree of `\(f\)' can be either \(f(a, b)\) or \(f(b, a)\).
\begin{figure*}[h]
\centering
\begin{subfigure}{.5\textwidth}
\Tree [.\(r\) \qroof{\ \hspace{0.5cm}\ }.\(f\)
         [.\(r\) \qroof{\ \hspace{0.5cm}\ }.\(f\)
            [.{\(\underset{r}{\ddots}\)} \qroof{\ \hspace{0.5cm}\ }.\(f\) {\(\bot\)} ]
         ]
      ] 
\end{subfigure}
\begin{subfigure}{.2\textwidth}
\Tree [.\(f\) \(a\) \(b\) ] \hfill \textbf{or} \hfill \Tree [.\(f\) \(b\) \(a\) ]
\end{subfigure}
\end{figure*}

Formally, the language is defined by a top down NTA
\(N = (Q, \Sigma, \delta, Q_0)\) where
\[
\begin{array}{rcl}
	       Q & = & \{q_r, q_f, q_a, q_b\}                                                                \\
	  \Sigma & = & \Sigma_0 \cup \Sigma_2 \text{ where } \Sigma_0 = \{a, b, \bot\}, \Sigma_2 =  \{r, f\} \\
	  \delta & = & \delta_0 \cup \delta_2                                                                \\
	\delta_0 & = & \{(q_a, a), (q_b, b), (q_r, \bot)\}                                                   \\
	\delta_2 & = & \{(q_r, r, q_f, q_r), (q_f, f, q_a, q_b), (q_f, f, q_b, q_a)\}                        \\
	     Q_0 & = & \{q_r\}
\end{array}
\]

Here we prove by contradiction that this language cannot be accepted by any top down almost DTA.
Assuming that we can find a set of top down DTA, \(\{A_1,\dots,A_n\}\),
such that \(L(N) = \bigcup_{k=1}^n L(A_k)\),
we consider following \(n+1\) trees in \(L(N)\).
Each tree consists of \(n+1\) `\(r\)'s with exactly one `\(r\)' whose left subtree is
\(f(a,b)\) and the rest \(n\) left subtrees of other `\(r\)'s are \(f(b,a)\).
Let \(t_i\) denote the tree that the `\(r\)' of depth \(i\) is with \(f(a,b)\) as left subtree.
Therefore, there are \(n+1\) trees, \(\{t_0,\dots,t_n\}\).

Note that if we take any 2 trees \(t_i\) and \(t_j\) with \(i<j\),
they differ at the left subtree of \(i\)-th `\(r\)' but are the same for all
nodes with depth \(\leq i+1\) (including `\(f\)').
If a top down DTA \(A_k= (Q, \{a,b,\bot,f,r\}, \delta, \{q_0\})\) accepts both \(t_i\) and \(t_j\),
there must be 3 states \(q_f, q_1, q_2 \in Q\) such that \(q_0\) will reach \(q_f\)
with exactly same sequence of transitions for both \(t_i\) and \(t_j\), and
\[
\begin{array}{ll}
\delta(q_f, f) = (q_1, q_2) \\
\delta(q_1, a) = 1;\quad \delta(q_2, b) = 1 & \cdots \text{ for accepting } t_i\\
\delta(q_1, b) = 1;\quad \delta(q_2, a) = 1 & \cdots \text{ for accepting } t_j\\
\end{array}
\]
In this case, \(A_k\) will also accept another tree that the left subtree of
\(i\)-th `\(r\)' is \(f(a,a)\).
Apparently, this tree is not in \(L(N)\) because no subtree for any tree is \(f(a,a)\),
so \(t_i\) and \(t_j\) must not be accepted by one top down DTA at the same time.

Hence, the \(n+1\) trees must be accepted by \(n+1\) top down DTAs separately.
However, there are only \(n\) top down DTAs available.
As a result, we proved by contradiction that there is no top down almost DTA
with finite initial states whose language is \(L(N)\).


\question
Prove that any two-terminal series-parallel graph \(G = (V, E, s, t, I)\) has a
tree decomposition of width at most 2.
Hint:
Use induction (on the construction of \(G\)) to show the following stronger
observation: \(G\) has a tree decomposition of width 2, where the label of the
root contains \(\{s, t\}\).

Ans.

By definition, a two-terminal series-parallel graph (TTSPG) must be a series or
parallel composition of 2 TTSPGs;
hence we prove the claim by induction on the composition.

\begin{induction}
\basecase
The base case \(G_0\) is a graph with single edge connecting source and sink.
That is, \(G_0 = (V_0, E_0, s, t, I_0)\) where \(V_0 = \{s, t\}\),
\(E_0 = \{A\}\), and \(I_0 = \{(s, A, t), (t, A, s)\}\).

For \(G_0\), we can easily derive a tree decomposition
\[
\begin{array}{l}
\tau(G_0) = (V_{\tau(G_0)}, E_{\tau(G_0)}, L_{\tau(G_0)}) \text{ where} \\
V_{\tau(G_0)} = \{v\} \\
E_{\tau(G_0)} = \emptyset \\
L_{\tau(G_0)}(v) = \{s, t\}\\
\end{array}
\]
The tree decomposition is of width less than 2,
and the label of root node contains both \(s\) and \(t\).

\indstep
Given a TTSPG \(G\) that is a series-parallel composition of two TTSPGs 
\(G_1 = (V_1, E_1, s_1, t_1, I_1)\) and \(G_2 = (V_2, E_2, s_2, t_2, I_2)\).
In general, \(V_1\) and \(V_2\) are disjoint.
By induction hypothesis, we assume that there are 2 tree decompositions of width 2,
\(\tau(G_1) = (V_{\tau(G_1)}, E_{\tau(G_1)}, L_{\tau(G_1)})\)
and \(\tau(G_2) = (V_{\tau(G_2)}, E_{\tau(G_2)}, L_{\tau(G_2)})\).
Also the root label of them contains source and sink nodes of their graphs.
Formally, let \(r_1 \in V_{\tau(G_1)}\) be the root node of \(\tau(G_1)\).
We hypothesize that \(s_1 \in L_{\tau(G_1)}(r_1)\) and \(t_1 \in L_{\tau(G_1)}(r_1)\).
Same for \(r_2\) and \(\tau(G_2)\).

\textbf{Series Composition}

If \(G\) is the series composition of \(G_1\) and \(G_2\),
then we construct the tree decomposition \(\tau(G)\) by 

\[
\begin{array}{l}
	\tau(G) = (V_{\tau(G)}, E_{\tau(G)}, L_{\tau(G)}) \text{ where}                                                \\
	V_{\tau(G)} = V_{\tau(G_1)} \cup V_{\tau(G_2)} \cup \{r\}, r \notin V_{\tau(G_1)} \land r \notin V_{\tau(G_2)} \\
	E_{\tau(G)} = E_{\tau(G_1)} \cup V_{\tau(G_2)} \cup \{(r, r_1), (r_1, r), (r, r_2), (r_2, r)\}                 \\
	L_{\tau(G)}(v) = \left\{
\begin{array}{ll}
	\{s_1, t_1, t_2\}                                 & \text{if } v = r                                                \\
	L_{\tau(G_1)}(v)                                  & \text{if } v \in V_{\tau(G_1)}                                   \\
	L_{\tau(G_2)}(v)                                  & \text{if } v \in V_{\tau(G_2)}\land s_2 \notin L_{\tau(G_2)}(v) \\
	(L_{\tau(G_2)}(v) \setminus \{s_2\}) \cup \{t_1\} & \text{otherwise}
\end{array}\right.  \\
\end{array}
\]

The label of new root node \(r\) clearly contains the source \(s_1\) and sink
node \(t_2\) of the composed graph \(G\).
The width of the constructed \(\tau(G)\) is at most 2 by observing \(L_{\tau(G)}\).
For the root node case, there are only 3 elements.
For other cases, the number of elements in any label from \(L_{\tau(G_1)}\) or
\(L_{\tau(G_2)}\) must be less than or equal to 3 by induction hypothesis.
Thus, we know the tree width of \(\tau(G)\) is 2 because we only replace one node
 with another node and don't change the number of nodes in the label.

For node coverage and edge coverage, since by hypothesis all nodes and edges in
\(G_1\) must be covered in \(L_{\tau(G_1)}\), and so are nodes and edges in \(G_2\).
We only consider the new edges connecting with \(t_1\) due to \(s_2\) merged into \(t_1\).
These edges are covered by the 4th case of \(L_{\tau(G)}\).
Because any edge from/to \(s_2\) in \(I_2\) is covered by some labels in \(L_{\tau(G_2)}\),
replacing \(s_2\) with \(t_1\) on all labels in \(L_{\tau(G_2)}\) can guarantee to cover
all the new edges.

Finally, we argue that the coherence is preserved.
The coherence between nodes in \(V_{\tau(G_1)}\) trivially holds because of inductive hypothesis on \(\tau(G_1)\),
so does the coherence in \(V_{\tau(G_2)}\).
Thus, we only have to consider any 2 nodes \(v_1 \in V_{\tau(G_1)}\) and
\(v_2 \in V_{\tau(G_2)}\) where \(t_1 \in L_{\tau(G)}(v_1)\cap L_{\tau(G)}(v_2)\).
Note that no node other than \(t_1\) can satisfy the above condition since, by definition,
\(L_{\tau(G)}(v_1) \subseteq V_1\), \(L_{\tau(G)}(v_2) \subseteq (V_2 \setminus \{s_2\})\cup\{t_1\}\),
and \(V_1 \cap V_2 = \emptyset\).
\(\therefore L_{\tau(G)}(v_1)\cap L_{\tau(G)}(v_2) \subseteq V_1 \cap ((V_2 \setminus \{s_2\})\cup\{t_1\}) = \{t_1\}\)

Now considering the path connecting \(v_1\) and \(v_2\),
we know the path passes through \(r_1\), \(r\), and \(r_2\) by the tree construction.
Because \(t_1 \in L_{\tau(G)}(v_1) \cap L_{\tau(G)}(r_1)\),
by coherence in \(\tau(G_1)\), every node \(u\) on the path from \(v_1\) to \(r_1\) satisfies \(t_1 \in L_{\tau(G)}(u)\).
The same goes for every node on the path from \(v_2\) to \(r_2\).
\(r_1\) and \(r_2\) are connected by \(r\) where \(t_1 \in L_{\tau(G)}(r)\) as well.
Therefore, we show that every node on the path connecting any \(v_1\) and \(v_2\)
satisfies the coherence.

\textbf{Parallel Composition}

For parallel composition, we construct following tree decomposition.
\[
\begin{array}{l}
\tau(G) = (V_{\tau(G)}, E_{\tau(G)}, L_{\tau(G)}) \text{ where}                                                \\
V_{\tau(G)} = V_{\tau(G_1)} \cup V_{\tau(G_2)} \cup \{r\}, r \notin V_{\tau(G_1)} \land r \notin V_{\tau(G_2)} \\
E_{\tau(G)} = E_{\tau(G_1)} \cup V_{\tau(G_2)} \cup \{(r, r_1), (r_1, r), (r, r_2), (r_2, r)\}                 \\
L_{\tau(G)}(v) = \left\{
\begin{array}{ll}
	\{s_1, t_1\}                                      & \text{if } v = r                                                \\
	L_{\tau(G_1)}(v)                                  & \text{if } v \in V_{\tau(G_1)}                                   \\
	L_{\tau(G_2)}(v)                                  & \text{if } v \in V_{\tau(G_2)} \land s_2 \notin L_{\tau(G_2)}(v) \land t_2 \notin L_{\tau(G_2)}(v)\\
	(L_{\tau(G_2)}(v) \setminus \{s_2\}) \cup \{s_1\} & \text{if } v \in V_{\tau(G_2)} \land s_2 \in    L_{\tau(G_2)}(v) \land t_2 \notin L_{\tau(G_2)}(v)\\
	(L_{\tau(G_2)}(v) \setminus \{t_2\}) \cup \{t_1\} & \text{if } v \in V_{\tau(G_2)} \land s_2 \notin L_{\tau(G_2)}(v) \land t_2 \in    L_{\tau(G_2)}(v)\\
    (L_{\tau(G_2)}(v) \setminus \{s_2, t_2\}) \cup \{s_1, t_1\} & \text{otherwise}
\end{array}\right.  \\
\end{array}
\]

The source \(s_1\) and sink node \(t_1\) of the composed graph \(G\) are in the
label of new root node \(r\).
The width of the constructed \(\tau(G)\) is also at most 2 if we analyze
\(L_{\tau(G)}\) case by case.
For each case we remove and add the same number of nodes on each label;
therefore the width of \(\tau(G)\) will be the same as the width of either
\(\tau(G_1)\) or \(\tau(G_2)\).
By induction hypothesis, we know the width is at most 2.

The node and edge coverage also holds based on the same reasoning provided for
series composition.
The new edges connected form nodes in \(V_2\) to \(s_1\) and \(t_1\) are covered
by labels where \(s_2\) and \(t_2\) were replaced.

Finally, the coherence holds as well.
It is similar to the reasoning for series composition.
We have to consider any 2 nodes \(v_1 \in V_{\tau(G_1)}\) and \(v_2 \in V_{\tau(G_2)}\)
where there is a node \(n \in L_{\tau(G)}(v_1)\cap L_{\tau(G)}(v_2)\).
Note that, by definition, \(L_{\tau(G)}(v_1) \subseteq V_1\), 
\(L_{\tau(G)}(v_2) \subseteq (V_2 \setminus \{s_2, t_2\})\cup\{s_1, t_1\}\),
and \(V_1 \cap V_2 = \emptyset\).
Therefore, \(L_{\tau(G)}(v_1)\cap L_{\tau(G)}(v_2) \subseteq \{s_1, t_1\}\).
Either \(n = s_1\) or \(n = t_1\).

For both cases, we know \(n \in L_{\tau(G)}(r_1)\) because of induction hypothesis,
and \(n \in L_{\tau(G)}(r_2)\) because \(s_2\) and \(t_2\) in \(L_{\tau(G_2)}(r_2)\)
are replaced by \(s_1\) and \(t_1\).
Since the path connecting \(r_1\) and \(v_1\) must satisfy coherence,
so does the path connecting \(r_2\) and \(v_2\)
we know that the path passing through \(v_1, r_1, r, r_2, v_2\) must satisfy coherence
because \(s_1\) and \(t_1\) are also in \(L_{\tau(G)}(r)\).

\end{induction}
We conclude that the provided tree construction derived from a TTSPG is indeed
a tree decomposition,
i.e., it satisfies node coverage, edge coverage, and coherence.
Also we show that the width of the constructed tree decomposition is at most 2.
Thereby, we prove that the provided construction can produce tree decomposition
of width 2 for any TTSPG inductively.

\question

A \emph{pushdown automaton} \(P\) (without inputs and final states) is a tuple
\((Q, q_0, \Gamma, \bot, \delta)\), where \(Q\) is a finite set of states,
\(q_0 \in Q\) is the initial state, \(\Gamma\) is the finite stack alphabet,
\(\bot \notin \Gamma\)  is the initial stack symbol,
and \(\delta = \delta_\mathtt{push} \cup \delta_\mathtt{pop} \cup \delta_\mathtt{nochg}\) is the transition relation.
\(\delta_\mathtt{push} \subseteq Q \times \Gamma \times Q\) are the push 
transitions: \((q_1, \gamma, q_2) \in \delta_\mathtt{push}\) means that the automaton can go from \(q_1\) to \(q_2\) by pushing \(\gamma\) onto the stack 
(independent of what is on top of the stack).
\(\delta_\mathtt{pop} \subseteq Q \times \Gamma \times Q\) are the pop transitions:
\((q_1, \gamma, q_2) \in \delta_\mathtt{pop}\) means that the automaton can go from \(q_1\) to \(q_2\) if \(\gamma\) is on top of the stack, and the result of taking the transition is to pop \(\gamma\) from the stack.
\(\delta_\mathtt{nochg} \subseteq Q \times Q\) are the no stack change transitions:
\((q_1, q_2) \in \delta_\mathtt{nochg}\) means that the automaton can go from \(q_1\) to \(q_2\) without changing the stack (independent of what is on top of the stack).

The graph of a pushdown automaton, \(P\), is the (infinite) graph \(G(P) = (V, E)\) where \(V = \bot \Gamma^* Q\) and \(E\) is defined as follows.
\((\bot \gamma^1_1 \gamma^1_2 \cdots \gamma^1_n q_1, \bot \gamma^2_1 \gamma^2_2 \cdots \gamma^2_m q_2) \in E \) if one of the following 3 conditions hold.
Either \((q_1, \gamma, q_2) \in \delta_\mathtt{push}\) and \(m = n + 1\)
with \(\gamma^2_m = \gamma\) and \(\gamma^2_i = \gamma^1_i\) for \(i \leq n\),
or \((q_1, \gamma, q_2) \in \delta_\mathtt{pop}\) and \(m = n - 1\)
with \(\gamma^1_n = \gamma\) and \(\gamma^2_i = \gamma^1_i\) for \(i \leq n-1\).
Or \((q_1, q_2) \in \delta_\mathtt{nochg}\) and \(m = n\)
with \(\gamma^2_i = \gamma^1_i\) for \(i \leq n\)

The \emph{infinite \(n\)-ary tree} \(\mathcal{T}\) is the infinite tree where every node has exactly \(n\) children.
It can be viewed as a first order structure over the signature \(\tau = \{S_1, S_2, \dots, S_n\}\) where \(S_i\) is the \(i\)th child relation.
In other words, as a first order structure \(\mathcal{T} = (T, \{S_i^\mathcal{T}\}_{i=1}^n)\), where \(T\) is the set of nodes,
and \((u,v) \in S_i^\mathcal{T}\) iff \(v\) is the \(i\)th child of \(u\).

Consider a pushdown automaton \(P\) with \(\Gamma = \{1,2,\dots,k\}\) and
\(Q = \{k+1,k+2,\dots, n\}\).
Prove that for every MSO sentence \(\varphi\) (over signature \(\tau_G = \{E\}\))
there is an MSO sentence \(\varphi^*\) (over signature \(\tau_n\)) such that
\[
G(P) \models \varphi \qquad \text{iff} \qquad \mathcal{T} \models \varphi^*
\]

\emph{Hint:} Use the idea of interpretations.
For a vertex \(v = \bot\gamma_1\gamma_2\dots\gamma_m q\) of \(G(P)\),
let \(t(v)\) be the node of \(\mathcal{T}\) reached by following the path 
\(\gamma_1\gamma_2\dots\gamma_m q\) from the root.
And for a set of vertices \(U\) of \(G(P)\), let \(t(U) = \{t(v) \mid v\in U\}\)
The translation to \(\varphi^*\) can be done inductively,
maintaining the following invariant
\begin{multline*}
\qquad G(P) \models \varphi(x_1,\dots,x_s,X_1,\dots,X_t)[x_i \to v_i, X_i \to U_i] \\
\text{ iff }
\mathcal{T} \models \varphi^*(x_1,\dots,x_s,X_1,\dots,X_t)[x_i \to t(v_i), X_i \to t(U_i)] \qquad
\end{multline*}

In doing this construction, it might be useful to first define a formula Vert(\(x\))
such that, for a node \(n_1\) of \(\mathcal{T}\), Vert(\(n_1\)) holds iff
\(n_1 = t(v)\) for some \(v\) of \(G(P)\), i.e.,
\(n_1\) is reached by following a path of the form \(\gamma_1\gamma_2\dots\gamma_m q\),
where \(\gamma_i \in \{1,\dots,k\}\) and \(q  \in \{k + 1, \dots, n\}\).

\bigskip
Ans.

\newcommand{\Child}{\ensuremath{\text{Child}}}
\newcommand{\ChildG}{\ensuremath{\text{Child}_{\Gamma}}}
\newcommand{\ChildQ}{\ensuremath{\text{Child}_Q}}


Following the construction by \(t\) in the hint, we try to define Vert(\(x\)).
To simplify notations, we first define Child(\(x, x_{par}\)) predicate to denote if \(x\) is
a child of \(x_{par}\), and define \(\text{Child}_{\Gamma}/\text{Child}_{Q}\)
to denote a child with index in \([1..k]/[k+1...n]\).
\[
\Child(x, x_{par}) := \bigvee_{i=1}^n S_i~x_{par}~x ; \quad
\ChildG(x, x_{par}) := \bigvee_{i=1}^k S_i~x_{par}~x ; \quad
\ChildQ(x, x_{par}) := \bigvee_{i=k+1}^n S_i~x_{par}~x
\]
With Child(\(x, x_{par}\)), we further define root(\(x\)) to denote \(x\) is the root node
and Descend(\(x, x_{anc}\)) to denote \(x\) is a descendant of \(x_{anc}\).
For simplicity, we allow that \(x\) is a descendant of \(x\) itself.
\[
\text{root}(x) := \neg \exists x_{par}(\text{Child}(x, x_{par}))
\]
\[
\text{Descend}(x, x_{anc}) := \forall P ((P~x\land \forall z z_{par}(P~z \land \Child(z, z_{par}) \to P~z_{par})) \to P~x_{anc})
\]
Now we define Vert(\(x\)). Since every \( v \in V = \bot \Gamma^* Q\), we know \(x = t(v)\) must be some
\(i\)-th child of another node \(x_{par}\) where \(i > k\),
and every ancestor \(y\) of \(x_{par}\) must be some \(j\)-th child of its parent \(y_{par}\)
where \(j \leq k\), or \(y\) is the root node. Therefore,
\[
\begin{aligned}
	\text{Vert}(x) :=~& \exists x_{par}(\ChildQ(x, x_{par}) \land                                                                                             \\
	                  & \forall y (\text{Descend}(x_{par}, y) \to (\text{root}(y) \lor \forall y_{par}(\Child(y, y_{par}) \to \ChildG(y, y_{par}))))
\end{aligned}
\]

With Vert(\(x\)) predicate, we translate \(\varphi(x_1,\dots,x_s,X_1,\dots,X_t)\) to 
\(\varphi^*(x_1,\dots,x_s,X_1,\dots,X_t)\) with
\[
  \varphi^*(x_1,\dots,x_s,X_1,\dots,X_t) = 
     \bigwedge_{i=1}^s \text{Vert}(x_i) \land \bigwedge_{i=1}^t \forall y(X_i y \to \text{Vert}(y)) \land \psi^*
\]
to ensure that each \(x_i\) and \(X_i\) maps to a vertex in the graph of pushdown automaton.
Also \(\psi^*\) is inductively translated from \(\varphi\) with following cases.
\[
\begin{array}{lcl}
	\varphi                   &         & \psi^*                                                     \\
	\textbf{Base Cases:}      &  \\
	\bot                      & \mapsto & \bot                                                       \\
	x = y                     & \mapsto & x = y                                                      \\
	E x y                     & \mapsto & \text{Edge}(x, y)                                          \\
	X x                       & \mapsto & X x                                                        \\
	\textbf{Inductive Cases:} &  \\
	\varphi_1 \to \varphi_2   & \mapsto & \psi^*_1 \to \psi^*_2                                      \\
	\exists x \varphi         & \mapsto & \exists x(\text{Vert}(x) \land \psi^*)                     \\
	\exists X \varphi         & \mapsto & \exists X (\forall y(X y \to \text{Vert}(y)) \land \psi^*)
\end{array}
\]
Before we talk about the predicate Edge(\(x, y\)), we first discuss the usage of Vert\((x)\).
Note that Vert\((x)\) is only used like axioms for free and quantified variables.
This is because, if Vert\((x)\) is used in an arbitrary formula \(\varphi\),
negation of \(\varphi\) (using \(\varphi \to \bot\)) will likely break the Vert\((x)\) constraint.
Therefore, we introduce these Vert\((x_i)\) constraints outside the inductive construction
for free variables and at quantifiers for bounded variables.
As a result, Vert\((x)\) would not be broken by negation since it is never a sub-formula of negation,
or it is already evaluated to \textbf{true} or \textbf{false} due to quantifications.

Now we discuss how to translate an edge predicate.
From the given pushdown automaton \(P\), we already have the sets of transitions
\(\delta_\mathtt{push}\), \(\delta_\mathtt{pop}\), and \(\delta_\mathtt{nochg}\).
The transitions are defined on states \(q \in Q\) and alphabets \(\gamma \in \Gamma\).
For simplicity of our notations, we directly assume that any \(\gamma\) is also
a number in \(\{1,\dots, k\}\) and \(q\) is a number in \(\{k+1,\dots, n\}\).
The definition of Edge\((x, y)\) is given below.
Note that every transition is known and the number of transitions is finite.
The formula can be expanded to a large disjunction of conjuncts.

\[
\begin{array}{lll}
	\text{Edge}(x,y) = & \exists w \exists \alpha( \bigvee\limits_{(q_i, \gamma, q_j) \in \delta_\mathtt{push}} (S_{q_i} w x \land (S_\gamma w \alpha \land S_{q_j} \alpha y))\\
	                   & \lor \bigvee\limits_{(q_i, \gamma, q_j) \in \delta_\mathtt{pop}} ((S_\gamma w \alpha \land S_{q_i} \alpha x) \land S_{q_j} w y)\\
	                   & \lor \bigvee\limits_{(q_i, q_j) \in \delta_\mathtt{nochg}} (S_{q_i} w x \land S_{q_j} w y) )\\
\end{array}
\]

\end{document}
