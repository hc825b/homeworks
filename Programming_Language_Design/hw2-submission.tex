\documentclass[11pt]{homework}

\usepackage{bussproofs}
\usepackage{courier}
\def\defaultHypSeparation{\hskip 0.1in}

% TODO: replace these with your information
\newcommand{\hwname}{Chiao Hsieh}
\newcommand{\hwemail}{chsieh16@illinois.edu}
\newcommand{\hwtype}{Homework}
\newcommand{\hwnum}{2}
\newcommand{\hwclass}{CS 422: Programming Language Design}
\newcommand{\hwlecture}{}
\newcommand{\hwsection}{}

\renewcommand{\questiontype}{Problem}
\newcommand{\TT}{\ensuremath{\mathtt{true}}}
\newcommand{\FF}{\ensuremath{\mathtt{false}}}

\begin{document}
\maketitle

% Your content

\question
Give a Transition Semantics for SIMPL2 that is consistent with its Natural Semantics, and at the same level of
computation steps as given in the Transition Semantics for SIMPL1.  You may either give the whole description,
or describe the changes necessary to the Transition Semantics for SIMPL1 to create the Transition Semantics for
SIMPL2.  If you choose to describe how to change SIMPL1 semantics, your description must be clear as to what
changes are done where.

\newcommand{\BigCmd}[3]{\ensuremath{(\mathtt{\text{#1}}, \langle #2 \rangle)\Downarrow \langle \mathtt{#3} \rangle}}
\newcommand{\BigExpr}[3]{\ensuremath{(\mathtt{\text{#1}}, \langle #2 \rangle)\Downarrow #3 }}
\newcommand{\Map}[2]{\ensuremath{\mathtt{#1}\mapsto\mathtt{#2}}}

\begin{center}
    \BigCmd{while 0 < y do y := 0 - y od}{\Map{y}{2}}{\Map{y}{-2}}
\end{center}

Ans.

{\scriptsize

\begin{prooftree}
            \AxiomC{\BigExpr{0}{\Map{y}{2}}{0}}\noLine
            \UnaryInfC{\BigExpr{y}{\Map{y}{2}}{2}}\noLine
            \UnaryInfC{\(0 < 2 = \TT\)}
        \UnaryInfC{\BigExpr{0 < y}{\Map{y}{2}}{\TT}}
            \AxiomC{\BigExpr{0}{\Map{y}{2}}{0}}\noLine
            \UnaryInfC{\BigExpr{y}{\Map{y}{2}}{2}}\noLine
            \UnaryInfC{\(0 - 2 = -2\)}
        \UnaryInfC{\BigExpr{0 - y}{\Map{y}{2}}{-2}}
    \UnaryInfC{\BigCmd{y := 0 - y}{\Map{y}{2}}{\Map{y}{-2}}}
            \AxiomC{\BigExpr{0}{\Map{y}{-2}}{0}}\noLine
            \UnaryInfC{\BigExpr{y}{\Map{y}{-2}}{-2}}\noLine
            \UnaryInfC{\(0 < -2 = \FF\)}
        \UnaryInfC{\BigExpr{0 < y}{\Map{y}{-2}}{\FF}}
    \UnaryInfC{\BigCmd{while 0 < y do y := 0 - y od}{\Map{y}{-2}}{\Map{y}{-2}}}
\TrinaryInfC{\BigCmd{while 0 < y do y := 0 - y od}{\Map{y}{2}}{\Map{y}{-2}}}
\end{prooftree}

}

\question
Give a Context Semantics for SIMPL2 that is consistent with its Natural Semantics,
and at the same level of computation steps as given in the Context Semantics for SIMPL1.
You may either give the whole description, or describe the changes necessary to the
Transition Semantics for SIMPL1 to create the Transition Semantics for SIMPL2.
If you choose to describe how to change SIMPL1 semantics, your description must be clear
as to what changes are done where.

\newcommand{\TypChk}[3]{\ensuremath{\{#1\} \vdash \mathtt{\text{#2}} : \texttt{#3}}}
\newcommand{\Int}{\texttt{int}}
\newcommand{\Cmd}{\texttt{cmd}}
\newcommand{\Bool}{\texttt{bool}}
\newcommand{\TypInt}[1]{\ensuremath{\mathtt{#1}:\Int}}

\[
    \TypChk{\TypInt{y}}{while 0 < y do y := 0 - y od}{\Cmd}
\]

Ans.

{
    \begin{prooftree}
        \AxiomC{\TypChk{\TypInt{y}}{0}{\Int}}
        \AxiomC{\TypChk{\TypInt{y}}{y}{\Int}}
    \BinaryInfC{\TypChk{\TypInt{y}}{0 < y}{\Bool}}
            \AxiomC{\TypChk{\TypInt{y}}{0}{\Int}}
            \AxiomC{\TypChk{\TypInt{y}}{y}{\Int}}
        \BinaryInfC{\TypChk{\TypInt{y}}{0 - y}{\Int}}
    \UnaryInfC{\TypChk{\TypInt{y}}{y := 0 - y}{\Cmd}}
\BinaryInfC{\TypChk{\TypInt{y}}{while 0 < y do y := 0 - y od}{\Cmd}}
    \end{prooftree}
}

\end{document}
