\documentclass[11pt]{article}

% This is a package for drawing figures
% it is a part of standard latex 2e distribution
\usepackage{tikz}
\usetikzlibrary{shapes}
\usepackage{fullpage}


\usepackage{palatino}
\RequirePackage{ifthen}
\usepackage{latexsym}
\RequirePackage{amsmath}
\RequirePackage{amsthm}
\RequirePackage{amssymb}
\RequirePackage{xspace}
\RequirePackage{graphics}
\usepackage{xcolor}




\RequirePackage{textcomp}
\usepackage{keyval}
%\usepackage{listings}
\usepackage{xspace}
\usepackage{mathrsfs,paralist, amsmath,amssymb,url,listings,mathrsfs}
%\usepackage{pvs}
%\usepackage{supertabular,alltt,latexsym}
%\usepackage{multicol,multirow,epsfig}
%\usepackage[dvips, usenames]{color}
\usepackage{framed}
\usepackage{lipsum}
%\usepackage[dvipsnames]{color}

% copyright notice


\definecolor{reddish}{rgb}{1,.8,0.8}
\definecolor{blueish}{rgb}{0.8,.8,1}
\definecolor{greenish}{rgb}{.8,1,0.8}
\definecolor{yellowish}{rgb}{1,1,.20}


\usepackage[pdftex]{hyperref}
\hypersetup{
  pdftitle={Lecture notes for Modeling and Verification of Real-time and Hybrid Systems},
  pdfauthor={Sayan Mitra},
  colorlinks=true,
  citecolor={blue},
  linkcolor = {blue},
  pagecolor={blue},
  backref={true},
  bookmarks=true,
  bookmarksopen=false,
  bookmarksnumbered=true
}

%\newcommand{\remove}[1]{}

\input{prelude1}


\newcommand{\handout}[6]{
  \noindent
  \begin{center}
  \framebox{
    \vbox{
      \hbox to 5.78in { {\bf ECE/CS 584: Embedded and CPS  Verification } \hfill #2 }
      \vspace{4mm}
      \hbox to 5.78in { {\Large \hfill #5  \hfill} }
      \vspace{2mm}
       \hbox to 5.78in { {\Large \hfill #6  \hfill} }
      \vspace{2mm}
      \hbox to 5.78in { {\em #3 \hfill #4} }
    }
  }
  \end{center}
  \vspace*{4mm}
}

\newcommand{\smallheader}[4]{
  \noindent
  \begin{center}
  \framebox{
    \vbox{
      \hbox to 5.78in { {\bf ECE/CS 584: Embedded and CPS System Verification } \hfill #2 }
      \vspace{2mm}
      \hbox to 5.78in { {\em #3 \hfill #4} }
    }
  }
  \end{center}
  \vspace*{4mm}
}

\newcommand{\lecture}[4]{\handout{#1}{#2}{#3}{Scribe: #4}{Lecture #1}}

\newcommand{\homework}[2]{\smallheader{#1}{Spring 2016}{Homework #1}{#2}}
\newcommand{\solution}[2]{\smallheader{#1}{Spring 2016}{Solutions for Homework #1}{#2}}


\newcommand{\interestingfact}[1]{
	\noindent
	\begin{center}
	\colorbox{yellowish}{
	\parbox{11.5cm}{{\bf Factoid.} #1}
	}
	\end{center}
	\vspace*{4mm}
}
%\definecolor{MyGray}{rgb}{0.96,0.97,0.98}
\makeatletter\newenvironment{color1box}{%
   \begin{lrbox}{\@tempboxa}\begin{minipage}{\columnwidth}}{\end{minipage}\end{lrbox}%
   \colorbox{reddish}{\usebox{\@tempboxa}}
}\makeatother


\makeatletter\newenvironment{color3box}{%
   \begin{lrbox}{\@tempboxa}\begin{minipage}{\columnwidth}}{\end{minipage}\end{lrbox}%
   \colorbox{blueish}{\usebox{\@tempboxa}}
}\makeatother

% 1-inch margins, from fullpage.sty by H.Partl, Version 2, Dec. 15, 1988.
\topmargin 0pt
\advance \topmargin by -\headheight
\advance \topmargin by -\headsep
\textheight 8.9in
\oddsidemargin 0pt
\evensidemargin \oddsidemargin
\marginparwidth 0.5in
\textwidth 6.5in

\parindent 0in
\parskip 1.5ex
%\renewcommand{\baselinestretch}{1.25}

\begin{document}


\homework{1 on Discrete Models and Computation--- Due on February $12^{th}$, 2016}{Chiao Hsieh}

\paragraph{Problem 1 (Synchronous Dijkstra).}
The model of Dijkstra's token ring algorithm we presented in the lecture was asynchronous in the sense that each transition modeled the state-update of a single process. 
\begin{enumerate}[(a)]
\item Consider an initial state of the system with multiple tokens and write down two different executions of starting from that initial state.
States can be specified in terms of predicates on the state variables, e.g., $x[0] = 0 \wedge ..$. 

Ans.

An initial state with tokens on all $N$ processes (with $N>=3$) can be expressed as following predicate.
\[
x[0] = 0 \land \bigwedge_{i=1}^{N-2} x[i] = i \land x[N-1] = 0
\]

If we consider only 3 processes ($N=3$), the state will be
\(x[0] = 0 \land x[1] = 1 \land x[2] = 0\),
and all three processes satisfy the precondition to make the update.

\begin{itemize}
    \item Process $p_0$ updates first: $x[0] := (x[2] + 1) \mod k $ \\
          $x[0] = 0 \land x[1] = 1 \land x[2] = 0 \to x[0] = 1 \land x[1] = 1 \land x[2] = 0$
    \item Process $p_1$ updates first: $x[1] := x[0]$ \\
          $x[0] = 0 \land x[1] = 1 \land x[2] = 0 \to x[0] = 0 \land x[1] = 0 \land x[2] = 0$
    \item Process $p_2$ updates first: $x[2] := x[1]$\\
          $x[0] = 0 \land x[1] = 1 \land x[2] = 0 \to x[0] = 0 \land x[1] = 1 \land x[2] = 1$
\end{itemize}

\item Write the specification of a synchronous version of the same algorithm in which all the processes in the ring update their state simultaneously.

Ans.



\item Write the execution(s) of the synchronous model from the initial state in Part~(a).
\end{enumerate}

\paragraph{Problem 2 (LCR algorithm for Leader election).}
In this problem, you will create a model of a leader election algorithm in a unidirectional ring. Here is the informal description of the protocol:

Each process sends its identifier to its successor around the ring. When a process receives an incoming identifier, it compares that identifier to its own. If the incoming identifier is greater than its own, it keeps passing the identifier; if it is less than its own, it discards the incoming identifier; if it is equal to its own the process declares itself as the leader. 

\begin{enumerate}[(a)]
\item Write the model of the system with $n$ processes in the ring using the HIOA language. To get you started, the set of variables is: 
\begin{itemize}
\item $\mathit{send}$: The identifier to send or {\em null\/},
\item $\mathit{status}$: Takes values in $\{ \mathit{unknown},\mathit{leader}\}$ to indicate that the leader has been elected or not. 
\end{itemize}

\item Write an execution of the system in which status of at least one process is eventually set to $\mathit{leader\/}$. 
\item Write two properties that you think might be invariants of this system.
\end{enumerate}


\paragraph{Problem 3 (Multiplication).} 
Consider the following decision problem $\mathsf{Mult}$: Given binary
numbers $m,n,$ and $i$, determine if the $i$th bit of the binary
representation of $m\times n$ is $1$. As a language we could define
this as
\[
\mathsf{Mult} = \{(m,n,i)\: |\: \mbox{$i$th bit of $m\times n$ is $1$}\}
\]
Prove that $\mbox{Mult} \in \mbox{L}$ by giving the pseudo-code of an
algorithm and analyzing its memory requirements in terms of the number
of (additional, non-input) bits it stores.

\paragraph{Problem 4 (2SAT).}
Recall the following definitions for Boolean formulas. A variable $x$
or its negation $\neg x$ are called \emph{literals}. A \emph{clause}
is a disjunction of literals. A \emph{conjunctive normal form} (CNF)
formula is a conjunction of clauses. Note that because of de Morgan's
laws and distributivity of conjunction and disjunction, every Boolean
formula can be rewritten into an equivalent formula in CNF. A formula
$\varphi$ is said to be \emph{satisfiable} if there is a truth
assignment ${\bf a}$ to the variables such that $\varphi$ evaluates to
true under assignment ${\bf a}$.

A \emph{2CNF} formula is a CNF formula such that every clause has {\bf
  exactly} 2 literals. Let $\varphi$ be a 2CNF formula over variables
$X$. Associated with $\varphi$, we can define a directed graph $G =
(V,E)$, called the \emph{implication graph}, as follows. The vertices
of $G$ are the set of all literals over $X$. For literals
$\ell_1,\ell_2$, there is an edge $(\ell_1,\ell_2)$ in $G$ iff $(\neg
\ell_1 \vee \ell_2)$ is a clause in $\varphi$; here, we assume that
$\neg \neg x = x$.
\begin{enumerate}
\item Prove that $\varphi$ is unsatisfiable if and only if there is a
  variable $x$ such that there is a directed path from $x$ to $\neg x$
  in $G$
\item Based on the previous part, what can you say about the
  complexity of checking if a 2CNF formula is unsatisfiable? 
\end{enumerate}

\paragraph{Bonus Problem 5 (Invariant Checking of Boolean Automata).}
Consider an automaton ${\cal A} = (X,\Theta,A,{\cal D})$, where all
variables $x \in X$ have $\mathrm{type}(x)$ Boolean, $\Theta$ is given
by a Boolean predicate over $X$, and for each action $a \in A$, ${\cal
  D}$ has precondition/effect pair (as predicates over $X$) that
describe the transition on action $a$. Let $I$ be a predicate over $X$
as well. Let $\mathsf{Unsafe}$ be the decision problem of checking if
$I$ is {\bf not} an invariant of ${\cal A}$.
\begin{enumerate}
\item Prove that $\mathsf{Unsafe}$ is in $\mbox{PSPACE}$ by giving the
  pseudo-code of an algorithm. \emph{Hint:} Recall that since
  $\mbox{NPSPACE} = \mbox{PSPACE}$, your algorithm can be
  non-deterministic.
\item Prove that $\mathsf{Unsafe}$ is $\mbox{PSPACE}$-hard as
  follows. Let $L \in \mbox{PSPACE}$. Without loss of generality,
  assume that $M$ is 1-worktape Turing machine that solves $L$ using
  at most $p(n)$ worktape cells on an input of length $n$; here $p(n)$
  is assumed to be a polynomial. Show that $L \leq_{{\mbox L}}
  \mathsf{Unsafe}$. 
\end{enumerate}



%\paragraph($A \leq_L B, B \in C$}


\bibliography{sayan1}
\bibliographystyle{plain}
\end{document}
