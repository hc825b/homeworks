\documentclass[11pt]{article}

% This is a package for drawing figures
% it is a part of standard latex 2e distribution
\usepackage{tikz}
\usetikzlibrary{shapes}
\usepackage{fullpage}


\usepackage{palatino}
\RequirePackage{ifthen}
\usepackage{latexsym}
\RequirePackage{amsmath}
\RequirePackage{amsthm}
\RequirePackage{amssymb}
\RequirePackage{xspace}
\RequirePackage{graphics}
\usepackage{xcolor}




\RequirePackage{textcomp}
\usepackage{keyval}
%\usepackage{listings}
\usepackage{xspace}
\usepackage{mathrsfs,paralist, amsmath,amssymb,url,listings,mathrsfs}
%\usepackage{pvs}
%\usepackage{supertabular,alltt,latexsym}
%\usepackage{multicol,multirow,epsfig}
%\usepackage[dvips, usenames]{color}
\usepackage{framed}
\usepackage{lipsum}
%\usepackage[dvipsnames]{color}

% copyright notice


\definecolor{reddish}{rgb}{1,.8,0.8}
\definecolor{blueish}{rgb}{0.8,.8,1}
\definecolor{greenish}{rgb}{.8,1,0.8}
\definecolor{yellowish}{rgb}{1,1,.20}


\usepackage[pdftex]{hyperref}
\hypersetup{
  pdftitle={Lecture notes for Modeling and Verification of Real-time and Hybrid Systems},
  pdfauthor={Sayan Mitra},
  colorlinks=true,
  citecolor={blue},
  linkcolor = {blue},
  pagecolor={blue},
  backref={true},
  bookmarks=true,
  bookmarksopen=false,
  bookmarksnumbered=true
}

%\newcommand{\remove}[1]{}

\input{prelude1}


\newcommand{\handout}[6]{
  \noindent
  \begin{center}
  \framebox{
    \vbox{
      \hbox to 5.78in { {\bf ECE/CS 584: Embedded and CPS  Verification } \hfill #2 }
      \vspace{4mm}
      \hbox to 5.78in { {\Large \hfill #5  \hfill} }
      \vspace{2mm}
       \hbox to 5.78in { {\Large \hfill #6  \hfill} }
      \vspace{2mm}
      \hbox to 5.78in { {\em #3 \hfill #4} }
    }
  }
  \end{center}
  \vspace*{4mm}
}

\newcommand{\smallheader}[4]{
  \noindent
  \begin{center}
  \framebox{
    \vbox{
      \hbox to 5.78in { {\bf ECE/CS 584: Embedded and CPS System Verification } \hfill #2 }
      \vspace{2mm}
      \hbox to 5.78in { {\em #3 \hfill #4} }
    }
  }
  \end{center}
  \vspace*{4mm}
}

\newcommand{\lecture}[4]{\handout{#1}{#2}{#3}{Scribe: #4}{Lecture #1}}

\newcommand{\homework}[2]{\smallheader{#1}{Spring 2016}{Homework #1}{#2}}
\newcommand{\solution}[2]{\smallheader{#1}{Spring 2016}{Solutions for Homework #1}{#2}}


\newcommand{\interestingfact}[1]{
	\noindent
	\begin{center}
	\colorbox{yellowish}{
	\parbox{11.5cm}{{\bf Factoid.} #1}
	}
	\end{center}
	\vspace*{4mm}
}
%\definecolor{MyGray}{rgb}{0.96,0.97,0.98}
\makeatletter\newenvironment{color1box}{%
   \begin{lrbox}{\@tempboxa}\begin{minipage}{\columnwidth}}{\end{minipage}\end{lrbox}%
   \colorbox{reddish}{\usebox{\@tempboxa}}
}\makeatother


\makeatletter\newenvironment{color3box}{%
   \begin{lrbox}{\@tempboxa}\begin{minipage}{\columnwidth}}{\end{minipage}\end{lrbox}%
   \colorbox{blueish}{\usebox{\@tempboxa}}
}\makeatother

% 1-inch margins, from fullpage.sty by H.Partl, Version 2, Dec. 15, 1988.
\topmargin 0pt
\advance \topmargin by -\headheight
\advance \topmargin by -\headsep
\textheight 8.9in
\oddsidemargin 0pt
\evensidemargin \oddsidemargin
\marginparwidth 0.5in
\textwidth 6.5in

\parindent 0in
\parskip 1.5ex
%\renewcommand{\baselinestretch}{1.25}

\newcommand{\st}{\mathsf{state}}
\newcommand{\cntr}{\mathsf{counter}}
\newcommand{\execcm}{\Longrightarrow}



\begin{document}


\homework{3 PVS and Abstractions--- Due on March  $29^{th}$, 2016}{Chiao Hsieh}


\paragraph{Problem 1 (20 points).}
Consider a linear dynamical system $\dot{x} = Ax$, where $x \in \reals^n$ and $A$ is a $n \times n$ matrix. 

\begin{enumerate}[(a)]
\item Define the corresponding time-abstract automaton (discrete transition system),

Ans.

Since there is no discrete transitions in the system,
the time-abstract automaton for the system is essentially composed of all states reachable by the continuous transitions.
Let $\A = (x, \Theta, D)$ be a time-abstract automaton for the linear dynamical system defined by $\dot{x} = Ax$.
The initial states $\Theta$ should be the same as the initial states of the given dynamical system.
The discrete transitions $D$ is defined as
\[
D := \{(x_1, x_2) \mid \exists x_0, t_1, t_2(x_0 \in \Theta \land 0 \leq t_1 < t_2 \land x_1 = e^{A t_1}x_0 \land x_2 = e^{A t_2}x_0 )\}
\]

\item Suppose the set of initial states $\Theta$ is a convex polytope, that is, a set described by a set of linear inequalities.
For any time $t \geq 0$, show that the states reached by the system at time $t$, $\mathit{Reach}(t)$, is also a polytope.
 
Ans.

Assuming $\Theta$ is describe by a set of linear inequalities $B_0x_0 \leq b_0$ where $B_0$ is an $m \times n$ matrix and $b_0$ is an $m \times 1$ vector,
we prove that there exists another $B_1$ and $b_1$ s.t. $B_1x_1 \leq b_1$ for any reachable state $x_1 = e^{At} x_0$.
It leads to the result that $\mathit{Reach}(t)$ is also a convex polytope.

The prove is simple. Since the $n \times n$ matrix $e^{A}$ is always invertible and the inverse is $e^{-A}$,
we can derive $e^{-At}x_1 = e^{-At}e^{At}x_0 = x_0$. Therefore,
\[
 B_0x_0 \leq b_0 \implies B_0 (e^{-At} x_1) \leq b_0 \implies (B_0 e^{-At}) x_1 \leq b_0
\]
We can always find $B_1 = B_0 e^{-At}$ and $b_1 = b_0$ that shows $\mathit{Reach}(t)$ is also a convex polytope.

\end{enumerate}

 \paragraph{Problem 2 (20 points).} 
 Consider the billiard system with a single ball $\A$ and the satellite system $\B$ from HW2. Show that under appropriate assumptions, and variable and action naming, $\B$ is an I/O abstraction for $\A$. There some flexibility in how you set-up the problem. To make it interesting you should assume that both the velocity components of the billiard ball are non-zero ($v_x, v_y \neq 0$). The relevant part of the descriptions of the two systems are given below:
 
 \paragraph{Satellites}
 Consider two satellites orbiting the earth on circular orbits with (constant) angular speeds $\omega_1$ and $\omega_2$.  Write a hybrid automaton model of the position of the satellite-pair in the $[0,2\pi]^2$ space. When one of the satellites hit $0$ or $2\pi$ its position has to be reset.
 % Model this using an ouput action called $\act{jump}$.  
 
 \paragraph{Billiards}
 Consider a idealized billiard table of length $a$ and width $b$ and a ball rolling on the table with constant velocity $\vec{v} = (v_x, v_y)$. The table has no pockets, no friction, and it's boundary bounces the balls perfectly. Wall bounces are modeled by an output action called $\act{jump}$. 
 
 Ans.
 
Let the automaton of billiard system $\A = (X_1, \Theta_1, \{\act{jump}\}, D_1, \T_1)$ where
\[
\begin{array}{rcl}
	     X_1 & := & \{x, y, v_x, v_y\} \text{ where } \type{x} \in [0, a], \type{y} \in [0, b], \type{v_x}, \type{v_y} \in \reals \\
	\Theta_1 & := & (v_x \neq 0 \land v_y \neq 0)                                                                                 \\
	     D_1 & := & \{(0, y, v_x, v_y) \arrow{\act{jump}} (0, y, -v_x, v_y) \mid v_x < 0\}                               \\
&& \cup         \{(a, y, v_x, v_y) \arrow{\act{jump}} (a, y, -v_x, v_y) \mid v_x > 0\}                               \\
	         &    & \cup \{(x, 0, v_x, v_y) \arrow{\act{jump}} (x, 0, v_x, -v_y) \mid v_y < 0\}                          \\
 	         &    & \cup \{(x, b, v_x, v_y) \arrow{\act{jump}} (x, b, v_x, -v_y) \mid v_y > 0\}                          \\
\end{array}
\]
Trajectory $\T_1$ is defined by $\dot{x} = v_x, \dot{y} = v_y, \dot{v_x} = 0, \dot{v_y} = 0$ with the invariant $0 \leq x \leq a$ and $0 \leq y \leq b$.

The automaton of satellite system is defined as $\B = (X_2, \Theta_2, \{\act{jump}\}, D_2, \T_2 \})$.
$x$ denotes the position of the first satellite, and $y$ denotes the position of the second one.
\[
\begin{array}{rcl}
	     X_2 & := & \{x, y, \omega_1, \omega_2\} \text{ where } \type{x}, \type{y} \in [0, 2\pi], \type{\omega_1}, \type{\omega_2} \in \reals \\
	\Theta_2 & := & (\omega_1 \neq 0 \land \omega_2 \neq 0)                                                                                                   \\
	     D_2 & := & \{(0, y, \omega_1, \omega_2) \arrow{\act{jump}} (2\pi, y, \omega_1, \omega_2) | \omega_1 < 0\}                                    \\
	         &    & \cup \{(2\pi, y, \omega_1, \omega_2) \arrow{\act{jump}} (0, y, \omega_1, \omega_2) | \omega_1 > 0\}                               \\
	         &    & \cup \{(x, 0, \omega_1, \omega_2) \arrow{\act{jump}} (x, 2\pi, \omega_1, \omega_2) | \omega_2 < 0\}                               \\
	         &    & \cup \{(x, 2\pi, \omega_1, \omega_2) \arrow{\act{jump}} (x, 0, \omega_1, \omega_2) | \omega_2 > 0\}
\end{array}
\]
Trajectory $\T_2$ is defined by $\dot{x} = \omega_1, \dot{y} = \omega_2, \dot{\omega_1} = 0, \dot{\omega_2} = 0$  with invariant $0 \leq x, y \leq 2\pi$.

To simplify our problem, we prove $\B$ is an I/O abstraction for $\A$ when $a = 2\pi$ and $b = 2\pi$.
Giving a forward simulation relation $\R \subseteq val(X_1) \times val(X_2)$ which is defined as
\[
\begin{array}{rcl}
	 \R & := & R_1 \cup R_2 \cup R_3                                                                              \\
	R_1 & := & \{((x, y, v_x, v_y), (x, y, \omega_1, \omega_2)) \mid (v_x = \omega_1) \land (v_y = \omega_2)\}    \\
	R_2 & := & \{((x_1, y, v_x, v_y), (x_2, y, \omega_1, \omega_2)) \mid (v_x = -\omega_1) \land (v_y = \omega_2) \\
	    &    & \land ((x_1 = 0 \land x_2 = 2\pi) \lor (x_1 = 2\pi \land x_2 = 0))\}                               \\
	R_3 & := & \{((x, y_1, v_x, v_y), (x, y_2, \omega_1, \omega_2)) \mid (v_x = \omega_1) \land (v_y = -\omega_2) \\
	    &    & \land ((y_1 = 0 \land y_2 = 2\pi) \lor (y_1 = 2\pi \land y_2 = 0))\}
\end{array}
\]
We argue that $\R$ is a simulation relation following the three parts described in the lecture.
\begin{enumerate}
\item $\forall \vx_1(\vx_1 \in \Theta_1 \to \exists \vx_2(\vx_2 \in \Theta_2 \land \vx_1 \R \vx_2))$

It's easy to show that, for any initial state $\vx_1 = (x, y, v_x, v_y) \in \Theta_1$,
we can found $\vx_2 = (x, y, \omega_1, \omega_2) \in \Theta_2$ with $\omega_1 = v_x$ and $\omega_2 = v_y$
so that $\vx_1 \R \vx_2$.

\item $\forall a,\vx_1,\vx_1',\vx_2(((\vx_1 \sarrow{\A}{a} \vx_1') \land \vx_1 \R \vx_2) \to \exists \beta, \vx_2'(\vx_2  \sarrow{\B}{\beta} \vx_2' \land \vx_1' \R \vx_2' \land trace(\beta) = a))$

Since we have only one action $\act{jump}$ for both automata,
it is sufficient to consider that, for each discrete transition $\vx_1 \sarrow{\A}{\act{jump}} \vx_1'$ and $\vx_1 \R \vx_2$,
there is a discrete transition $\vx_2 \sarrow{\B}{\act{jump}} \vx_2'$ such that $\vx_1' \R \vx_2'$.

We first discuss $\vx_1 = (0, y, v_x, v_y) \arrow{\act{jump}} (0, y, -v_x, v_y) = \vx_1'$ when $v_x < 0$ in $\A$,
and we prove that there always exists $\vx_2'$ s.t. $\vx_2 \sarrow{\B}{\act{jump}} \vx_2'$.
Following the definition of $R_1$ and $R_2$ in $\R$, $\vx_2$ can be $(0, y, v_x, v_y)$ or $(2\pi, y, -v_x, v_y)$.
We can derive $\vx_2'$ in $\B$, that is, $(2\pi, y, v_x, v_y)$ or $(0, y, -v_x, v_y)$.

Finally, by $R_2 \subseteq \R$, we know $\vx_1' \R \vx_2'$ for either case of $\vx_2'$.
We can prove for other discrete transitions of $\A$ in the same manner.

\item $\forall \tau_1, \vx_2(\tau_1.\fstate \R \vx_2 \to 
       \exists \tau_2(\tau_2.\fstate = \vx_2 \land \tau_1.\lstate \R \tau_2.\lstate \land trace(\tau_1) = trace(\tau_2)))$

To prove this property, we consider the 3 parts of $\R$ separately.

$R_1$ is essentially the identity function when we specify $a = b = 2\pi$
since the types of variables and trajectories are the same for both automata;
therefore, for each $\tau_1.\fstate R_1 \vx_2$, there must be $\tau_2 = \tau_1$ with the same first and last state and same trace.

For $R_2$, we have to further consider the following 2 conditions.

(1) $\vx_1 = (0, y, v_x, v_y)$ and $\vx_2 = (2\pi, y, -v_x, v_y)$ with $v_x > 0$.

(2) $\vx_1 = (2\pi, y, v_x, v_y)$ and $\vx_2 = (0, y, -v_x, v_y)$ with $v_x < 0$.

(We don't have to consider $v_x < 0$ for (1) and $v_x > 0$ for (2) since they will enable discrete transition $\act{jump}$.)
{
\color{red}

It seems the relation $\R$ is not a simulation relation because the trajectory of $x$ and $y$ does not satisfy part of the property that 
$\forall \tau_1 \exists \tau_2(\tau_1.\fstate \R \tau_2.\fstate \land \tau_1.\lstate \R \tau_2.\lstate)$.
because $v_x$ and $\omega_1$ are in opposite directions. 
}
\end{enumerate}



 \paragraph{Problem 3. (30 points)} In this problem, you will model the $n$-process distributed token ring system (from last problem set) in PVS  and prove its key invariant. 

\paragraph{Recall the system description.} Consider $n$ processes $0, \ldots, n-1$ connected in a directed ring. 
We say process $i+1 \mod n$ is the successor of process $i$. Each process $i$, has a value $v_i$ which can be an element of the set $\{0, \ldots, k\}$ for some $k > n$. Each process behaves as follows: Process $i$, $i \neq 0$, is said to have a token iff $v_i \neq v_{i- 1}$. Process $0$ has a token iff $v_0 = v_{n-1}$. Each process has a real-valued period parameter  $\Delta_i > 0$. 
Exactly every $\Delta_i$ time, process $i$ performs the following action if it has the token: 
if $i = 0$ then $v_i := (v_i + 1) \mod n$, otherwise $v_i := v_{i- 1}$. 

\paragraph{Part (a)} 
A partially complete PVS specification of the system is provided from the homeworks page. This specification, albeit not correct, should parse and typecheck in PVS. Complete the specification with appropriate expressions in the lines marked ``Fill in''.

\paragraph{Part (b)} Check for type correctness. Are there any unproved TCCS ? Prove them. You may have to use basic lemmas on modular arithmetic from \texttt{prelude.pvs}.

\paragraph{Part (c)} The predicate {\em two\_val\/} implies that there is at most one token in the system. Prove invariance of {\em two\_val\/} using the PVS prover. This is broken down into two lemmas in the supplied theory. Partial proof are provided to get you started.



%\paragraph($A \leq_L B, B \in C$}


%\bibliography{sayan1}
%\bibliographystyle{plain}
\end{document}
