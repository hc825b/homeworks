\newcommand{\eng}[1]{\ \raisebox{1pt}{#1}}

\chapter{摘要}
在程序分析 \eng{(Program Analysis)} 領域中,
分析遞迴程式 \eng{(Recursive Program)} 是一項難以處理的課題。
現有程式分析工具往往因為無法處理程式中的遞迴函式,
既而忽略部分程式碼不進行分析,或甚至拒絕檢驗遞迴程式。
本篇論文針對遞迴程式的分析提出新的觀點與分析方法;
作者認為與其重新設計演算法且實作新的分析器,
不如改進現有的程式分析工具。
但更改他人實作的軟體工具並非易事,
因此本文提出的方法將現有的工具視為黑箱不做任何更動,
反而是利用程式變換方法 \eng{(Program Transformation)} 將待分析的遞迴
程式轉換成無遞迴的程式,再交給黑箱分析工具進行檢驗;
並使用黑箱工具提供被分析程式中的不變量 \eng{(Invariants)},
進而證明原遞迴程式的正確性。

本篇作者與實驗室團隊受惠於程式分析工具 \eng{\textsc{CPAchecker}},
實作出此演算法的雛型 \eng{\textsc{CPArec}},
且參與 \eng{2015} 年度軟體驗證競賽 \eng{(Competition on Software Verification)};
和其它頂尖實驗室所開發工具相較,我們的工具針對遞迴程式進行分析時,
亦表現出相當的效率及效能,獲得第三名的佳績。

關鍵字:軟體驗證、程序分析、靜態分析、遞迴程式、程式變換方法
