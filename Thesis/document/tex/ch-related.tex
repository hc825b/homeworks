
\chapter{Related Works}\label{ch:related}

In~\cite{LalR08,LalR09}, a program transformation technique for checking context-bounded concurrent programs to sequential analysis is developed. Numerous intraprocedural analysis techniques have been developed over the years. Many tools are in fact freely available (see, for instance, \textsc{Blast}~\cite{BeyerHJM07}, \textsc{CPAChecker}~\cite{BeyerK11}, and \textsc{UFO}~\cite{AlbarghouthiLGC12}). Interprocedural analysis techniques are also available (see~\cite{RepsHS95,BallR01,CousotCFMMMR05,CuoqKKPSY12,coverity,polyspace} for a partial list). Recently, recursive analysis attracts new attention. The Competition on Software Verification adds a new category for recursive programs in 2014~\cite{svcomp14}. Among the participants, \textsc{CBMC}~\cite{ClarkeKL04}, \textsc{Ultimate Automizer}~\cite{HeizmannCDEHLNSP13}, and \textsc{Ultimiate Kojak}~\cite{ErmisNDHP14} are the top three tools for the \textbf{recursive} category.

Inspired by \textsc{Whale}~\cite{AlbarghouthiGC12}, we use inductive invariants obtained from verifying under-approximation as candidates of summaries. Also, similar to \textsc{Whale}, we apply a Hoare logic proof rule for recursive calls from~\cite{Oheimb99}. However, our technique works on control flow graphs and builds on an intraprocedural analysis tool. It is hence very lightweight and modular. Better intraprocedural analysis tools easily give better recursive analysis through our technique. \textsc{Whale}, on the other hand, analyzes by exploring abstract reachability graphs. Since \textsc{Whale} extends summary computation and covering relations for recursion, its implementation is more involved.
% Although \textsc{Whale} is able to analyze recursive program in theory, its implementation does not appear to support this feature.
