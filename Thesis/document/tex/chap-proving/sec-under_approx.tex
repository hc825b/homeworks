
\section{Under-approximation}\label{sec:under-approximation}
In this section, we first introduce how to under-approximate a given function
$\fun{f}$ with $\underline{G}^\fun{f}=\method{under}(G^\fun{f})$.
The produced CFG $\underline{G}^\fun{f}$ should not contain any recursive calls,
and all computations in $\underline{G}^\fun{f}$ should be valid in original CFG
$G^\fun{f}$.
Further, we explain how we use \method{under} combined with two unwinding
methods to derive under-approximation of the unwound program $\prog{P}_k$.


\subsection{Approximating Functions}
Let $G^\fun{f} = \langle V, E, \cmd{cmd}^\fun{f}, \overline{\texttt{u}}^\fun{f},
\overline{\texttt{r}}^\fun{f},s,e \rangle$ be a control flow graph.
The control flow graph $\underline{G}^\fun{f} = \method{under}(G^\fun{f}) = 
\langle V, E,\underline{\cmd{cmd}}^\fun{f}, \overline{\texttt{u}}^\fun{f},
\overline{\texttt{r}}^\fun{f},s,e \rangle$ is obtained by replacing every
function call in $G$ with $\mathtt{assume\ false}$
(Figure~\ref{figure:under-approximation}).
That is,
\begin{equation*}
  \underline{\cmd{cmd}}^\fun{f} (\ell, \ell') =
  \left\{
    \begin{array}{ll}
      \mathtt{assume\ false} &
      \text{if } \cmd{cmd}^\fun{f} (\ell, \ell')
      \text{ contains function calls}\\
      \cmd{cmd}^\fun{f} (\ell, \ell') &
      \text{otherwise}
    \end{array}
  \right.
\end{equation*}

\begin{figure}[t]
  \centering
    \begin{tikzpicture}[scale=1.2,->,>=stealth',shorten >=1pt,auto,
      node distance=2cm,thick,
      node/.style={circle,draw,minimum width=0.8cm,inner sep=0}]

      \node[node] (0) at (-4, 0)  {$\ell$};
      \node[node] (1) at (-4, -2) {$\ell'$};

      \node[node] (00) at (0, 0)  {$\ell$};
      \node[node] (01) at (0, -2) {$\ell'$};
      \node (arrow_s) at (-2.5, -1) {};
      \node (arrow_e) at (-0.5, -1) {};

      \path
        (arrow_s) edge [dotted]
                  node {} (arrow_e)
        (0) edge 
            node {$\overline{\mathtt{x}} := \fun{g} (\overline{p})$} (1)

        (00) edge 
             node {$\mathtt{assume\ false}$} (01);
    \end{tikzpicture}

  \caption{Under-approximation}
  \label{figure:under-approximation}
\end{figure}

\begin{proposition}
  Let $G^\fun{f}$ be a control flow graph. $P$ and $Q$ are logic formulae with
  free variables over program variables of $G^\fun{f}$. If $\assert{P}
  G^\fun{f} \assert{Q}$, then 
  $\assert{P} \underline{G}^\fun{f} \assert{Q}$.
\end{proposition}
The above holds because the computation of $\underline{G}^\fun{f}$ under-
approximates the computation of $G^\fun{f}$.
If all computation of $G^\fun{f}$ from a state satisfying $P$ always ends with
a state satisfying $Q$,
the same should also hold for the computation of $\underline{G}^\fun{f}$.


\subsection{Approximating Programs from Function Inlining Method}
Following how we do unwinding in \S~\ref{subsec:inlining}, we can under-
approximate the program through simply computing under-approximation of
$\fun{main}$ function as below.
\[
\prog{\underline{P}}_k = \{\underline{G}^\fun{main}_k\}
\]
$\underline{G}^\fun{main}_k$ doesn't contain any function call,
so it is unnecessary to copy other functions from $\prog{P}_k$.
Thus, the unwound program can only contains $\fun{main}$ for less memory usage.

\todo[inline]{Prove that, in each iteration, the under-approximation is more
accurate than previous one.}

\subsection{Approximating Programs from Function Duplicating Method}

