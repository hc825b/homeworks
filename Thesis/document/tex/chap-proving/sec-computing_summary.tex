
\section{Computing Summaries}\label{sec:computing-summary}
\todo[inline]{Rewrite marking function}
In order to help extract summaries from the inductive invariant,
$\method{unwind}(G^\fun{f})$ annotates in $K^\fun{f}$ the outermost pair of the
entry and exit locations ${s_i}$ and ${e_i}$ of each unwound function $\fun{g}$
with an additional superscript $\fun{g}$, i.e., $s_i^\fun{g}$ and $e_i^\fun{g}$.
The function $\method{mark}_\fun{f}(G^\fun{g})$ returns a CFG that is identical
to $G^\fun{g}$, except that, for the case that no location with superscript
$\fun{g}$ appears in $V$ (the locations of $G^\fun{f}$), it annotates the entry
and exit locations, $s_k$ and $e_k$, of the returned CFG with superscript
$\fun{g}$, i.e., $s_k^\fun{g}$ and $e_k^\fun{g}$.
That is,
\[
\method{mark}_\fun{f}(\langle
V, E,\cmd{cmd}^\fun{g}, \overline{\texttt{u}}^\fun{g}, \overline{\texttt{r}}^\fun{g},s,e \rangle) =
\langle V, E,\cmd{cmd}^\fun{g}, \overline{\texttt{u}}^\fun{g}, \overline{\texttt{r}}^\fun{g},s,e \rangle
\]
Note that, for each unwinding of function call,
we mark only the outermost pair of its entry and exit locations.

\missingfigure{Mark the program locations to extract invariants}

Let the CFG for the $\fun{main}$ function $\underline{G}^{\fun{main}}_k = \langle V, E, \underline{\cmd{cmd}}^{\fun{main}}, \overline{\mathtt{u}}^{\fun{main}}, \overline{\mathtt{r}}^{\fun{main}},s,e \rangle$.
Function ComputeSummary($\prog{P}_k$, $\Pi (\underline{G}^{\fun{main}}_k, \TT)$)
extracts summaries from the inductive invariant
$\Pi (\underline{G}^{\fun{main}}_k, \TT)= \{ I_\ell : \ell \in V\}$
(Algorithm~\ref{algorithm:compute-summary}).

\begin{algorithm}
\begin{doublespace}
  \KwIn
  {
    $\prog{P}_k$: a program;
    $\{ I_\ell : \ell \in V \}$: an inductive invariant of $\underline{G}^{\fun{main}}_k$
  }
  \KwOut{$S[\bullet]$: function summaries}

  \ForEach{function $\fun{f}$ in the program $\prog{P}_k$}
  { 
    $S[\fun{f}]$ := $\TT$\;
    \ForEach{pair of locations $(s_i^\fun{f},e_i^\fun{f})\in V\times V$}
    {
      \If{$I_{s_i^\fun{f}}$ contains return variables of $\fun{f}$}
      {
        $S[\fun{f}]$ := $S[\fun{f}]\wedge \forall X_\fun{f}. I_{e_i^\fun{f}}$
      }
      \Else
      {
        $S[\fun{f}]$ := $S[\fun{f}]\wedge \forall X_\fun{f}. (I_{s_i^\fun{f}} \implies I_{e_i^\fun{f}})$
      }
    }
    
  }
 
  \Return $S[\bullet]$\;
\end{doublespace}
  \caption{
  $\textmd{ComputeSummary}(\prog{P}_k, \Pi (\underline{G}^{\fun{main}}_k, \TT))$}
  \label{algorithm:compute-summary}
\end{algorithm}

For each function $\fun{f}$ in the program $\prog{P}_k$,
we first initialize its summary $S[\fun{f}]$ to $\TT$.
The set $X_\fun{f}$ contains all variables appearing in
$\underline{G}^{\fun{main}}_k$ except the set of formal parameters and return
variables of $\fun{f}$.
For each pair of locations $(s_i^\fun{f},e_i^\fun{f})\in V\times V$ in $\underline{G}^{\fun{main}}_k$,
if the invariant of location $s_i^\fun{f}$ contains return variables of $\fun{f}$,
we update $S[\fun{f}]$ to the formula $S[\fun{f}]\wedge \forall X_\fun{f}.I_{e_i^\fun{f}}$.
Otherwise, we update it to a less restricted version $S[\fun{f}]\wedge \forall X_\fun{f}. (I_{s_i^\fun{f}} \implies I_{e_i^\fun{f}})$ (Figure~\ref{figure:updating-summary}).

\begin{figure}[t]
  \centering
    \begin{tikzpicture}[scale=1.2,->,>=stealth',shorten >=1pt,auto,node
      distance=2cm,thick,node/.style={circle,draw,minimum width=0.8cm,inner sep=0}]

      \draw [fill=gray!10] (3.5, -1) ellipse (1.8 and 1.5);
      \node (text) at (3.5, -1) {$\method{mark}_\fun{f}(\method{rename}(G_\fun{f},i))$};
      \node[node] (00) at (0, 0)  {$\ell$};
      \node[node] (01) at (0, -2) {$\ell'$};
      \node[node] (10) at (3.5, 0)  {$s_i^\fun{f}$};
      \node[node] (11) at (3.5, -2) {$e_i^\fun{f}$};

      \node (arrow_s) at (5.5, -1) {};
      \node (arrow_e) at (6.5, -1) {};
      
      \node[rectangle,text centered,draw] (text) at (9, -1)
      { add $\forall X_{\fun{f}}.(I_{s_i^\fun{f}} \implies I_{e_i^\fun{f}})$
        to $S[\fun{f}]$}; 
      
      \path
        (arrow_s) edge [dotted]
                  node {} (arrow_e)
        (00) edge 
             node {$\overline{\mathtt{u}}_i^\fun{f} := \overline{p}$} (10)

        (11) edge 
             node {$\overline{\mathtt{x}} := \overline{\mathtt{r}}_i^\fun{f}$} (01) ;
    \end{tikzpicture}

  \caption{Updating a Summary}
  \label{figure:updating-summary}
\end{figure}
\begin{proposition}
\label{propposition:invariant}
Let $Q$ be a formula over all variables in $\underline{G}^{\fun{main}}_k$ except $\overline{\mathtt{r}}^\fun{f}$.
We have $\assert{Q}\
  \overline{\mathtt{r}}^\fun{f} := \fun{f}
  (\overline{\mathtt{u}}^\fun{f})\ \assert{Q}$.
\end{proposition}
The proposition holds because the only possible overlap of variables in $Q$ and
in $\overline{\mathtt{r}}^\fun{f} := \fun{f} (\overline{\mathtt{u}}^\fun{f})$
are the formal parameters $\overline{\mathtt{u}}^\fun{f}$.
However, we assume that values of formal parameters do not change in a function
(see Section~\ref{ch:preliminaries});
hence the values of all variables in $Q$ stay the same after the execution of
the function call $\overline{\mathtt{r}}^\fun{f} := \fun{f} (\overline{\mathtt{u}}^\fun{f})$.

\begin{proposition}
  \label{propposition:strengthen_postcondition}
  Given the CFG $\underline{G}^{\fun{main}}_k = \langle V, E, \underline{\cmd{cmd}}^{\fun{main}}, \overline{\mathtt{u}}^{\fun{main}}, \overline{\mathtt{r}}^{\fun{main}},s,e \rangle$.
  If $\assert{\TT}\ \overline{\mathtt{r}}^\fun{f} := \fun{f}
     (\overline{\mathtt{u}}^\fun{f})\ \assert{S[\fun{f}]}$ holds, then
  $\assert{I_{s_i^\fun{f}}}\ \overline{\mathtt{r}}^\fun{f} := \fun{f}
   (\overline{\mathtt{u}}^\fun{f})\ \assert{
     I_{e_i^\fun{f}}}$ for all $(s_i^\fun{f},e_i^\fun{f})\in V\times V$.
\end{proposition}
For each pair $(s_i^\fun{f},e_i^\fun{f})\in V\times V$, we consider two cases:
\begin{enumerate}
\item $I_{s_i^\fun{f}}$ contains some return variables of $\fun{f}$:\\
In this case, the conjunct $\forall X_\fun{f}.I_{e_i^\fun{f}}$ is a part of $S[\fun{f}]$, we then have
\begin{prooftree}
  \AxiomC{$\assert{\TT}\ \overline{\mathtt{r}}^\fun{f} := \fun{f}
       (\overline{\mathtt{u}}^\fun{f})\ \assert{S[\fun{f}]}$ }
  \RightLabel{Postcondition Weakening}
    \UnaryInfC{$\assert{\TT}\ \overline{\mathtt{r}}^\fun{f}:=\fun{f}(\overline{\mathtt{u}}^\fun{f})\ \assert{\forall X_\fun{f}. I_{e_i^\fun{f}}}$}
  \RightLabel{Postcondition Weakening}
  \UnaryInfC{$\assert{\TT}\ \overline{\mathtt{r}}^\fun{f} := \fun{f}(\overline{\mathtt{u}}^\fun{f})\ \assert{I_{e_i^\fun{f}}}$}
  \RightLabel{Precondition Strengthening}
    \UnaryInfC{$\assert{I_{s_i^\fun{f}}}\ \overline{\mathtt{r}}^\fun{f} := \fun{f}(\overline{\mathtt{u}}^\fun{f})\ \assert{I_{e_i^\fun{f}}}$}
\end{prooftree}

\item $I_{s_i^\fun{f}}$ does not contain any return variables of $\fun{f}$:\\
In this case, the conjunct $\forall X_\fun{f}.(I_{s_i^\fun{f}} \implies I_{e_i^\fun{f}})$ is a part of $S[\fun{f}]$, we then have
\begin{prooftree}
 \AxiomC{}
 \LeftLabel{Prop.~\ref{propposition:invariant}}
 \UnaryInfC{$\assert{I_{s_k^\fun{f}}}\ 
    \overline{\mathtt{r}}^\fun{f} := \fun{f} (\overline{\mathtt{u}}^\fun{f})\
    \assert{I_{s_k^\fun{f}}}$}
\AxiomC{$\assert{\TT}\ \overline{\mathtt{r}}^\fun{f} := \fun{f}
       (\overline{\mathtt{u}}^\fun{f})\ \assert{S[\fun{f}]}$ }
  %\RightLabel{Weakening
\UnaryInfC{$\assert{\TT}\ \overline{\mathtt{r}}^\fun{f}:=\fun{f}(\overline{\mathtt{u}}^\fun{f})\ \assert{\forall X_\fun{f}. (I_{s_k^\fun{f}} \implies I_{e_k^\fun{f}})}$}
%\RightLabel{Weakening}
\UnaryInfC{$\assert{\TT}\ \overline{\mathtt{r}}^\fun{f}:=\fun{f}(\overline{\mathtt{u}}^\fun{f})\ \assert{I_{s_k^\fun{f}} \implies I_{e_k^\fun{f}}}$}
%\RightLabel{Strengthening}
 \UnaryInfC{$\assert{I_{s_k^\fun{f}}}\
    \overline{\mathtt{r}}^\fun{f} := \fun{f} (\overline{\mathtt{u}}^\fun{f})\
    \assert{I_{s_k^\fun{f}} \implies I_{e_k^\fun{f}}}$}

 \BinaryInfC{$\assert{I_{s_k^\fun{f}}}\ 
    \overline{\mathtt{r}}^\fun{f} := \fun{f} (\overline{\mathtt{u}}^\fun{f})\
    \assert{I_{e_k^\fun{f}}}$}
\end{prooftree}
\end{enumerate}

