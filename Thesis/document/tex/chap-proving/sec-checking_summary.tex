
\section{Checking Summaries}\label{sec:checking-summary}

Here we explain how to handle the function $\textmd{CheckSummary} (\prog{P}_k, S[\bullet])$, where $\prog{P}_k$ is an unwound program and $S[\bullet]$ is an
array of function summaries.
Let $G^\fun{f}_k = \langle V, E, \cmd{cmd}^\fun{f}, \overline{\mathtt{u}}^\fun{f}, \overline{\mathtt{r}}^\fun{f},s,e \rangle$ be a
control flow graph for the function $\fun{f}$ in $\prog{P}_k$. In order to check whether the function
summary $S[{\fun{f}}]$ for $\fun{f}$ specifies the relation
between the formal parameters and return values of $\mathtt{f}$, 
we define another control flow graph
$\hat{G}^\fun{f}_{k,S} = \langle V, E, \hat{\cmd{cmd}}^\fun{f}, \overline{\mathtt{u}}^\fun{f}, \overline{\mathtt{r}}^\fun{f},s,e \rangle$ where
\begin{equation*}
  \begin{array}{rcl}
    \hat{\cmd{cmd}}^\fun{f} (\ell, \ell') & = &
    \left\{
      \begin{array}{ll}
        \overline{\mathtt{x}} := 
        \overline{\mathtt{nondet}};
        \mathtt{assume\ }S[{\fun{g}}][
        \mathtt{\overline{u}^g} \mapsto \overline{p},
        \mathtt{\overline{r}^g} \mapsto \overline{\mathtt{x}}]    
        &
        \text{ if } \cmd{cmd}^\fun{f} (\ell, \ell') =
        \overline{\mathtt{x}} := \fun{g} (\overline{p})\\
		\mathtt{\overline{r}}^\fun{f} := \overline{q}
        &
        \text{ if } \cmd{cmd}^\fun{f} (\ell, \ell') = \mathtt{return\ }
        \overline{q}\\
        \cmd{cmd}^\fun{f} (\ell, \ell')
        &
		\text{ otherwise}
      \end{array}
    \right.
  \end{array}
\end{equation*}

\begin{figure}[t]
  \centering
  \begin{tikzpicture}[->,>=stealth',shorten >=1pt,auto,node
      distance=2cm,thick,node/.style={circle,draw}]

      \node[node] (00) at (0, 0)  {$\ell$};
      \node[node] (01) at (0, -2) {$\ell'$};

      \node (arrow_s0) at ( .3, -1) {};
      \node (arrow_e0) at (1.3, -1) {};

      \node[node] (10) at (1.6, 0)  {$\ell$};
      \node[node] (11) at (1.6, -2) {$\ell'$};

      \node[node] (20) at (7.4, 0)  {$\ell$};
      \node[node] (21) at (7.4, -2) {$\ell'$};

      \node (arrow_s1) at (7.8, -1) {};
      \node (arrow_e1) at (8.7, -1) {};

      \node[node] (30) at (9, 0)  {$\ell$};
      \node[node] (31) at (9, -2) {$\ell'$};

      
      \path
        (00) edge [left]
             node {$\overline{\mathtt{x}} := \fun{g}(\overline{p})$} (01)

        (arrow_s0) edge [dotted]
                  node {} (arrow_e0)

        (10) edge
             node {$
               \begin{array}{l}
                 \overline{\mathtt{x}} := \overline{\mathtt{nondet}};\\
                 \mathtt{assume\ }S[\fun{g}]
                 [\mathtt{\overline{u}^g} \mapsto \overline{p},
                  \mathtt{\overline{r}^g} \mapsto \overline{\mathtt{x}}]
               \end{array}
             $} (11)

        (20) edge [left]
             node {$\mathtt{return\ } \overline{q}$} (21) 

        (arrow_s1) edge [dotted]
                  node {} (arrow_e1)

        (30) edge 
             node {$\mathtt{\overline{r}} := \overline{q}$} (31) 
             ;
    \end{tikzpicture}

  \caption{Instantiating a Summary}
  \label{figure:instantiating-summary}
\end{figure}

The control flow graph $\hat{G}^{\fun{f}}_{k,S}$ replaces every
function call in $G_k^\fun{f}$ by instantiating a function
summary (Figure~\ref{figure:instantiating-summary}).
Using the Hoare Logic proof rule for recursive functions~\cite{Oheimb99}, we have the
following proposition:
\begin{proposition}
  \label{proposition:check_summary}
  Let $G^\fun{f}_k = \langle V, E, \cmd{cmd}^\fun{f}, \overline{\mathtt{u}}^\fun{f}, \overline{\mathtt{r}}^\fun{f},s,e \rangle$ be the control flow graph for the function
  $\fun{f}$ and $S[\bullet]$ be an array of logic formulae over the formal
  parameters and return variables of each function. If $\assert{\TT}\
  \hat{G}^\fun{g}_{k,S}\ \assert{S[\fun{g}]}$ for every
  function $\fun{g}$ in $\prog{P}$, then $\assert{\TT}\ \mathtt{\overline{r}}^\fun{f} :=
  \fun{f} (\overline{\mathtt{u}}^\fun{f})\ \assert{S[\fun{f}]}$.
\end{proposition}

It is easy to check $\assert{\TT}\ \hat{G}^\fun{g}_{k,S}\
\assert{S[\fun{g}]}$ by program analysis. Let $G_k^{\mathtt{f}}$ be
the control flow graph for the function $\fun{f}$ and
$\hat{G}^\fun{g}_{k,S} = \langle V, E, \hat{\cmd{cmd}}^\fun{f}, \overline{\mathtt{u}}^\fun{f}, \overline{\mathtt{r}}^\fun{f},s,e \rangle$ as
above. Consider another control flow graph $\tilde{G}^\fun{f}_{k,S} =
\langle \tilde{V}, \tilde{E}, \tilde{\cmd{cmd}}^\fun{f} , \overline{\mathtt{u}}^\fun{f}, \overline{\mathtt{r}}^\fun{f},s,e \rangle$ where
\begin{equation*}
  \begin{array}{rcl}
    \tilde{V} & = & V \cup \{ \tilde{e} \}\\
    \tilde{E} & = & E \cup \{ (e, \tilde{e}) \}\\
    \tilde{\cmd{cmd}}^\fun{f} (\ell, \ell') & = &
    \left\{
      \begin{array}{ll}
        \hat{\cmd{cmd}}^\fun{f} (\ell, \ell') &
        \text{ if } (\ell, \ell') \in E\\
        \mathtt{assert\ } S[\fun{f}] &
        \text{ if } (\ell, \ell') = (e, \tilde{e})
      \end{array}
    \right.
  \end{array}
\end{equation*}

\begin{corollary}
  Let $G^\fun{f}_k = \langle V, E, \cmd{cmd}^\fun{f}, \overline{\mathtt{u}}^\fun{f}, \overline{\mathtt{r}}^\fun{f},s,e \rangle$ be the control flow graph for the function
  $\fun{f}$ and $S[\bullet]$ be an array of logic formulae over the formal
  parameters and return variables of each function. If $\method{BasicChecker}
  (\tilde{G}^\fun{g}_{k,S})$ returns $\mathit{Pass}$ for every function
  $\fun{g}$ in \prog{P}, then $\assert{\TT}\ \mathtt{\overline{r}}^\fun{f} :=
  \fun{f} (\overline{\mathtt{u}}^\fun{f})\ \assert{S[\fun{f}]}$.
  \label{corollary:check-summary}
\end{corollary}

\begin{algorithm}[t]
\begin{doublespace}
  \KwIn{$\prog{P}_k$ : an unwound program; $S[\bullet]$ : an array of function summaries}
  \KwOut{$\TT$ if all function summaries are valid; $\FF$ otherwise}
  \ForEach{function $G_k^\fun{g} \in \prog{P}_k$}
  {
    \If{$\method{BasicChecker} (\tilde{G}_{k,S}^{\fun{g}}) \neq
      \mathit{Pass}$}
    {
      \Return $\FF$
    }
  }
  \Return $\TT$\;
\end{doublespace}
  \caption{$\textmd{CheckSummary} (\prog{P}_k, S)$}
  \label{algorithm:check-summary}
\end{algorithm}
