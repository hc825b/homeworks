
\chapter{Overview}\label{ch:overview}

Let \textsc{BasicAnalyzer} denote a recursion-free inductive program analyzer,
and let a program $\Prog{P}=\{G^{\fun{main}}\}\cup\{G^{\fun{f}} : \fun{f}\mbox{ is a function}\}$
consist of the CFGs of the $\fun{main}$ function and functions that may be
invoked (transitively) from $\fun{main}$.
Since non-recursive functions can be replaced by their control flow graphs
after proper variable renaming,
we assume that \Prog{P} only contains the $\mathtt{main}$ and recursive functions.
If \Prog{P} does not contain recursive functions, 
\textsc{BasicAnalyzer} is able to check \Prog{P} by computing inductive invariants.


When \Prog{P} contains recursive functions, we transform $G^{\fun{main}}$ into a
recursion-free program $\underline{G}^{\fun{main}}$. The program $\underline{G}^{\fun{main}}$
under-approximates the computation of $G^{\fun{main}}$. That is, every computation
of $\underline{G}^{\fun{main}}$ is also a computation of $G^{\fun{main}}$. If
\textsc{BasicAnalyzer} finds an error in $\underline{G}^{\fun{main}}$, our
algorithm terminates and reports it. Otherwise,
\textsc{BasicAnalyzer} has computed an inductive invariant for the
recursion-free under-approximation $\underline{G}^{\fun{main}}$. Our algorithm
computes function summaries of functions in \Prog{P} from the inductive invariant of
$\underline{G}^{\fun{main}}$. It then checks if every function summary
over-approximates the computation of the corresponding function. If
so, the algorithm terminates and reports that all assertions in \Prog{P}
are satisfied. If a function summary does not over-approximate the
computation, our algorithm unwinds the recursive function and
reiterates~(Algorithm~\ref{algorithm:overview}).

\begin{algorithm}[htb]
\begin{doublespace}
  \KwIn{A program $\Prog{P}=\{G^{\fun{main}}\}\cup\{G^{\fun{f}} : \fun{f}\mbox{ is a function}\}$}

  $k \leftarrow 0$\;
  $\Prog{P}_0 \leftarrow \Prog{P}$\;
  %\lForEach{function $\fun{f}$ such that  $G_\fun{f} \in \Prog{P}$}
  %{
  %  $S[\mathtt{f}] \leftarrow \mathtt{true}$\;
  %}
  \Repeat{$\textmd{complete?}$}
  {
    $k \leftarrow k + 1$\;
    $\Prog{P}_{k} \leftarrow $ unwind every CFG in $\Prog{P}_{k-1}$\;
    \Switch{\textsc{BasicAnalyzer} ($\underline{G}^{\fun{main}}_k$)}
    {
      \Case{$\mathit{Pass(\Pi (\underline{G}^{\fun{main}}_k, \TT))}$:}
      {    
        S := ComputeSummary($\Prog{P}_k,$ $\Pi (\underline{G}^{\fun{main}}_k, \TT)$)
      }
      \lCase{$\mathit{Error}$:}
      {
        \Return $\mathit{Error}$
      }
    }
    $\textmd{complete?} \leftarrow \textmd{CheckSummary} (\Prog{P}_k, S)$\;
  }
  \Return $\mathit{Pass} (\Pi (\underline{G}^{\fun{main}}_k, \TT))$, $S$\;
\end{doublespace}
  \caption{Overview}
  \label{algorithm:overview}
\end{algorithm} 

To see how to under-approximate computation, consider a control flow
graph $G^{\fun{main}}_k$. The
under-approximation $\underline{G}^{\fun{main}}_k$ is 
obtained by substituting the command $\mathtt{assume\ false}$ for
every command with recursive function calls
(Figure~\ref{figure:under-mccarthy91}). The substitution
effectively blocks all recursive invocations. Any computation of
$\underline{G}^{\fun{main}}_k$ hence is also a computation of $G^{\fun{main}}_k$. Note that
$\underline{G}^{\fun{main}}_k$ is recursion-free. \textsc{BasicAnalyzer} is able to
check the under-approximation~$\underline{G}^{\fun{main}}_k$.
\begin{figure}[htb]
  \centering
    \scalebox{1}{
    \begin{tikzpicture}[->,>=stealth',shorten >=1pt,auto,node
      distance=2cm,thick,node/.style={circle,draw}]
      \node[node] (0) at (-4, 0) {$s$}; %[label=above:$\mathtt{main()}$]
      \node[node] (1) at (-4, -1) {$1$};
      \node[node] (2) at (-4, -2) {$2$};
      \node[node] (3) at (-4, -3) {$3$};
      \node[node] (4) at (-4, -4) {$e$};
      \node[node] (00) at ( 0,  0) {\smallnode{$s_1^{\fun{mc91}}$}};
      \node[node] (01) at (-1, -2) {$1_1$};
      \node[node] (02) at ( 1, -0.8) {$2_1$};
      \node[node] (03) at ( 1, -2) {$3_1$};
      \node[node] (04) at ( 1, -3.2) {$4_1$};
      \node[node] (05) at ( 0, -4) {\smallnode{$e_1^{\fun{mc91}}$}};

      \path
        (0) edge 
            node [left] {$\mathtt{assume\ n >= 0}$} (1)
        (1) edge [bend left=15]
            node [above=2] {$\mathtt{m_1 := n}$} (00)
        (2) edge
            node [left] {$
              \begin{array}{l}
                \mathtt{assert\ {[}r = 91\ or}\\
                \mathtt{\ \ \ \ \ \ \ \ \ \ \ (n > 101\ and\ \ }\\
                \mathtt{\ \ \ \ \ \ \ \ \ \ \ \ r = n - 10){]}}
              \end{array}$} (3) 
        (3) edge 
            node [left] {$\mathtt{return\ 0}$} (4)

        (00) edge [bend right=30]
            node [left] {$\mathtt{assume\ m_1 > 100}$} (01)
            edge [bend left=30]
            node [right] {$\mathtt{assume\ not(m_1 > 100)}$} (02)
        (01) edge [bend right=30]
            node [left] {$
              \begin{array}{c}
                \mathtt{r^{mc91} :=}\\
                \mathtt{m_1 - 10}
              \end{array}$} (05)
        (02) edge 
            node [right] {$\mathtt{assume\ false}$} (03)
        (03) edge 
            node [right] {$\mathtt{assume\ false}$} (04)
        (04) edge [bend left=30]
            node [right] {$\mathtt{r^{mc91}_1 := t_1}$} (05)

        (05) edge [bend left=30]
             node [below=8] {$\mathtt{r := r^{mc91}_1}$} (2);

    \end{tikzpicture}
	}  
  \caption{Under-approximation of McCarthy 91}
  \label{figure:under-mccarthy91}
\end{figure}

When $\textsc{BasicAnalyzer}$ does not find any error in the
under-approximation $\underline{G}^{\fun{main}}_k$, it computes an inductive
invariant $\Pi (\underline{G}^{\fun{main}}_k, \TT)$. Our algorithm then computes
summaries of functions in \Prog{P}. For each function $\fun{f}$ with
formal parameters $\overline{\mathtt{u}}^{\fun{f}}$ and return
variables $\overline{\mathtt{r}}^{\fun{f}}$, a \emph{function summary} for
$\fun{f}$ is a 
first-order conjunctive formula which specifies the relation between
its formal parameters and return variables. The algorithm
ComputeSummary($\Prog{P}_k$, $\Pi (\underline{G}^{\fun{main}}_k, \TT)$) computes summaries $S$ by inspecting
the inductive invariant $\Pi (\underline{G}^{\fun{main}}_k, \TT)$~(Section~\ref{sec:computing-summary}). 

After function summaries are computed, Algorithm~\ref{algorithm:overview} 
verifies whether function summaries correctly specify computations of
functions by invoking CheckSummary$(\Prog{P}_k, S)$. 
The algorithm CheckSummary$(\Prog{P}_k, S)$ checks this by constructing a recursion-free control flow 
graph $\tilde{G}^{\mathtt{f}}$ with additional assertions for each
function $\mathtt{f}$ and verifying $\tilde{G}^{\mathtt{f}}$ with
\textsc{BasicAnalyzer}. The control flow graph
$\tilde{G}^{\mathtt{f}}$ is obtained by substituting function
summaries for function calls.
\hide{
 In Algorithm~\ref{algorithm:overview},
$S[\mathtt{g}]$ contains the function summary for the function
$\mathtt{g}$.  
}
It is transformed from $G^{\mathtt{f}}$ by the
following three steps:
\begin{enumerate}
\item Replace every function call by instantiating the summary for the
  callee;
\item Replace every return command by assignments to return variables;
\item Add an assertion to validate the summary at the end.
\end{enumerate}
\hide{
\begin{enumerate}
\item Replace every function call $\overline{\mathtt{x}} := \mathtt{g}
  (\overline{p})$ in $G_{\mathtt{f}}$ by  
  \begin{equation*}
    \begin{array}{l}
      \overline{\mathtt{x}} := \overline{\mathtt{nondet}};\\
      \mathtt{assume}\ 
      S[{\mathtt{g}}]\{\overline{\mathtt{u}} \mapsto \overline{p}, 
      \overline{\mathtt{r}}^{\fun{g}} \mapsto \overline{\mathtt{x}}\}
    \end{array}
  \end{equation*}
  where $\overline{\mathtt{u}}^{\fun{g}}$
  are the formal parameters of the function $\mathtt{g}$;
\item Replace every $\mathtt{return\ } \overline{q}$ command by
  \begin{equation*}
    \overline{\mathtt{r}}^{\fun{f}} := \overline{q}
  \end{equation*}
\item Add the command $\mathtt{assert\ }S[{\mathtt{f}}]$ at the end of
  $G^{\mathtt{f}}$. 
\end{enumerate}
}
Figure~\ref{figure:check-summary-mccarthy91} shows the control flow
graph $\tilde{G}^{\mathtt{mc91}}$ with the function summary
$S[{\mathtt{mc91}}] = \mathtt{not (m \geq 0)}$. Observe that
$\tilde{G}^{\mathtt{mc91}}$ is
recursion-free. \textsc{BasicAnalyzer} is able to check
$\tilde{G}^{\mathtt{mc91}}$ and invalidates this function summary. 

\begin{figure}
  \centering
    \begin{tikzpicture}[->,>=stealth',shorten >=1pt,auto,node
      distance=2cm,thick,node/.style={circle,draw}]
      \node[node] (s) at ( 0,  0) {$s$}; %[label=above:$\mathtt{mc91(n)}$]
      \node[node] (1) at (-1, -2) {$1$};
      \node[node] (2) at ( 1, -0.8) {$2$};
      \node[node] (3) at ( 1, -2) {$3$};
      \node[node] (4) at ( 1, -3.2) {$4$};
      \node[node] (5) at ( 0, -4) {$5$};
      \node[node] (e) at ( 0, -5) {$e$};

      \path
        (s) edge [bend right=30]
            node [left] {$\mathtt{assume\ m > 100}$} (1)
            edge [bend left=30]
            node [right] {$\mathtt{assume\ not(m > 100)}$} (2)
        (1) edge [bend right=30]
            node [left] {$\mathtt{r^{mc91} := m - 10}$} (5)
        (2) edge 
            node [right] {$
              \begin{array}{l}
              \mathtt{s := nondet;}\\
              \mathtt{assume\ not(m + 11 \geq 0)}
              \end{array}
              $} (3)
        (3) edge 
            node [right] {$
              \begin{array}{l}
              \mathtt{t := nondet;}\\
              \mathtt{assume\ not(s \geq 0)}
              \end{array}$} (4)
        (4) edge [bend left=30]
            node [right] {$\mathtt{r^{mc91} := t}$} (5)
        (5) edge 
            node [right] {$\mathtt{assert\ not(m \geq 0)}$} (e)
        ;
    \end{tikzpicture}
  \caption{Check Summary in McCarthy 91}
  \label{figure:check-summary-mccarthy91}
\end{figure}

In order to refine function summaries, our algorithm unwinds recursive
functions as usual. More precisely, consider a recursive function
$\mathtt{f}$ with formal parameters $\overline{\mathtt{u}}^{\fun{f}}$
and return variables $\overline{\mathtt{r}}^{\fun{f}}$. Let 
$G^{\mathtt{f}}$ be  
the control flow graph of $\mathtt{f}$ and $G^{\fun{main}}_k$ be a
control flow graph 
that invokes $\mathtt{f}$. To unwind $\mathtt{f}$ in $G^{\fun{main}}_k$, 
we first construct a control flow graph $H^{\mathtt{f}}$ by 
replacing every $\mathtt{return}\ \overline{q}$ command in
$\mathtt{f}$ with the assignment $\mathtt{\overline{r}}^{\fun{f}} :=
\overline{q}$. For each edge $(\ell,  
\ell')$ labeled with the command $\overline{\mathtt{x}} :=
\mathtt{f}(\overline{p})$ in $G^{\fun{main}}_k$, we remove 
the edge $(\ell, \ell')$, make a fresh copy $K^{\mathtt{f}}$ of $H^{\mathtt{f}}$ by
renaming all nodes and variables, and then add two edges: add an edge
from $\ell$ to the entry node of $K^{\mathtt{f}}$ that assigns
$\overline{p}$ to fresh copies of formal parameters in 
$K^{\mathtt{f}}$ and another edge from the exit node to $\ell'$ that
assigns fresh copies of return variables to
$\overline{\mathtt{x}}$. The control flow graph $G^{\fun{main}}_{k+1}$ 
is obtained by unwinding every function call in $G^{\fun{main}}_k$. 
Figure~\ref{figure:unwind-mccarthy91} shows the control flow graph
obtained by unwinding $\mathtt{main}$ twice. Note that the
unwinding graph $G^{\fun{main}}_{k+1}$ still has recursive function calls. Its
under-approximation $\underline{G}^{\fun{main}}_{k+1}$ is more
accurate than $\underline{G}^{\fun{main}}_k$. 

\todo[inline]{Adjust figures}
\begin{figure}
  \centering
  %\resizebox{1\textwidth}{!}
  {
    \begin{tikzpicture}[
      ->,>=stealth',shorten >=1pt,thick,
      auto, node distance=2cm,
      node/.style={circle,draw},
      execute at begin node={\begin{varwidth}{0.4cm}},
      execute at end node={\end{varwidth}},
    ]
      \node[node] (0s) at (-4, 0) {$s$}; %[label=above:$\mathtt{main()}$]
      \node[node] (01) at (-4, -1) {$1$};
      \node[node] (02) at (-4, -2) {$2$};
      \node[node] (03) at (-4, -3) {$3$};
      \node[node] (0e) at (-4, -4) {$e$};

      \node[node] (s) at (-1,  0) {\smallnode{$s_1^{\fun{mc91}}$}};
      \node[node] (1) at (-2, -2) {$1_1$};
      \node[node] (2) at ( 1, -0.8) {$2_1$};
      \node[node] (3) at ( 1, -2) {$3_1$};
      \node[node] (4) at ( 1, -3.2) {$4_1$};
      \node[node] (e) at (-1, -4) {\smallnode{$e_1^{\fun{mc91}}$}};

      \node[node] (s') at ( 5,  2.6) {$s_2$};
      \node[node] (1') at ( 4,  0.6) {$1_2$};
      \node[node] (2') at ( 6,  1.8) {$2_2$};
      \node[node] (3') at ( 6,  0.6) {$3_2$};
      \node[node] (4') at ( 6, -0.6) {$4_2$};
      \node[node] (e') at ( 5, -1.4) {$e_2$};

      \node[node] (s'') at ( 5, -2.6) {$s_3$};
      \node[node] (1'') at ( 4, -4.6) {$1_3$};
      \node[node] (2'') at ( 6, -3.4) {$2_3$};
      \node[node] (3'') at ( 6, -4.6) {$3_3$};
      \node[node] (4'') at ( 6, -5.8) {$4_3$};
      \node[node] (e'') at ( 5, -6.6) {$e_3$};

      \path
        (0s) edge 
            node [left] {$\mathtt{assume\ n >= 0}$} (01)
        (01) edge [bend left=15]
            node [above=2] {$\mathtt{m_1 := n}$} (s)
        (02) edge 
            node [left] {$
              \begin{array}{l}
                \mathtt{assert\ {[}r = 91\ or}\\
                \mathtt{\ \ \ \ \ \ \ \ \ \ \ (n > 101\ and\ \ }\\
                \mathtt{\ \ \ \ \ \ \ \ \ \ \ \ r = n - 10){]}}
              \end{array}$} (03) 
        (03) edge 
            node [left] {$\mathtt{return\ 0}$} (0e)


        (s) edge [bend right=30]
            node [right] {$
              \begin{array}{c}
                \mathtt{assume}\\ 
                \mathtt{m_1 > 100}                
              \end{array}$} (1)
            edge [bend left=60]
            node [above] {$
              \begin{array}{l}
                \mathtt{assume}\\
                \mathtt{not(m_1 > 100)\ }
              \end{array}$} (2)
        (1) edge [bend right=30]
            node [above right] {$
              \begin{array}{c}
                \mathtt{r^{mc91}_1 :=}\\
                \mathtt{m_1 - 10}
              \end{array}$} (e)
        (2) edge [bend left=50]
            node [above left] {$\mathtt{m_2 := m_1 + 11}$} (s')
        (3) edge 
            node [below] {$\mathtt{m_3 := s_1}$} (s'')
        (4) edge [bend left=30]
            node [below] {$\mathtt{r^{mc91}_1 := t_1}$} (e)
        (e) edge [bend left=15]
            node [below=9] {$\mathtt{r := r^{mc91}_1}$} (02)

        (s') edge [bend right=30]
             node [left] {$
               \begin{array}{c}
                 \mathtt{assume}\\
                 \mathtt{m_2 > 100}
               \end{array}$} (1')
             edge [bend left=30]
             node [right] {$\mathtt{assume\ not(m_2 > 100)}$} (2')
        (1') edge [bend right=30]
             node [left] {$\mathtt{r^{mc91}_2 := m_2 - 10}$} (e')
        (2') edge 
             node [right] {$\mathtt{s_2 := mc91(m_2+11)}$} (3')
        (3') edge 
             node [right] {$\mathtt{t_2 := mc91(s_2)}$} (4')
        (4') edge [bend left=30]
            node [right] {$\mathtt{r^{mc91}_2 := t_2}$} (e')
        (e') edge 
             node [above] {$\mathtt{s_1 := r^{mc91}_2}$} (3)

        (s'') edge [bend right=30]
              node [left] {$\mathtt{assume\ m_3 > 100}$} (1'')
              edge [bend left=30]
              node [right] {$\mathtt{assume\ not(m_3 > 100)}$} (2'')
        (1'') edge [bend right=30]
              node [left] {$
                \begin{array}{c}
                \mathtt{r^{mc91}_3 :=}\\
                \mathtt{m_3 - 10}  
                \end{array}
                $} (e'')
        (2'') edge 
              node [right] {$\mathtt{s_3 := mc91(m_3+11)}$} (3'')
        (3'') edge 
              node [right] {$\mathtt{t_3 := mc91(s_3)}$} (4'')
        (4'') edge [bend left=50]
              node [right] {$\mathtt{r^{mc91}_3 := t_3}$} (e'')
        (e'') edge [bend left=50]
              node [below left] {$\mathtt{t_1 := r^{mc91}_3}$} (4)
        ;
    \end{tikzpicture}
    }
  \caption{Unwinding McCarthy 91}
  \label{figure:unwind-mccarthy91}
\end{figure}

