
\section{Unwinding}\label{sec:unwinding}
As introduced in Chapter~\ref{ch:overview}, our algorithm produces more and more accurate under-approximations through unwinding the program.
The function $\textsc{unwind}(G^\fun{f})$ not only returns a CFG $K^\fun{f}$ obtained by replacing all function call edges in $G^\fun{f}$ with the CFG of the called function, but also keeps the program behavior of $K^\fun{f}$ be the same as the behavior of $G^\fun{f}$ (Figure~\ref{figure:unwinding}). The formal definition is given below.

\begin{figure}[t]
  \centering
  \begin{tikzpicture}[->,>=stealth',shorten >=1pt,auto,node
      distance=2cm,thick,node/.style={circle,draw}]

      \node[node] (0) at (-4, 0)  {$\ell$}; %[label=above:$\mathtt{main()}$]
      \node[node] (1) at (-4, -2) {$\ell'$};

      \draw [fill=gray!10] (4, -1) ellipse (1.8 and 1.5);
      \node (text) at (4, -1) {$\textsc{rename}(G_\fun{g},i)$};
      \node[node] (00) at (0, 0)  {$\ell$};
      \node[node] (01) at (0, -2) {$\ell'$};
      \node[node] (10) at (4, 0)  {\smallnode{$s_i$}};
      \node[node] (11) at (4, -2) {\smallnode{$e_i$}};
      \node (arrow_s) at (-2.5, -1) {};
      \node (arrow_e) at (-0.5, -1) {};

      \path
        (arrow_s) edge [dotted]
                  node {} (arrow_e)
        (0) edge 
            node {$\overline{\mathtt{x}} := \mathtt{g} (\overline{p})$} (1)

        (00) edge 
             node {$\overline{\mathtt{u}}_i^\fun{g} := \overline{p}$} (10)

        (11) edge 
             node {$\overline{\mathtt{x}} := \mathtt{\overline{r}_i^\fun{g}}$} (01) ;
    \end{tikzpicture}

  \caption{Unwinding Function Calls}
  \label{figure:unwinding}
\end{figure}

We first define the function $\textsc{rename}(G^\fun{f},i)$. It returns a CFG obtained by first replacing every return command $\mathtt{return}\ \overline{q}$ with assignments to return variables $\overline{\texttt{r}}^\fun{f} := \overline{q}$ and then renaming all variables and locations in $G^\fun{f}$ with the given index value $i$. The function $\textsc{addindex}(c, i)$ is used to rename all variable occurrences with index $i$ in a command $c$. Formally,
\[
\begin{array}{l}
\textsc{rename}(\langle V, E,\textmd{cmd}^\fun{f}, \overline{\texttt{u}}^\fun{f}, \overline{\texttt{r}}^\fun{f},s,e \rangle,i) = \langle V_i, E_i,\textmd{cmd}_i^\fun{f},
\overline{\texttt{u}}_i^\fun{f}, \overline{\texttt{r}}_i^\fun{f},s_i,e_i \rangle \text{ such that}\\
V_i = \{\ell_i:\ell \in V\} \\
E_i = \{(\ell_i,\ell'_i): (\ell,\ell') \in E\} \\
\textmd{cmd}_i^\fun{f}(\ell_i,\ell'_i) = 
  \left\{
  \begin{array}{ll}
   \textsc{addindex}(\overline{\texttt{r}}^\fun{f}:=\overline{q}, i) &\text{if } \textmd{cmd}^\fun{f}(\ell,\ell') = \mathtt{return}\  \overline{q} \\
   \textsc{addindex}(\textmd{cmd}^\fun{f}(\ell,\ell'), i) &\text{otherwise}
  \end{array}
  \right.
\end{array}
\]
Detail of $\textsc{addindex}(c, i)$ is not described here for simplicity.

Then, given a CFG $G^\fun{f}=\langle
V, E,\textmd{cmd}^\fun{f}, \overline{\texttt{u}}^\fun{f}, \overline{\texttt{r}}^\fun{f},s,e \rangle$,
we use $\hat{E} =\{(\ell, \ell')\in E: \textmd{cmd}^\fun{f} (\ell,\ell')= (\overline{\texttt{x}}:=\mathtt{g}(\overline{p}))\}$ to denote the set of function call edges in $E$ and define a function $\textsc{idx}(e)$ that maps a call edge $e$ to a unique index value.
\hide{ % TODO
The predicate below then can be used to c a renamed CFG $\textsc{rename}(G^\fun{g},i)$ according to a function call edge $(\ell, \ell')$ in $G^\fun{f}$.
\[
\begin{split}
{UnwindCall}(\ell,\ell') \equiv (\ell,\ell')\in\hat{E} \wedge \textmd{cmd}^\fun{f}(\ell,\ell')=(\overline{\texttt{x}}:=\mathtt{g}(\overline{p}))\wedge\textsc{idx}(\ell, \ell')=i  \\
  \wedge \textsc{rename}(G^\fun{g},i)=\langle V_i, E_i,\textmd{cmd}^\fun{g}_i, \overline{\texttt{u}}^\fun{g}_i, \overline{\texttt{r}}^\fun{g}_i,s_i,e_i\rangle
\end{split}
\]
}
Finally, we provide the formal definition of $\textsc{unwind}(G^\fun{f})$.
\todo[inline]{The definition of \textmd{cmd}is not complete}
\[
\begin{array}{l}
\textsc{unwind}(G^\fun{f}) = \langle V_u, E_u,\textmd{cmd}_u^\fun{f}, \overline{\texttt{u}}^\fun{f}, \overline{\texttt{r}}^\fun{f},s,e \rangle \text{ such that} \\

\begin{split}
 V_u = V\cup\bigcup\{V_i:(\ell,\ell')\in\hat{E} \wedge \textmd{cmd}^\fun{f}(\ell,\ell')=(\overline{\texttt{x}}:=\mathtt{g}(\overline{p}))\wedge\textsc{idx}(\ell, \ell')=i  \\
  \wedge \textsc{rename}(G^\fun{g},i)=\langle V_i, E_i,\textmd{cmd}^\fun{g}_i, \overline{\texttt{u}}^\fun{g}_i, \overline{\texttt{r}}^\fun{g}_i,s_i,e_i\rangle \}
\end{split}\\

\begin{split}
 E_u = E \setminus \hat{E} \cup \bigcup\{E_i\cup\{(\ell, s_i),(e_i, \ell')\}: (\ell, \ell')\in \hat{E} \wedge \textmd{cmd}^\fun{f}(\ell, \ell')=(\overline{\texttt{x}}:=\mathtt{g}(\overline{p})) \\
  \wedge \textsc{idx}(\ell, \ell')=i \wedge \textsc{rename}(G^\fun{g},i) =
\langle V_i, E_i,\textmd{cmd}^\fun{g}_i, \overline{\texttt{u}}^\fun{g}_i, \overline{\texttt{r}}^\fun{g}_i,s_i,e_i\rangle \} 
\end{split}\\

\textmd{cmd}_u^\fun{f} (\ell, \ell') =
\left\{
  \begin{array}{ll}
  (\overline{\texttt{u}}_i:=\overline{p}) &\text{if } (\ell,\ell')=(\ell,s_i)
  \\
  (\overline{\texttt{x}}:=\overline{\texttt{r}}^\fun{g}_i) &\text{if } (\ell,\ell')=(e_i,\ell') \\
  \textmd{cmd}^\fun{g}_i(\ell, \ell') &\text{if } (\ell, \ell') \in E_i \\
  \textmd{cmd}^\fun{f}(\ell,\ell') &\text{otherwise}
  \end{array}
\right.

\end{array}
\]

%$\textsc{unwind}(G) = \langle
%V', E' \rangle$ such that 
%(1) For all edges $(\ell, \ell')\in E$ with $\textmd{cmd} (\ell, \ell')=$


We first define the function \(\textsc{duplicate}(Prog, \fun{rec},i)\). to produce a copy of the given function $\fun{f}$.
\begin{equation*}
  \textsc{duplicate}(\langle \mathbf{pname},main,\fun{body}\rangle, \fun{rec},i) 
    = \langle \mathbf{pname}\cup\{rec_1, rec_2\dotsc rec_i\},main,\fun{body}\rangle
\end{equation*}



\begin{proposition}
  Let $G^\fun{f}$ be a control flow graph. $P$ and $Q$ are logic formulae with
  free variables over program variables of $G^\fun{f}$. $\assert{P}\ G^\fun{f}\ 
  \assert{Q}$ if and only if $\assert{P}\ \textsc{unwind} (G^\fun{f})\ \assert{Q}$.
\end{proposition}
Essentially, $G^\fun{f}$ and $\textsc{unwind} (G^\fun{f})$ represent the same function $\fun{f}$. The only difference is that the latter has more program variables after unwinding, but this does not affect the states over program variables of $G^\fun{f}$ before and after the function.
