\stepcounter{subsection}

\begin{frame}{Overview}
\tikzstyle{decision} = [diamond, aspect=2, draw, align=center]
\tikzstyle{data} = [draw,trapezium,trapezium left angle=70,trapezium right angle=-70,minimum height=0.6cm]
\tikzstyle{input} = [data, text width=1.7cm]
\tikzstyle{output} = [rounded corners, draw, text width=1.5cm, align=center,minimum height=0.7cm]
\tikzstyle{cross_out} = [red, pos=0.5, auto=false, cross out, -, draw, thick]

\begin{figure}
\begin{tikzpicture}[node distance=1cm and 1.2cm, auto,>=latex', thick]
  % We need to set at bounding box first. Otherwise the diagram
  % will change position for each frame.
  \useasboundingbox (-5.7,-1) rectangle (4.3,4);

  % Nodes
  \node[decision] (analyzer) {Analyzer};
  \node[input, left=of analyzer] (rf_prog) {Intra-proc. Program};

  \node[data, above=.5cm of analyzer,text width=18mm] (proof) {Summary Candidates};

  \node[decision,above=.5cm of proof] (check) {Check};

  \node[input] (rc_prog) at (rf_prog |- check) {Recursive Program};
  \node[output, right=of analyzer] (unsafe) {ERROR};
  \node[output] (safe) at (check -| unsafe) {PASS};

  % Edges
  \path[->]
    (rf_prog) edge (analyzer)
    (analyzer) edge node[anchor=north,sloped]{Error} (unsafe);

  \path[->]
    (rc_prog) edge 
      node[anchor=east]{Under-approx.}
    (rf_prog);

  \path[->]
    (analyzer) edge node[red, anchor=west]{Pass, Compute Summaries} (proof);
  \path[->]
    (proof) edge (check)
    (check) edge node[anchor=south,sloped]{Error, Refine} (rf_prog)
    (check) edge node[anchor=south,sloped]{Pass} (safe);

\end{tikzpicture}
\end{figure}
\end{frame}


\begin{frame}{Over-approximation of Recursive Call}
  Over-approximate recursive function call
  \begin{itemize}
    \item Use $\mathtt{assume}\ S[\mathtt{g}]$
          if $S[\mathtt{g}]$ summarizes the behavior of $\mathtt{g}$
    \item<2-> Derive possible summaries from output of \textsc{BasicAnalyzer}
  \end{itemize}

  \begin{figure}
  \centering
  \begin{tikzpicture}[->,>=stealth',shorten >=1pt,auto,node
    distance=2cm,thick,node/.style={circle,draw}]

    \node[node] (0) at (-4, 0)  {$\ell$};
    \node[node] (1) at (-4, -2) {$\ell'$};

    \node[node] (00) at (0, 0)  {$\ell$};
    \node[node] (01) at (0, -2) {$\ell'$};
    \node (arrow_s) at (-2.25, -1) {};
    \node (arrow_e) at (-0.25, -1) {};

    \path
    (arrow_s) edge [dotted] node {} (arrow_e)
    (0) edge
      node {$\overline{\mathtt{x}} := \mathtt{g} (\overline{p})$}
    (1)

    (00) edge node {$\mathtt{assume}\ S[\mathtt{g}]$} (01);
  \end{tikzpicture}
  \end{figure}
  \note {Block some of the execution paths to under-approximate computations}

\end{frame}

\begin{frame}{Desired Program Analyzer}

\begin{overlayarea}{\textwidth}{.12\textheight}
  \textsc{BasicAnalyzer}
  \begin{itemize}
  \item Provide \emph{\alert{inductive invariants}} when program passed analysis
  \end{itemize}
\end{overlayarea}
  \begin{figure}
    \tikzstyle{every node}=[font=\small]
    \begin{tikzpicture}[->,>=stealth',shorten >=1pt,auto,node
      distance=2cm,thick,node/.style={circle,draw}]
      \useasboundingbox (-5,-4) rectangle (5,0.5);

      \node[node,label=above:$\mathtt{main()}$] (0) at (0, 0) 
        [label=right:\color{Blue}\{$\mathtt{true}$\}] {$s$};
      \node[node] (1) at (0, -1)
        [label=right:\color{Blue}\{$\mathtt{n \geq 0}$\}] {$1$};
      \node[node] (2) at (0, -2)
        [label=right:\color{Blue}\{$\mathtt{false}$\}] {$2$};
      \node[node] (3) at (0, -3)
        [label=right:\color{Blue}\{$\mathtt{false}$\}] {$3$};
      \node[node] (4) at (0, -4)
        [label=right:\color{Blue}\{$\mathtt{false}$\}] {$e$};

      \path
        (0) edge 
            node [left] {$
              \begin{array}{r}
                \mathtt{n = nondet}; \\
                \mathtt{assume\ n >= 0}
              \end{array}$} (1)
        (1) edge 
            node [left] {\Ccancel[red]{$\color{OliveGreen}{\mathtt{r := mc91(n)}}$}} (2)
        (1) edge
            node [right]{${\mathtt{assume\ false}}$} (2)
                
              
        (2) edge 
            node [left] {$
              \begin{array}{l}
                \mathtt{assert\ {[}r = 91\ or}\\
                \mathtt{\ \ \ \ \ \ \ \ \ \ \ (n > 101\ and\ \ }\\
                \mathtt{\ \ \ \ \ \ \ \ \ \ \ \ r = n - 10){]}}
              \end{array}$} (3) 
        (3) edge 
            node [left] {$\mathtt{return\ 0}$} (4);
    \end{tikzpicture}
  \end{figure}
\end{frame}

\begin{frame}{Compute Function Summaries}

\begin{overlayarea}{\textwidth}{.12\textheight}
  \only<1>{Inductive Invariants for under-approximation
  }\only<2>{Invariants enclose under-approximated function call
  }\only<3>{Extract pair of invariants for function call
  }\only<4>{Replace actual parameters to formal parameters
  }\only<5->{
    Compute with $\mathtt{(\{true\}, \{r^{mc91} \geq 91 \ and\ r^{mc91}=m-10\})}$
  }
  
  \begin{itemize}
    \item {
      \only<1>{Over-approximation of computation of under-approximation
      }\only<2>{Possibly over-approximate original function call
      }\only<3>{$\mathtt{(\{true\}, \{r \geq 91 \ and\ r=n-10\})}$
      }\only<4>{$\mathtt{(\{true\}, \{{\color{BrickRed}r^{mc91}} \geq 91 \ and\ {\color{BrickRed}r^{mc91}}={\color{BrickRed}m}-10\})}$
      }\only<5->{Summary: $\mathtt{true \implies (r^{mc91} \geq 91 \ and\ r^{mc91}=m-10})$}
    }
  \end{itemize}
\end{overlayarea}

\begin{overlayarea}{\textwidth}{.6\textheight}
\begin{figure}
  \centering
  \resizebox{\linewidth}{!}
  {  
    \tikzstyle{every node}=[font=\small]
    \tikzstyle{inv_DC}=[label=right:\color{Blue}\{$\mathtt{...}$\}]
    \begin{tikzpicture}[->,>=stealth',shorten >=1pt,auto,node
      distance=2cm,thick,node/.style={circle,draw}]
      \useasboundingbox (-7.5,-4.5) rectangle (4,0.8);
      
      \node[node,label=above:$\mathtt{main()}$] (0) at (-4, 0)
        [label=left:\color{Blue}\{$\mathtt{true}$\}] {$s$};
      \node[node] (1) at (-4, -1) 
        [label=left:\alt<1>{\color{Blue}}{\color{Red}}\{$\mathtt{true}$\}]
        {$1$};
      \node[node] (2) at (-4, -2) 
        [label=right:\alt<1>{\color{Blue}}{\color{Red}}\scriptsize
          $\begin{Bmatrix}
           \mathtt{r \geq 91 \ and} \\
           \mathtt{r=n-10}
          \end{Bmatrix}$
        ]
        {$2$};
      \node[node, inv_DC] (3) at (-4, -3) {$3$};
      \node[node, inv_DC] (4) at (-4, -4) {$e$};

      \node[node] (00) at ( 0,  0) [label=5:\color{Blue}\{$\mathtt{...}$\}]
        {\smallnode{$s_1^{\fun{mc91}}$}};
      \node[node, inv_DC] (01) at (-1, -2) {$1_1$};
      \node[node, inv_DC] (02) at ( 1, -0.8) {$2_1$};
      \node[node, inv_DC] (03) at ( 1, -2) {$3_1$};
      \node[node, inv_DC] (04) at ( 1, -3.2) {$4_1$};
      \node[node] (05) at ( 0, -4) [label=-5:\color{Blue}\{$\mathtt{...}$\}]
        {\smallnode{$e_1^{\fun{mc91}}$}};

      \path
        (0) edge 
            node [left] {$
              \begin{array}{r}
                \mathtt{n = nondet}; \\
                \mathtt{assume\ n >= 0}
              \end{array}$} (1)
        (2) edge
            node [left] {$
              \begin{array}{l}
                \mathtt{assert\ {[}r = 91\ or}\\
                \mathtt{\ \ \ \ \ \ \ \ \ \ \ (n > 101\ and\ \ }\\
                \mathtt{\ \ \ \ \ \ \ \ \ \ \ \ r = n - 10){]}}
              \end{array}$} (3) 
        (3) edge 
            node [left] {$\mathtt{return\ 0}$} (4);

      \path
        (1) edge[draw=none] node[anchor=center]
          {\color{Gray}{$\mathtt{r := mc91(n)}$}}
        (2);


      \path
        (1) edge [bend left]
            node [above] {$\mathtt{m_1 := n}$} (00)
        (00) edge [bend right]
            node [left] {$\mathtt{assume\ m_1 > 100}$} (01)
            edge [bend left=30]
            node [right] {$\mathtt{assume\ not(m_1 > 100)}$} (02)
        (01) edge [bend right]
            node [left] {$
              \begin{array}{c}
                \mathtt{r^{mc91}_1 :=}\\
                \mathtt{m_1 - 10}
              \end{array}$} (05)
        (02) edge 
            node [right] {$\mathtt{assume\ false}$} (03)
        (03) edge 
            node [right] {$\mathtt{assume\ false}$} (04)
        (04) edge [bend left]
            node [right] {$\mathtt{r^{mc91}_1 := t_1}$} (05)

        (05) edge [bend left]
             node [below=8pt] {$\mathtt{r := r^{mc91}_1}$} (2);

    \uncover<4>{
      \node[node, dashed, left=2.5cm of 0] (decl)
        [label=above:\color{BrickRed}$\mathtt{mc91(m)}$] {$s$};
    }

    \end{tikzpicture}
  }
  \label{figure:inductive-invariants}
\end{figure}
\end{overlayarea}
  \note [figure]{The pair of locations for extract summary is different than described in paper.}
\end{frame}


\begin{frame}{Compute Function Summaries $\sim$}
\begin{itemize}
  \item This method produces summary for one function call
  \item<2-> Function Summary Candidate:\\
            {\color{red}Conjunction} of summaries from all function calls
  \item<2-> Details in \S 5.3.
\end{itemize}

\uncover<3->{
\begin{center}
  \Large How to check derived candidates?
\end{center}
}
\end{frame}
