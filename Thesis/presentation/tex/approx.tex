\stepcounter{subsection}


\begin{frame}{Overview}
\tikzstyle{decision} = [diamond, aspect=2, draw, align=center]
\tikzstyle{data} = [draw,trapezium,trapezium left angle=70,trapezium right angle=-70,minimum height=0.6cm]
\tikzstyle{input} = [data, text width=1.7cm]
\tikzstyle{output} = [rounded corners, draw, text width=1.5cm, align=center,minimum height=0.7cm]
\tikzstyle{cross_out} = [red, pos=0.5, auto=false, cross out, -, draw, thick]

\begin{figure}
\begin{tikzpicture}[node distance=1cm and 1.2cm, auto,>=latex', thick]
  % We need to set at bounding box first. Otherwise the diagram
  % will change position for each frame.
  \useasboundingbox (-5.7,-1) rectangle (4.3,4);

  % Nodes
  \node[decision] (analyzer) {Analyzer};
  \node[input, left=of analyzer] (rf_prog) {Intra-proc. Program};

  \node[data, above=.5cm of analyzer,text width=18mm] (proof) {Summary Candidates};

  \node[decision,above=.5cm of proof] (check) {Check};

  \node[input] (rc_prog) at (rf_prog |- check) {Recursive Program};
  \node[output, right=of analyzer] (unsafe) {ERROR};
  \node[output] (safe) at (check -| unsafe) {PASS};

  % Edges
  \path[->]
    (rf_prog) edge (analyzer)
    (analyzer) edge node[anchor=north,sloped]{Error} (unsafe);

  \path[->]
    (rc_prog) edge 
      node[red, anchor=east]{Under-approx.}
    (rf_prog);

  \path[->]
    (analyzer) edge node[anchor=west]{Pass, Compute Summaries} (proof);
  \path[->]
    (proof) edge (check)
    (check) edge node[red, anchor=south,sloped]{Error, Refine} (rf_prog)
    (check) edge node[anchor=south,sloped]{Pass} (safe);

\end{tikzpicture}
\end{figure}
\end{frame}

\begin{frame}[fragile]{Example C Program}

  \begin{columns}[T]
    \begin{column}[T]{.4\linewidth}
      \begin{minted}{c}
int main() {
  int n, r;
  assume(n >= 0);

  r = mc91(n);
  assert(r == 91 || 
    (n>101 && r==n-10));

  return 0;
}
      \end{minted}
    \end{column}

    \begin{column}[T]{.4\linewidth}
      \begin{minted}{c}
int mc91(int m) {
  int s, t;
  if(m > 100)
    return m - 10;
  /* else */
  s = mc91(m + 11);
  t = mc91(s);

  return t;
}
      \end{minted}
    \end{column}
  \end{columns}

\end{frame}

\begin{frame}{Preliminaries}

\begin{overlayarea}{\textwidth}{.18\textheight}
  \begin{columns}[T]
    \begin{column}[T]{.37\linewidth}
      Program \uncover<2->{Safety}
      \begin{itemize}
        \item \alt<1>{Set of functions}{All assertions hold}
        \uncover<1>{\item One \textbf{main} function}
      \end{itemize}
    \end{column}
    \begin{column}[T]{.47\linewidth}

    \only<1>{Function: Control Flow Graph
    }\only<2>{Assumption on functions
    }\only<3>{Special Commands}
    \begin{itemize}
      \item \only<1>{Node: program location
            }\only<2>{Formal parameters are read-only
            }\only<3->{$\mathtt{nondet}$}
      \only<1,3->{
        \item \only<1>{Edge: program command
            }\only<3->{$\mathtt{assume}$}
      }
    \end{itemize}

    \end{column}
  \end{columns}
\end{overlayarea}

\begin{overlayarea}{\textwidth}{.61\textheight}
\begin{figure}
  \centering
  % TODO try to reduce the space between text and figure
  \resizebox{\textwidth}{!}
  {
    \tikzstyle{every node}=[font=\small]
    \begin{tikzpicture}[->,>=stealth',shorten >=1pt,auto,node
      distance=2cm,thick,node/.style={circle,draw}]

      \node[node,label=above:$\mathtt{main()}$] (0) at (0, 0)  {$s$};
      \node[node] (1) at (0, -1) {$1$};
      \node[node] (2) at (0, -2) {$2$};
      \node[node] (3) at (0, -3) {$3$};
      \node[node] (4) at (0, -4) {$e$};

      \path
        (0) edge 
            node [left] {$\mathtt{assume\ n >= 0}$} (1)
        (1) edge 
            node [left] {$\mathtt{r := mc91(n)}$} (2)
        (2) edge 
            node [left] {$
              \begin{array}{l}
                \mathtt{assert\ {[}r = 91\ or}\\
                \mathtt{\ \ \ \ \ \ \ \ \ \ \ (n > 101\ and\ \ }\\
                \mathtt{\ \ \ \ \ \ \ \ \ \ \ \ r = n - 10){]}}
              \end{array}$} (3) 
        (3) edge 
            node [left] {$\mathtt{return\ 0}$} (4);
    \end{tikzpicture}

    \begin{tikzpicture}[->,>=stealth',shorten >=1pt,auto,node
      distance=2cm,thick,node/.style={circle,draw}]
      \node[node,label=above:$\mathtt{mc91(m)}$] (0) at ( 0,  0) {$s$};
      \node[node] (1) at (-1, -2) {$1$};
      \node[node] (2) at ( 1, -0.8) {$2$};
      \node[node] (3) at ( 1, -2) {$3$};
      \node[node] (4) at ( 1, -3.2) {$4$};
      \node[node] (5) at ( 0, -4) {$e$};

      \path
        (0) edge [bend right=30]
            node [left] {$\mathtt{assume\ m > 100}$} (1)
            edge [bend left=30]
            node [right] {$\mathtt{assume\ not(m > 100)}$} (2)
        (1) edge [bend right=30]
            node [left] {$\mathtt{return\ m - 10}$} (5)
        (2) edge 
            node [right] {$\mathtt{s := mc91(m + 11)}$} (3)
        (3) edge 
            node [right] {$\mathtt{t := mc91(s)}$} (4)
        (4) edge [bend left=30]
            node [right] {$\mathtt{return\ t}$} (5);
    \end{tikzpicture}
  }
  \caption{McCarthy 91 function}
  \label{figure:mccarthy91}
\end{figure}
\end{overlayarea}
\end{frame}

\begin{frame}{\only<1>{Under}\only<2>{Over}-approximation of Recursive Call}
  \only<1>{Under}\only<2>{Over}-approximate recursive function call
  \begin{itemize}
    \item Use $\mathtt{assume\ \only<1>{false}\only<2>{\mathtt{S}}}$ 
          \only<2>{if $\mathtt{S}$ summarizes the behavior of $\mathtt{g}$}
  \end{itemize}

  \begin{figure}
  \centering
  \begin{tikzpicture}[->,>=stealth',shorten >=1pt,auto,node
    distance=2cm,thick,node/.style={circle,draw}]

    \node[node] (0) at (-4, 0)  {$\ell$};
    \node[node] (1) at (-4, -2) {$\ell'$};

    \node[node] (00) at (0, 0)  {$\ell$};
    \node[node] (01) at (0, -2) {$\ell'$};
    \node (arrow_s) at (-2.25, -1) {};
    \node (arrow_e) at (-0.25, -1) {};

    \path
    (arrow_s) edge [dotted] node {} (arrow_e)
    (0) edge 
      node {$\overline{\mathtt{x}} := \mathtt{g} (\overline{p})$}
    (1)

    (00) edge node {$\mathtt{assume\ \only<1>{false}\only<2>{\mathtt{S}}}$} (01);
  \end{tikzpicture}
  \end{figure}
  \note {Block some of the execution paths to under-approximate computations}

\end{frame}

\begin{frame}{Under-approximation of Recursive Program}

  Under-approximate recursive program
  \begin{itemize}
    \uncover<2>{\item Replace all recursive calls with $\mathtt{assume\ false}$}
  \end{itemize}

  \begin{figure}
    \tikzstyle{every node}=[font=\small]
    \begin{tikzpicture}[->,>=stealth',shorten >=1pt,auto,node
      distance=2cm,thick,node/.style={circle,draw}]
      \useasboundingbox (-5,-4) rectangle (5,0.5);

      \node[node,label=above:$\mathtt{main()}$] (0) at (0, 0)  {$s$};
      \node[node] (1) at (0, -1) {$1$};
      \node[node] (2) at (0, -2) {$2$};
      \node[node] (3) at (0, -3) {$3$};
      \node[node] (4) at (0, -4) {$e$};

      \path
        (0) edge 
            node [left] {$\mathtt{assume\ n >= 0}$} (1)
        (1) edge 
            node [left]  {\only<2->{\Ccancel[red]}{\color{OliveGreen}$\mathtt{r := mc91(n)}$}} (2)
        (1) edge 
            node [right] {\only<2->{$\color{red}\mathtt{assume\ false}$}} (2)
        (2) edge 
            node [left] {$
              \begin{array}{l}
                \mathtt{assert\ {[}r = 91\ or}\\
                \mathtt{\ \ \ \ \ \ \ \ \ \ \ (n > 101\ and\ \ }\\
                \mathtt{\ \ \ \ \ \ \ \ \ \ \ \ r = n - 10){]}}
              \end{array}$} (3) 
        (3) edge 
            node [left] {$\mathtt{return\ 0}$} (4);
    \end{tikzpicture}
  \end{figure}
\end{frame}

\begin{frame}[fragile]{Transformed C Program}
  \begin{minted}{c}
int main() {
  int n, r;
  assume(n >= 0);

  /* r = mc91(n); */ assume((bool) 0);
  assert(r == 91 || 
    (n>101 && r==n-10));

  return 0;
}
  \end{minted}
\end{frame}

\begin{frame}{More Accurate Approximation}
  
  How to find more accurate under-approximation?
  \begin{itemize}
    \uncover<2->{\item Unwind \uncover<3>{and Replace}}
  \end{itemize}

\begin{overlayarea}{\textwidth}{.6\textheight}
\begin{figure}
  \centering
  \resizebox{\linewidth}{!}
  {  
    \tikzstyle{every node}=[font=\small]
    \begin{tikzpicture}[->,>=stealth',shorten >=1pt,auto,node
      distance=2cm,thick,node/.style={circle,draw}]
      \useasboundingbox (-7.5,-4.5) rectangle (4,0.8);
      
      \node[node,label=above:$\mathtt{main()}$] (0) at (-4, 0) {$s$};
      \node[node] (1) at (-4, -1) {$1$};
      \node[node] (2) at (-4, -2) {$2$};
      \node[node] (3) at (-4, -3) {$3$};
      \node[node] (4) at (-4, -4) {$e$};
    \uncover<2->{
      \node[node] (00) at ( 0,  0){\smallnode{$s_1^{\fun{mc91}}$}};
      \node[node] (01) at (-1, -2) {$1_1$};
      \node[node] (02) at ( 1, -0.8) {$2_1$};
      \node[node] (03) at ( 1, -2) {$3_1$};
      \node[node] (04) at ( 1, -3.2) {$4_1$};
      \node[node] (05) at ( 0, -4) {\smallnode{$e_1^{\fun{mc91}}$}};
    }

      \path
        (0) edge 
            node [left] {$\mathtt{assume\ n >= 0}$} (1)
        (2) edge
            node [left] {$
              \begin{array}{l}
                \mathtt{assert\ {[}r = 91\ or}\\
                \mathtt{\ \ \ \ \ \ \ \ \ \ \ (n > 101\ and\ \ }\\
                \mathtt{\ \ \ \ \ \ \ \ \ \ \ \ r = n - 10){]}}
              \end{array}$} (3) 
        (3) edge 
            node [left] {$\mathtt{return\ 0}$} (4);

      \path
        (1) edge[draw=none] node [left]
          {\color{OliveGreen}{\only<2->{\Ccancel[red]}{$\mathtt{r := mc91(n)}$}}} 
        (2);
    \uncover<1>{
      \path (1) edge (2);
    }
    \uncover<2->{
      \path
        (1) edge [bend left]
            node [above] {$\mathtt{m_1 := n}$} (00)
        (00) edge [bend right]
            node [left] {$\mathtt{assume\ m_1 > 100}$} (01)
            edge [bend left=30]
            node [right] {$\mathtt{assume\ not(m_1 > 100)}$} (02)
        (01) edge [bend right]
            node [left] {$
              \begin{array}{c}
                \mathtt{r^{mc91} :=}\\
                \mathtt{m_1 - 10}
              \end{array}$} (05)
        (02) edge 
            node [right] {
              \only<2>{$\mathtt{s_1 := mc91(m_1 + 11)}$
              }\only<3->{\color{red}$\mathtt{assume\ false}$}
            } (03)
        (03) edge 
            node [right] {
              \only<2>{$\mathtt{t_1 := mc91(s_1)}$
              }\only<3->{\color{red}$\mathtt{assume\ false}$}
            } (04)
        (04) edge [bend left]
            node [right] {$\mathtt{r^{mc91}_1 := t_1}$} (05)

        (05) edge [bend left]
             node [below=8pt] {$\mathtt{r := r^{mc91}_1}$} (2);
    }
    \end{tikzpicture}
  }
  %\caption{Under-approximation of McCarthy 91}
  \label{figure:under-mccarthy91}
\end{figure}
\end{overlayarea}
\end{frame}


\begin{frame}[fragile]{Transformed C Program}

  \begin{columns}[T]
    \begin{column}[T]{.4\linewidth}
      \begin{minted}{c}
int main() {
  int n, r;
  assume(n >= 0);

  /* r = mc91(n); */
  /* Unwind Begin */{
  ...
  }/* Unwind End */
  assert(r == 91 || 
    (n>101 && r==n-10));

  return 0;
}
      \end{minted}
    \end{column}

    \begin{column}[T]{.4\linewidth}
      \begin{minted}{c}
/* Unwind Begin */{
  int r_mc91_1, m_1 = n;
  int s_1, t_1;
  if(m_1 > 100){
    r_mc91_1 = m_1 - 10;
    goto RET_1;
  }/* else */
  assume(0);
  assume(0);
  r_mc91_1 = t_1;
  goto RET_1;
RET_1:;
  r = r_mc91_1;
}/* Unwind End */
      \end{minted}
    \end{column}
  \end{columns}

\end{frame}

\begin{frame}{Much More Accurate Approximation}

  Unwind more times and Replace at last
\begin{figure}
  \centering
  \resizebox{1\textwidth}{!}{
    \begin{tikzpicture}[->,>=stealth',shorten >=1pt,auto,node
      distance=2cm,thick,node/.style={circle,draw}]
      \node[node, label=above:$\mathtt{main()}$] (0s) at (-4, 0) {$s$};
      \node[node] (01) at (-4, -1) {$1$};
      \node[node] (02) at (-4, -2) {$2$};
      \node[node] (03) at (-4, -3) {$3$};
      \node[node] (0e) at (-4, -4) {$e$};

      \node[node] (s) at (-1,  0) {$s_1$};
      \node[node] (1) at (-2, -2) {$1_1$};
      \node[node] (2) at ( 1, -0.8) {$2_1$};
      \node[node] (3) at ( 1, -2) {$3_1$};
      \node[node] (4) at ( 1, -3.2) {$4_1$};
      \node[node] (e) at (-1, -4) {$e_1$};

      \node[node] (s') at ( 5,  2.6) {$s_2$};
      \node[node] (1') at ( 4,  0.6) {$1_2$};
      \node[node] (2') at ( 6,  1.8) {$2_2$};
      \node[node] (3') at ( 6,  0.6) {$3_2$};
      \node[node] (4') at ( 6, -0.6) {$4_2$};
      \node[node] (e') at ( 5, -1.4) {$e_2$};

      \node[node] (s'') at ( 5, -2.6) {$s_3$};
      \node[node] (1'') at ( 4, -4.6) {$1_3$};
      \node[node] (2'') at ( 6, -3.4) {$2_3$};
      \node[node] (3'') at ( 6, -4.6) {$3_3$};
      \node[node] (4'') at ( 6, -5.8) {$4_3$};
      \node[node] (e'') at ( 5, -6.6) {$e_3$};

      \path
        (0s) edge 
            node [left] {$\mathtt{assume\ n >= 0}$} (01)
        (01) edge [bend left=15]
            node [above=2pt] {$\mathtt{m_1 := n}$} (s)
        (02) edge 
            node [left] {$
              \begin{array}{l}
                \mathtt{assert\ {[}r = 91\ or}\\
                \mathtt{\ \ \ \ \ \ \ \ \ \ \ (n > 101\ and\ \ }\\
                \mathtt{\ \ \ \ \ \ \ \ \ \ \ \ r = n - 10){]}}
              \end{array}$} (03) 
        (03) edge 
            node [left] {$\mathtt{return\ 0}$} (0e)


        (s) edge [bend right=30]
            node [right] {$
              \begin{array}{c}
                \mathtt{assume}\\ 
                \mathtt{m_1 > 100}                
              \end{array}$} (1)
            edge [bend left=60]
            node [above] {$
              \begin{array}{l}
                \mathtt{assume}\\
                \mathtt{not(m_1 > 100)\ }
              \end{array}$} (2)
        (1) edge [bend right=30]
            node [above right] {$
              \begin{array}{c}
                \mathtt{r^{mc91}_1 :=}\\
                \mathtt{m_1 - 10}
              \end{array}$} (e)
        (2) edge [bend left=50]
            node [above left] {$\mathtt{m_2 := m_1 + 11}$} (s')
        (3) edge 
            node [below] {$\mathtt{m_3 := s_1}$} (s'')
        (4) edge [bend left=30]
            node [below] {$\mathtt{r^{mc91}_1 := t_1}$} (e)
        (e) edge [bend left=15]
            node [below=9pt] {$\mathtt{r := r^{mc91}_1}$} (02)

        (s') edge [bend right=30]
             node [left] {$
               \begin{array}{c}
                 \mathtt{assume}\\
                 \mathtt{m_2 > 100}
               \end{array}$} (1')
             edge [bend left=30]
             node [right] {$\mathtt{assume\ not(m_2 > 100)}$} (2')
        (1') edge [bend right=30]
             node [left] {$\mathtt{r^{mc91}_2 := m_2 - 10}$} (e')
        (2') edge 
             node [right] {\color{red}$\mathtt{assume\ false}$} (3')
        (3') edge 
             node [right] {\color{red}$\mathtt{assume\ false}$} (4')
        (4') edge [bend left=30]
            node [right] {$\mathtt{r^{mc91}_2 := t_2}$} (e')
        (e') edge 
             node [above] {$\mathtt{s_1 := r^{mc91}_2}$} (3)

        (s'') edge [bend right=30]
              node [left] {$\mathtt{assume\ m_3 > 100}$} (1'')
              edge [bend left=30]
              node [right] {$\mathtt{assume\ not(m_3 > 100)}$} (2'')
        (1'') edge [bend right=30]
              node [left] {$
                \begin{array}{c}
                \mathtt{r^{mc91}_3 :=}\\
                \mathtt{m_3 - 10}  
                \end{array}
                $} (e'')
        (2'') edge 
              node [right] {\color{red}$\mathtt{assume\ false}$} (3'')
        (3'') edge 
              node [right] {\color{red}$\mathtt{assume\ false}$} (4'')
        (4'') edge [bend left=50]
              node [right] {$\mathtt{r^{mc91}_3 := t_3}$} (e'')
        (e'') edge [bend left=50]
              node [below left] {$\mathtt{t_1 := r^{mc91}_3}$} (4)
        ;
    \end{tikzpicture}
    }
  %\caption{Unwinding McCarthy 91}
  \label{figure:unwind-mccarthy91}
  \vspace{-1cm}
\end{figure}
\end{frame}

\begin{frame}{Alternative Unwinding Method}

\alert{TODO}

Describe the alternative method, and how it can prevent size explosion

\end{frame}
