\stepcounter{subsection}

\begin{frame}{Brief Summary}
  Algorithm
  \begin{itemize}
    \item Use established analyzers as \textsc{BasicAnalyzer}
    \item Enhance \textsc{BasicAnalyzer} via syntactic transformation
  \end{itemize}
  \vfill
  Proof
  \begin{itemize}
    \item Use proof of under-approximation as summary candidates
    \item Verify candidates using Hoare proof rule for recursion  
  \end{itemize}
  \vfill
  Experiment Result
  \begin{itemize}
    \item Our approach is as good as tools specialized for recursion
  \end{itemize}
\end{frame}

\begin{frame}{Advantages}
  Lightweight
  \begin{itemize}
    \item Avoid hacking into \textsc{BasicAnalyzer}
    \item Syntactic transformation only
    \item Much less efforts on implementation
  \end{itemize}
  \vfill
  Modular
  \begin{itemize}
    \item Standard functionality from \textsc{BasicAnalyzer}
    \item Benefit from future advanced analyzers
  \end{itemize}
\end{frame}

\begin{frame}{Work in progress}
  Exponential growth of size of transformed program
  \begin{itemize}
    \item Due to unwinding recursive calls in each iteration
    \item Better transformation for analyzers supporting function call
  \end{itemize}
  \vfill
  Participating Competition on Software Verification 2015
  \begin{itemize}
    \item \textsc{CPArec} is freely available on GitHub
    \item \href{https://github.com/fmlab-iis/cparec}{https://github.com/fmlab-iis/cparec}
  \end{itemize}
\end{frame}

\begin{frame}{Future Work}

Can this technique handle other kinds of program features, e.g.~Pointer, Concurrency, etc.?

\tikzstyle{decision} = [diamond, aspect=2, draw, align=center]
\tikzstyle{data} = [draw,trapezium,trapezium left angle=70,trapezium right angle=-70,minimum height=0.6cm]
\tikzstyle{input} = [data, text width=1.7cm]
\tikzstyle{output} = [rounded corners, draw, text width=1.5cm, align=center,minimum height=0.7cm]
\tikzstyle{cross_out} = [red, pos=0.5, auto=false, cross out, -, draw, thick]

\begin{figure}
\begin{tikzpicture}[node distance=1cm and 1.2cm, auto,>=latex', thick]
  % We need to set at bounding box first. Otherwise the diagram
  % will change position for each frame.
  \useasboundingbox (-5.7,-1) rectangle (4.3,4);

  % Nodes
  \node[decision] (analyzer) {Analyzer};
  \node[input, left=of analyzer] (rf_prog) {Program w/o feat.};

  \node[input] (rc_prog) at (rf_prog |- check) {Program w/ feat.};

  \node[output, right=of analyzer] (unsafe) {ERROR};
  
  \node[output] (safe) at (check -| unsafe) {PASS};


  % Edges
  \path[->]
    (rf_prog) edge (analyzer)
    (analyzer) edge node[anchor=north,sloped]{Error} (unsafe);
  \path[->]
    (analyzer) edge node[anchor=south,sloped]{Pass} (safe);

  \path[->] (rc_prog) edge node [cross_out]{} (analyzer);

\end{tikzpicture}
\end{figure}

\end{frame}