
%%% Template originaly created by Karol Kozioł (mail@karol-koziol.net) and modified for ShareLaTeX use

\documentclass[a4paper,11pt]{article}

\usepackage{amsmath}
\usepackage[T1]{fontenc}
\usepackage[utf8]{inputenc}
\usepackage{graphicx}
\usepackage{xcolor}

\usepackage{sansmath}
\renewcommand\familydefault{\sfdefault}
\usepackage{tgheros}

\usepackage{amsmath,amssymb,amsthm,textcomp}
\usepackage{enumerate}
\usepackage{multicol}
\usepackage{tikz}
\usetikzlibrary{shapes, positioning}

\usepackage{geometry}
\geometry{total={210mm,297mm},
left=25mm,right=25mm,%
bindingoffset=0mm, top=20mm,bottom=20mm}


\linespread{1.3}

\newcommand{\linia}{\rule{\linewidth}{0.5pt}}

% custom theorems if needed
\newtheoremstyle{mytheor}
    {1ex}{1ex}{\normalfont}{0pt}{\scshape}{.}{1ex}
    {{\thmname{#1 }}{\thmnumber{#2}}{\thmnote{ (#3)}}}

\theoremstyle{mytheor}
\newtheorem{defi}{Definition}

% my own titles
\makeatletter
\renewcommand{\maketitle}{
\begin{center}
\vspace{2ex}
{\huge \textsc{\@title}}
\vspace{1ex}
\\
\linia\\
\@author \hfill \@date
\vspace{4ex}
\end{center}
}
\makeatother
%%%

% custom footers and headers
\usepackage{fancyhdr}
\pagestyle{fancy}
\lhead{}
\chead{}
\rhead{}
\lfoot{Automatic~Verification Assignment \#2}
\cfoot{}
\rfoot{Page \thepage}
\renewcommand{\headrulewidth}{0pt}
\renewcommand{\footrulewidth}{0pt}
%

% all section titles centered and bolded
\usepackage{sectsty}
\allsectionsfont{\bfseries\large}
%
% add section label
\renewcommand\thesection{Problem~\arabic{section}:}
%

%%%----------%%%----------%%%----------%%%----------%%%

\begin{document}

\title{Homework Assignment~\#2}

\author{R02943142 Hsieh, Chiao}

%\date{01/01/2014}

\maketitle

\section{Lattices and Complete Lattices}
Prove that every finite lattice is a complete lattice. Be clear about the cases
of the supremum and the infimum of an empty subset.
\medskip \\
Answer.
\smallskip \\
To prove that every finite lattice is a complete lattice, we have to prove that
there are $\bigvee S$ and $\bigwedge S$ for all $S \subseteq L$ in any given
finite lattice $L$.

First, we can derive $\bigwedge S$ for all nonempty subsets $S$ by following means:
\smallskip\\
Assume $S=\{x_1,x_2,\dots,x_n\}$ is a nonempty subset of a lattice $L$, we derive
$y_n$ by
\begin{align*}
L \ni y_0 &= x_1 \\
L \ni y_1 &= y_0 \wedge x_1 = x_1 \wedge x_1 = x_1 \\
L \ni y_2 &= y_1 \wedge x_2 = x_1 \wedge x_2 \\
L \ni y_3 &= y_2 \wedge x_3 = x_1 \wedge x_2 \wedge x_3 \\
\dots \\
L \ni y_n &= y_{n-1} \wedge x_n = x_1 \wedge x_2 \wedge x_3 \wedge \dots \wedge x_n
\end{align*}
By definition of $\wedge$, we know $y_i \leq x_i$ and $y_i \leq y_{i-1}$
and, therefore, $y_n \leq x_i$ for all $i$ in $[1 \dots n]$. This means
$y_n$ is a lower bound of $S$. Further, we prove $y_n$ is $\bigwedge S$.
\smallskip\\
Given $z \in L$, any lower bound of $S$,
\begin{align*}
y_n \wedge z &= y_{n-1} \wedge (x_n \wedge z) = y_{n-1} \wedge z \\
y_{n-1} \wedge z &= y_{n-2} \wedge (x_{n-1} \wedge z) = y_{n-2} \wedge z \\
\dots \\
y_0 \wedge z     &= x_1 \wedge z = z
\end{align*}
That is, for any lower bound $z$, $z \wedge y_n = z \implies z \leq y_n$;
hence $y_n$ is the greatest lower bound $\bigwedge S$.
Dually we can derive $\bigvee S$ in a similar manner.

Second, when $S = \emptyset$, we have to prove $L$ has a top element, 
$\top = \bigvee L = \bigwedge S$ and a bottom element, 
$\bot = \bigwedge L = \bigvee S$. The proof is directly derived from 
that, since $L \subseteq L$ and $L$ is finite, $\bigvee L$ and $\bigwedge 
L$ must exist by the proof in first paragraph. 

Combining both paragraphs, we proved that there are $\bigvee S$ and
$\bigwedge S$ for all $S \subseteq L$ in every finite lattice $L$.
Thus, every finite lattice is a complete lattice. Q.E.D.

\section{Complete Partial Orders}
Prove that every complete lattice is a complete partial order.
\medskip \\
Answer.
\smallskip \\
To prove that every complete lattice $CL$ is a complete partial order, it 
suffices to show that:
\begin{enumerate}
\item $CL$ has a bottom element $\bot$ and
\item $\bigsqcup D$ exists for each directed subset $D$ of $CL$
\end{enumerate}
The first condition is mentioned in Problem 1 that every complete lattice
has a bottom element. The second condition is trivially fulfilled by the
definition of complete lattice. Since, by definition, $\bigvee S$ exists 
for every subset $S \subseteq CL$, $\bigsqcup D$ must exist for directed 
subsets $D$ of $CL$. Hence, we finished the proof. Q.E.D.

\section{Continuous Maps}
For a discrete ordered set of your choice, find a self-map on the set
(i.e., a function mapping from the set to itself) that is monotonic
(order-preserving), but not $\sqcup$-continuous. Please state monotonicity
and $\sqcup$-continuity precisely in terms of the chosen ordered set
before presenting the example self-map and explaining why it meets the
requirements.
\medskip \\
Answer.
\smallskip \\
$\mathbb{N}_{\top} = \mathbb{N} \oplus \{\top\}$ is a discrete infinite
(complete partial order) set. The elements in $\mathbb{N}$ follow the
order in natural number, and all numbers are less than $\top$, a manually
introduced top element. \\
We can then define a monotonic self-map $f: \mathbb{N}_{\top} \mapsto \mathbb{N}_{\top}$ s.t.
\begin{equation*}
f(x) = \left\{
  \begin{array}{lr}
    1  & : x \in \mathbb{N} \\
    2  & : x = \top
  \end{array}
\right.
\end{equation*}
The mapping $f$ preserves order since for all $x, y \in \mathbb{N}, 
x \leq y \implies 1 = f(x) \leq f(y) = 1$, and for all $x \in \mathbb{N},
x \leq \top \implies 1 = f(x) \leq f(\top) = 2$.

The mapping $f$ is not $\sqcup$-continuous. We can disprove
by examining the subset $\mathbb{N} \subseteq \mathbb{N}_{\top}$. 
$\mathbb{N}$ is a directed subset because, for any two elements $x, y 
\in \mathbb{N}$, we know either $x \leq y$ or $y \leq x$ and, in either 
case, $\exists z = x \vee y$ s.t. $z \in \mathbb{N}$ and $z \in \{x, 
y\}^u$. However, $f(\bigsqcup \mathbb{N}) = f(\top) = 2 \neq 1 = 
\bigsqcup \{1,\dots,1\} = \bigsqcup f(\mathbb{N})$.
Therefore, $f$ is not a $\sqcup$-continuous mapping.

\end{document}